\begin{frame}{Forma A-Normal}
    \begin{itemize}
        \item A forma A-normal (\textit{A-normal form}, ou ANF) é uma versão de cálculo-$\lambda$ comumente utilizada como IR.

        \item Um $\lambda$-termo estará em ANF se não puderem ser aplicadas mais reduções do tipo A. \citeonline{sabyr1992} definem o conjunto de reduções A pelas regras:
              \begin{flushleft}
                  $(\lambda x.V x) \longrightarrow V  \hspace{21ex} x \not\in FV(V) \hspace{4ex} (\eta_{V})$

                  $E[((\lambda x.M) N)] \longrightarrow ((\lambda x.E[M]) N)  \hspace{3ex} x \not\in FV(E) \hspace{5ex} (\beta_{lift})$

                  $E[((M N) L)] \longrightarrow ((\lambda x.E[L]) (M N))  \hspace{2ex} x \not\in FV(E,L) \hspace{3ex} (\beta_{flat})$

                  $((\lambda x.x) M) \longrightarrow M \hspace{33ex} (\beta_{id})$

                  $((\lambda x.E[(y x)]) M) \longrightarrow E[(y M)] \hspace{6ex} x \not\in FV(E[y]) \hspace{3ex} (\beta_{\Omega})$
              \end{flushleft}
    \end{itemize}
\end{frame}