\begin{frame}{Teoria de Tipos}
    Russel em 1908 apresentou uma contradição na Teoria de Conjuntos\\
    \textbf{Paradoxo de Russell:}
    \begin{equation}
        \text{ Seja } R = \{ x \mid x \notin x \}, \text{ então } R \in R \iff R \notin R\nonumber
    \end{equation}

    Outra maneira de descrever este paradoxo é com o paradoxo do barbeiro:\\
    \qquad{}\qquad{}Imagine uma cidade com apenas um barbeiro, onde ele somente barbeia aqueles que não se barbeiam
\end{frame}

\begin{frame}{Teoria de Tipos}
    \textbf{Aplicações:}\\
    \begin{itemize}
        \item Formalização de sistemas de tipos para linguagens de programação
        \item Construção de assistentes de provas
              \begin{itemize}
                  \item[$\blacktriangleright$] Coq utiliza Cálculo de Construções~\cite{COQUAND1998}
              \end{itemize}
        \item Linguagens como Idris e Agda, também permitem a verificação de provas formais
    \end{itemize}
\end{frame}

\begin{frame}{Teoria de Tipos}
    Em linguagens de programação~\cite{PIERCE2002}:
    \begin{itemize}
        \item Tipos simples
              \begin{itemize}
                  \item[$\blacktriangleright$] Tipo fixo a um termo
                  \item[$\blacktriangleright$] $Int \rightarrow Int$
              \end{itemize}
        \item[]
        \item Tipos polimórficos
              \begin{itemize}
                  \item[$\blacktriangleright$] Generalidade
                  \item[$\blacktriangleright$] $a \rightarrow a$
              \end{itemize}
        \item[]
        \item Tipos dependentes
              \begin{itemize}
                  \item[$\blacktriangleright$] Tipos dependem de valores
                  \item[$\blacktriangleright$] $Vector(n) \rightarrow Vector(m) \rightarrow Vector(n+m)$
              \end{itemize}
    \end{itemize}
\end{frame}

\begin{frame}{Teoria de Tipos}
    \citeonline{PIERCE2002} define duas variedades de polimorfismo:
    \begin{itemize}
        \item Paramétrico
              \begin{itemize}
                  \item[$\blacktriangleright$] Única definição de função genérica
                  \item[$\blacktriangleright$] Função identidade
              \end{itemize}
        \item[]
        \item Com sobrecarga
              \begin{itemize}
                  \item[$\blacktriangleright$] Múltiplas implementações de uma função
                  \item[$\blacktriangleright$] Sobrecarga de operadores
              \end{itemize}
    \end{itemize}
\end{frame}

% Se sobrar tempo adicionar exemplos dos polimorfismos ^

\begin{frame}{Teoria de Tipos}
    \textbf{Cálculo Lambda Simplesmente Tipado:}\\
    \begin{itemize}
        \item Variante do Cálculo-$\lambda$ que incorpora tipos~\cite{CHURCH1940}
        \item Cada função recebe e retorna valores de tipos específicos
    \end{itemize}
    Sua sintaxe básica inclui:

    \begin{itemize}
        \item Variáveis: $x, y, z, \ldots$
        \item Tipos: $T ::= \textbf{Int} \mid \textbf{Bool} \mid T \to T$
        \item Termos: $\lambda x:T. \tau \mid \tau_1 \tau_2 \mid x$
        \item[]
    \end{itemize}

    Regra de tipagem para abstrações lambda:
    \[
        \frac{\Gamma, x:T_1 \vdash \tau:T_2}{\Gamma \vdash (\lambda x:T_1. \tau): T_1 \to T_2}\nonumber
    \]
    % Se o termo $\tau$ possui o tipo $T_2$ sob o contexto onde $x$ possui o tipo $T_1$, então a abstração $\lambda x:T_1. \tau$ tem o tipo $T_1 \to T_2$\\
\end{frame}

% Se sobrar tempo adicionar explicação dos termos ^

\begin{frame}{Teoria de Tipos}
    \textbf{Correspondência Curry-Howard:}
    \begin{itemize}
        \item Correspondência entre proposições intuicionistas lógicas e tipos
        \item Correspondência entre provas e programas
        \item O tipo $A \to B$ neste cálculo pode ser visto como a implicação lógica: se $A$, então $B$
        \item Método sistemático para raciocinar sobre sistemas de inferência de tipos
              \begin{itemize}
                  \item[$\blacktriangleright$] Linguagem ML por~\citeonline{DAMAS1982}, com o algoritmo W
                  \item[$\blacktriangleright$] Linguagem Haskell com a extensão do sistema Damas-Milner
              \end{itemize}
    \end{itemize}
\end{frame}
