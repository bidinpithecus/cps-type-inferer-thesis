\begin{frame}{Estilo de Passagem de Continuação (CPS)}
  \textbf{Cálculo Lambda:}\\
  \citeonline{church1932set} define o cálculo-$\lambda$, que é representado pela seguinte gramática:

  \begin{equation}
    e ::= x \mid \lambda x. e \mid e e\nonumber
  \end{equation}

  \begin{itemize}
    \item \textbf{Variável:} identificadores no sistema
    \item \textbf{Abstração:} função que associa um identificador $x$ a um termo $e$
    \item \textbf{Aplicação:} aplicação de um termo a outro
  \end{itemize}
\end{frame}

\begin{frame}{Estilo de Passagem de Continuação (CPS)}
  Variáveis no cálculo-$\lambda$ podem ser:
  \begin{itemize}
    \item \textbf{Livres:} quando não estão associadas a uma abstração de função
          \begin{itemize}
            \item[--] $\lambda x. y$
          \end{itemize}
    \item \textbf{Ligadas:} quando estão associadas a uma abstração de função
          \begin{itemize}
            \item[--] $(\lambda x. x) y$
          \end{itemize}
    \item[]
  \end{itemize}

  Para analisar expressões:
  \begin{itemize}
    \item \textbf{$\alpha$-redução:} Renomeação de variáveis ligadas.
          \begin{align}
            \lambda x . e[x] & \rightarrow \lambda y . e[y]\nonumber
          \end{align}

    \item \textbf{$\beta$-redução:} Aplicação de função.
          \begin{align}
            (\lambda x . e_1) e_2 & \rightarrow e_1 [e_2 / x]\nonumber
          \end{align}

    \item \textbf{$\eta$-redução:} Expansão de função.
          \begin{align}
            \lambda x . (e \, x) & \rightarrow e \quad \text{se } x \text{ não ocorre livre em } e\nonumber
          \end{align}
  \end{itemize}
\end{frame}