\chapter{Proposta}\label{ch:proposta}

Com base na tese de doutorado de \citeonline{thielecke1997}, \citeonline{TORRENS2024} propuseram uma semântica operacional para o Cálculo de Continuações baseada em reescrita de termos, demonstrando sua confluência.
O presente trabalho, por sua vez, tem como objetivo investigar a viabilidade de formalizar um sistema de tipos, juntamente com um algoritmo de inferência para CPS, inspirado nas contribuições de \citeonline{TORRENS2024}.

A validação do algoritmo será realizada por meio de experimentos, mesmo que isso não constitua uma prova formal de sua correção.
Destaca-se que a prova de completude e consistência do sistema de tipos desenvolvido pode ser um trabalho futuro.

Os próximos passos do trabalho proposto envolvem a formalização do sistema de tipos, bem como o desenvolvimento de um algoritmo de inferência para esse sistema.
Por fim, serão realizados testes com expressões de tipos conhecidos para validar a implementação.
A seguir, apresenta-se o cronograma previsto para a próxima fase deste trabalho.

\begin{enumerate}
	\item Formalização do sistema de tipos;\label{enum-formalizacao-sistema-tipos}
	\item Formalização do algoritmo de inferência;\label{enum-formalizacao-algoritmo}
	\item Implementação em Haskell, conforme definido nas Etapas~\ref{enum-formalizacao-sistema-tipos} e~\ref{enum-formalizacao-algoritmo};
	\item Validação dos resultados.
\end{enumerate}

\begin{table}[htbp]
	\centering
	\begin{tabular}{|c|c|c|c|c|c|c|c|c|}
		\hline
		\multirow{2}{*}{\textbf{\small{Etapas}}} & \textbf{\small{2024/1}} & \multicolumn{6}{c|}{\textbf{\small{2024/2}}} \\
		\cline{2-8}
		& \textbf{Dez} & \textbf{Jan} & \textbf{Fev} & \textbf{Mar} & \textbf{Abr} & \textbf{Maio} & \textbf{Jun} \\
		\hline
		\textbf{\small{1}}  & \cellcolor{gray} & \cellcolor{gray} & \cellcolor{gray} &  &  &  & \\
		\hline
		\textbf{\small{2}}  &  &  & \cellcolor{gray} & \cellcolor{gray} &  &  & \\
		\hline
		\textbf{\small{3}}  &  &  &  & \cellcolor{gray} & \cellcolor{gray} & \cellcolor{gray} & \\
		\hline
		\textbf{\small{4}}  &  &  &  &  &  & \cellcolor{gray} & \cellcolor{gray}\\
		\hline
	\end{tabular}
	\caption{Cronograma Proposto para o TCC2}
\end{table}
% \section{Formalização}\label{sec:formalizacao}
% \section{Implementação}\label{sec:implementacao}

Partindo para a parte prática do trabalho, os módulos e funções serão apresentados de maneira gradual, de modo a facilitar o entendimento do fluxo inteiro do programa.
Vale destacar aqui que é feito também a implementação do sistema Hindley-Milner, porém, como este foi feito somente para poder usar o sistema de continuações de maneira mais cômoda, não será entrado em detalhes nos que dizem respeito à esse sistema de tipos.
Inicialmente, na Subseção~\ref{subsec:cps-adt}, será discutido sobre a maneira como foi representado o sistema de tipos.
Em sequência, a Subseção~\ref{subsec:cps-translations} trará detalhes sobre a implementação das funções de tradução de cálculo lambda simplesmente tipado para CPS.
Posteriormente, a Subseção~\ref{subsec:cps-inferer} irá tratar da inferência em si, juntamente da verificação do tipo.
Por fim, a geração de código abordada na Subseção~\ref{subsec:cps-code-gen} serve como uma maneira de se testar a tradução do código.

\subsection{Tipos de Dados}\label{subsec:cps-adt}

Para representar os comandos, bem como os tipos do sistema, foram utilizados tipos de dados algébricos (ADTs, do inglês \textit{algebraic data types}), disponíveis no Listing~\ref{code:cps-adt}.

\lstinputlisting[style=haskell, label=code:cps-adt, caption={Definição dos tipos de dados}]{Code/Type-Inferer/CPSTyping.hs}
Para os comandos, dois construtores podem ser observados, o \texttt{Jump} e o \texttt{Bind}, sendo responsáveis por construir respectivamente os comandos de \textit{Jump}, onde há um salto \texttt{Id} com \texttt{[Id]} parâmetros.
Sendo assim, o salto $k(x)$ seria representado por este ADT: $\mathtt{Jump\ k\ [x]}$.
E o comando \textit{Bind}, onde há outro \texttt{Command} definido recursivamente, a função \texttt{Id} com argumentos \texttt{[Id]} e por fim outro \texttt{Command} também definido recursivamente.
O \textit{bind} $\mathtt{let}\ k(x) = k(x)\ \mathtt{in}\ k(x)$, portanto seria definido pelo seguinte ADT: $\mathtt{Bind\ (Jump\ k\ [x])\ k\ [x]\ (Jump\ k\ [x])}$.

Os tipos, foram representados com dois tipos algébricos diferentes, um para os monotipos $\mathtt{CPSMonoType}$, e uma para os politipos $\mathtt{CPSPolyType}$.
As variáveis do monotipo são construídas a partir dos construtores $\mathtt{TVar\ Id}$ e $\mathtt{TInt}$, onde no primeiro, o $\mathtt{Id}$ serve para obter a variável de tipo atribuída àquela variável, enquanto que as funções que não retornam são representadas as partir do construtor de negação $\mathtt{TNeg\ [CPSMonoType]}$ sendo os argumentos dela definidos recursivamente sobre si.
O contexto por sua vez, é um tipo que utiliza o $\mathtt{Data.Map}$ disponível no pacote \textit{containers} para mapear uma variável para um tipo polimórfico.
As substituições são representadas utilizando o mesmo $\mathtt{Data.Map}$, onde desta vez é mapeado uma variável de tipo para um monotipo.

\subsection{Traduções}\label{subsec:cps-translations}
Para facilitar os testes efetuados e ainda tornar o fluxo de execução mais direto, foram implementas as funções de tradução de expressões e de tipos de acordo com as definições das Seções~\ref{subsec:cps-translation} e~\ref{subsec:typed-cps-translation}.

\lstinputlisting[style=haskell, label=cps:cbn-initial-cont, caption={Continuação inicial}]{Code/Type-Inferer/CPS_initial_cont.hs}
Para todas as computações, um contexto inicial precisa conter a continuação inicial.
Este será o objeto a ter seu tipo inferido.
Afim de praticidade, esta continuação será sempre a mesma, dada por `k', como mostrado no Código~\ref{cps:cbn-initial-cont}.

\lstinputlisting[style=haskell, label=cps:cbn-expr-translation, caption={Tradução das expressões para CBN}]{Code/Type-Inferer/CPS_CBN_expr_translation.hs}
No Código~\ref{cps:cbn-expr-translation}, são apresentadas as funções responsáveis para traduzir as expressões para CBN.
O ponto de partida desta computação será a função $\mathtt{cbnExprTrans \dblcolon Expr \to Command}$, onde ela irá receber a expressão em cálculo-$\lambda$ e retornará o comando traduzido.
Sua responsabilidade é criar as mônadas de estado que farão o controle do índice das variáveis frescas necessárias e chamar as funções de tradução passando a continuação inicial.

Tomando como exemplo a função identidade em cálculo-$\lambda$ ($\lambda x.\ x$), é possível perceber como os termos crescem em CPS.
Isto torna o desenvolvimento diretamente neste cálculo não apropriado, mas ainda, nota-se que a implementação da função de tradução é bastante direta em relação a sua definição formal.
Os únicos pontos de atenção são em relação à geração das variáveis frescas, mas que como foi dito anteriormente, não eram completamente necessários, visto que eles são ligados imediatamente.
Ao traduzir então a função, tem-se que o equivalente em CPS é o mostrado no Código~\ref{cps:id-cps-cbn}.
\lstinputlisting[style=haskell, label=cps:id-cps-cbn, caption={Tradução da função identidade em CBN}]{Code/Type-Inferer/CPS/id-cbn.cps}
Este comportamento fica ainda mais visível quando uma função um pouco maior é traduzida, por exemplo o numeral de Church dois ($\lambda f.\ \lambda x.\ f\ (f\ x)$).
Sua tradução portanto é mostrada no Código~\ref{cps:church-two-cps-cbn}.
\lstinputlisting[style=haskell, label=cps:church-two-cps-cbn, caption={Tradução do numeral de Church ``2'' em CBN}]{Code/Type-Inferer/CPS/church-two-cbn.cps}
De maneira semelhante, foram feitas as mesmas funções utilizadas no \textit{call-by-name}, porém adaptadas para o CBV, respeitando as diferenças presentes na definicão formal da função.
\lstinputlisting[style=haskell, label=cps:cbv-expr-translation, caption={Tradução das expressões para CBV}]{Code/Type-Inferer/CPS_CBV_expr_translation.hs}
A partir dessas diferenças nas definições, nota-se também particularidades nas traduções destas, por exemplo ao traduzir a mesma função identidade, em CBV, é obtido o resultado apresentado no Código~\ref{cps:id-cps-cbv}.
\lstinputlisting[style=haskell, label=cps:id-cps-cbv, caption={Tradução da função identidade em CBV}]{Code/Type-Inferer/CPS/id-cbv.cps}
Neste exemplo da função identidade, pouca diferença entre as duas estratégias de avaliação pode ser notada.
Isso se deve não ao tamanho da expressão, e sim dos elementos desta.
Neste caso, há somente uma abstração lambda com uma variável.
A seguir, é exibido novamente o numeral de Church dois, porém para o CBV, onde mais diferenças podem ser observadas.
O motivo disto é os elementos da função, que diferentemente da identidade, conta com mais construtores para representá-la, exibido no Código~\ref{cps:church-two-cps-cbv}.
\lstinputlisting[style=haskell, label=cps:church-two-cps-cbv, caption={Tradução do numeral de Church ``2'' em CBV}]{Code/Type-Inferer/CPS/church-two-cbv.cps}

Mostrado anteriormente no Código~\ref{cps:id-cps-cbn}, a expressão CPS resultante da tradução da função identidade em CBN difere da mesma traduzida em CBV, presente no Código~\ref{cps:id-cps-cbv}.
O mesmo pode ser observado ao traduzir a função identidade tipada.
Por exemplo, ao executar a função para a identidade em CBN, o tipo obtido é $\neg\neg(\neg\neg\alpha,\ \neg\alpha)$.
Já em CBV, para a mesma função, tem-se que o tipo traduzido é $(\alpha,\ \neg\alpha)$.

Para raciocinar sobre os tipos do sistema, tem que ser levado em consideração o que estes representam, contradições.
Ao tomar como exemplo o tipo resultante da função identidade em CPS a partir da tradução por valor (CBV), isto é, $(\alpha,\ \neg\alpha)$\footnote{Apesar de estar sendo traduzido o tipo da função identidade, como esta tradução é feita para os tipos simples, note que aqui, está sendo representado monomorficamente pois não há a generalização do $\alpha$.}, deve-se pensar que este representa o absurdo de ter $\alpha$ como argumento e $\neg\alpha$ como continuação, ao mesmo tempo.

\lstinputlisting[style=haskell, label=cps:cbn-type-translation, caption={Tradução dos tipos para CBN}]{Code/Type-Inferer/CPS_CBN_type_translation.hs}
Aqui no Código~\ref{cps:cbn-type-translation}, a função responsável por traduzir um tipo utilizando a estratégia \textit{call-by-name}, a função $\mathtt{cbvTypeTranslation \dblcolon LambdaMonoType \to CPSPolyType}$ irá receber um tipo simples no cálculo-$\lambda$ simplesmente tipado, e retornar um tipo polimórfico em CPS.
Perceba que na definição, o retorno era um tipo simples em CPS. 
Essa diferença é justificada ao se observar o corpo desta função, onde há a definição da função $\mathtt{cbvTrans}$.
Esta é quem efetivamente faz a computação, pois é nela que acontece o casamento de padrões para determinar o tipo sendo traduzido.
Ainda, esta função retorna um tipo polimórfico pelo fato de que em momento posterior a essa tradução, a subtipagem do tipo traduzido e do tipo inferido precisa ser verificada.

\lstinputlisting[style=haskell, label=cps:cbv-type-translation, caption={Tradução dos tipos para CBV}]{Code/Type-Inferer/CPS_CBV_type_translation.hs}
Para que essas duas funções de tradução de tipos (CBN e CBV) sejam usadas do mesmo modo, elas seguem a messma assinatura. 
Seus comportamentos são o mesmo, a diferir somente nas diferenças das definições da função, isto é, como é feita a tradução.

O sistema de tipos proposto neste trabalho, entretanto, é polimórfico, ou seja, aqui há a adição de variáveis de tipos quantificadas.
Sendo assim, esta função de tradução, se aplicando neste caso de uso, não necessariamente retornará o tipo mais geral de uma expressão, mas sempre um subtipo deste.

\subsection{Inferência}\label{subsec:cps-inferer}
% Explicar inferência
% \lstinputlisting[style=haskell, label=cps:typing, caption={Definição dos tipos de dados}]{Code/Type-Inferer/CPSTyping.hs}
% Explicar código

\subsection{Geração de Código}\label{subsec:cps-code-gen}
Uma vez que as provas de completude e consistência não estavam no escopo deste trabalho e a validação do algoritmo se deu por meio de testes, um conjunto considerável de testes foi construído.
Para isolar os testes, ou seja, testar as funções de maneira independente para garantir que, se a inferência apresentasse algum erro, fosse certo que o erro estaria na inferência e não por conta de uma tradução incorreta, dois teoremas apresentados em~\cite{plotkin1975call} foram utilizados.

Este teorema, chamado de Teorema da Simulação, válido tanto para \textit{call-by-name} quanto para \textit{call-by-value}, afirma que, dado um programa $M$ em cálculo-$\lambda$, o resultado de sua computação há de ser o mesmo que o resultado da computação da tradução para CPS com a função identidade.
Ou seja, simulando a execução do programa feito em cálculo-$\lambda$ no cálculo de continuações.
Ou então, formalmente, $Eval(M) = Eval([M] (\lambda x.\ x))$, onde a função $Eval$ é responsável por avaliar uma expressão, efetivamente a computando.

Foi implementada a simulação do cálculo-$\lambda$ no cálculo de continuações a partir das funções de tradução para programas que computam os numerais de Church.
Isto é feito computando a função $\lambda$ onde, é incrementado o valor que representa o número de aplicações feitas (essencialmente como um numeral de Church é computado), passando como continuação a função identidade.
Fazendo com que assim, a função lambda que representa um numeral de church é simulada pelo cálculo de continuações.

\lstinputlisting[style=haskell, label=cps:code-gen, caption={Geração de código para computação de numerais de Church}]{Code/Type-Inferer/CPS_code_gen.hs}
O código Haskell gerado pode ser dividido em três partes principais, o cabeçalho de caráter informativo, que explicita o programa em cálculo-$\lambda$ de entrada para ter gerado aquele programa em CPS.
Em seguida, há a definição das funções $\mathtt{cbn}$ e $\mathtt{cbv}$, ou seja, os programas correspondente em CPS para as duas traduções daquela entrada.
Por fim, a última parte é responsável pela computação do numeral de Church, as funções definidas irão calcular o número representado pelas expressões em CPS e retornar por fim uma tupla contendo o resultado do calculado pelo \textit{call-by-name} e \textit{call-by-value} respectivamente.

Ao se traduzir o numeral de Church 0 representado em cálculo-$\lambda$ por $\lambda f.\lambda x.\ x$, para CBN e CBV, tem-se:
\lstinputlisting[style=haskell, label=cps:church-zero-cps-cbn, caption={Tradução do numeral de Church ``0'' em CBN}]{Code/Type-Inferer/CPS/church-zero-cbn.cps}
\lstinputlisting[style=haskell, label=cps:church-zero-cps-cbv, caption={Tradução do numeral de Church ``0'' em CBV}]{Code/Type-Inferer/CPS/church-zero-cbv.cps}
Desta forma, o código gerado ao se traduzir esta função, é ilustrado a seguir:
\lstinputlisting[style=haskell, label=cps:church-zero-output, caption={Código gerado ao traduzir o numeral de Church ``0''}]{Code/Type-Inferer/CPS/church-zero.hs}
Ao executar o código e chamar a função $\mathtt{main}$ deste programa, o resultado obtido é justamente a computação do numeral para as duas traduções, ou seja, $\mathtt{(0,\ 0)}$.

\subsection{Fluxo Principal}\label{subsec:cps-main-program}
O fluxo completo de execução do programa principal contempla todas as funções apresentadas nesta seção, com a adição de funções auxiliares.
Essas são aplicadas em sequência, de modo a realizar uma série de ações.

Inicialmente, é passado o caminho de um arquivo contendo um programa em cálculo-$\lambda$ com adição do `let'.
O conteúdo então é processado pelo \textit{parser} e representado pelos tipos de dados algébricos para o cálculo lambda.
Uma vez que o programa já está sendo representado pelos ADTs, e ainda tem seu tipo inferido, é possível iniciar o processamento descrito pelas funções apresentadas.
A primeira delas é a tradução para CPS tanto em \textit{call-by-name} quanto em \textit{call-by-value}, as exibindo logo em seguida.
Com as traduções da expressão feitas o código Haskell já pode ser gerado, salvando assim no diretório \texttt{output} com mesmo nome do arquivo de entrada.
Como passo posterior, tem-se a tradução dos tipos para ambas as estratégias de avaliação.
Os passos finais envolvem a inferência de ambas as traduções, juntamente da verificação de subtipagem, onde esta informará se o tipo traduzido é um subtipo do inferido.

\lstset{extendedchars=false, escapeinside=''}
\lstinputlisting[style=output, label=cps:main-execution, caption={Execução do programa principal}]{Code/Type-Inferer/CPS_main_execution.cps}
Ao executar o programa com o comando \texttt{cabal run}, passando também o arquivo de entrada \texttt{input/church-zero.in}, é processado e exibida todas as informações que foi citada anteriormente, inclusive a geração do código Haskell em \texttt{output/church-zero.hs}.
É possível perceber que na saída do programa, é mostrado o tipo traduzido (na sequência da mensagem ``\textit{Expected Continuation Type:}'') e o tipo inferido (que sucede a mensagem ``\textit{Inferred Continuation Type:}'').
Logo em seguida, o questionamento ``\textit{Do the types match?}'' é o trecho da saída que compete à subtipagem, retornando ``\textit{Yes}'' caso esse seja um subtipo deste, o que indica uma inferência compatível com a tradução ou ``\textit{No}'' caso a verificação falhe, indicando uma inferência incorreta.

\lstset{extendedchars=false, escapeinside=''}
\lstinputlisting[style=output, label=cps:execution-generated, caption={Execução do programa gerado}]{Code/Type-Inferer/CPS_execution_generated.cps}
Ainda, a compilação e execução do código Haskell gerado pode ser conferida no Código~\ref{cps:execution-generated} acima, ilustrando exatamente o comportamento detalhado anteriormente.


% \section{Conclusão}\label{sec:conclusao}
As IRs são muito importantes na compilação, elas permitem otimizar muitos processos para assim tornar mais eficiente o código.
Os sistemas de tipos permitem provas de propriedades de um sistema, garantindo que programas tenham comportamento esperado.
Por exemplo, uma função que opera sobre um conjunto numérico, utilizando um sistema de tipos, seria possível identificar um uso incorreto desta função ao passar um argumento de outro tipo, podendo assim impedir que a função compute para evitar um comportamento inesperado.
Sendo assim, pode-se dizer que o sistema de tipos adiciona uma camada de segurança ao programa.

Uma das maiores motivações de adicionar essa camada de segurança na representação intermediária, ou seja de desenvolver um sistema de tipos para o CPS, é extender a prova de programas para as etapas seguintes em que o sistema já não tem mais ciência dos tipos, ou seja, o ligador (do inglês \textit{linker}).
Considerando o cenário onde um programa possua milhares de arquivos fonte, com IRs não tipadas, as provas das propriedades de funções definidas em um arquivo somente são passadas para outra caso estes sejam compilados juntos -- o que se torna inviável em programas tão grandes.
Já com uma IR tipada, se uma função $f$ definida em um arquivo espera receber um inteiro for utilizada em outro módulo onde na verdade está sendo passado como argumento uma cadeia de caracteres, o erro seria facilmente identificado no momento da ligação (do inglês \textit{linking}), não havendo a necessidade de recompilar o arquivo que contém a função ou ainda todos os que o utilizam.

Mesmo que o sistema não funcione para o `let' no CBV, a tradução está correta, onde isso significa respeitar o teorema da simulação.
Isto é, no caso de uma implementação não tipada da tradução, não haveria problemas em simular a expressão lambda no cálculo de continuações.
Como a tradução para CBV não preserva tipos, um dos trabalhos futuros incluiria a correção deste.
Para tal, seria necessário propor um outro sistema de tipos polimórfico para o cálculo de continuações onde, sejam distinguidos tipos e cotipos dentre os argumentos para que o algoritmo de inferência saiba quando utilizar um ou outro.

Outro trabalho futuro são as provas de consistência e completude do sistema de tipos proposto em relação ao algoritmo de inferência.
Neste trabalho foram executados alguns casos de testes que demostraram empiricamente que o algoritmo está correto.
Em conjunto, um avanço seria a criação de um compilador que faça uso desta IR tipada, colocando em prática toda a teoria aqui apresentada.

