\section{Resultados}\label{sec:resultados}
Uma vez que todo o fluxo do programa foi exibido, seus resultados podem ser apresentados e compreendidos.
Diversos testes foram executados onde, a partir da função de tradução de tipos, puderam ter seus tipos verificados para validar a implementação.
Para tal, estão disponibilizados no repositório do projeto, arquivos de entrada com funções contendo diferentes características.
Afim de observar o comportamento do programa abrangendo uma maior gama de opções, algumas das funções testadas são extensas e contam com combinações de regras do cálculo-$\lambda$ simplesmente tipado.

\subsection{Combinador S}
Um desses, é o combinador S ($\lambda x.\lambda y.\lambda z.\ x\ z\ (y\ z)$), que apesar de não ser um termo extenso, este utiliza de uma combinação dos três construtores (variáveis, abstrações e aplicações) para que o termo seja optido.
O tipo deste, é representado por $(\alpha \to \beta \to \gamma) \to (\alpha \to \beta) \to \alpha \to \gamma$ e suas traduções, tanto em CBN quanto em CBV são grandes demais para serem aqui colocadas, mas estas podem ser encontradas no código fonte do repositório mencionado.
A tradução do seu tipo entretanto, para CBN, é apresentada a seguir.
\lstset{extendedchars=false, escapeinside=''}
\begin{lstlisting}[style=output,caption={Tradução em CBN do tipo do combinador S}]
  '$\forall\alpha,\beta,\gamma.\ \neg\neg(\neg\neg\neg(\neg\neg\alpha,\ \neg\neg(\neg\neg\beta,\ \neg\gamma)),\ \neg\neg(\neg\neg\neg(\neg\neg\alpha,\ \neg\beta),\ \neg\neg(\neg\neg\alpha,\ \neg\gamma)))$'
\end{lstlisting}
Enquanto que, a partir do tipo inferido a seguir para a mesma estratégia de avaliação, uma substituição $S$ tal que ao aplicá-la no tipo traduzido torne-se o inferido, é $S = \{\ \alpha \mapsto \neg\alpha,\ \beta \mapsto \neg\beta,\ \gamma \mapsto \neg\gamma\ \}$, validando assim a inferência para este termo.
\lstset{extendedchars=false, escapeinside=''}
\begin{lstlisting}[style=output,caption={Inferência do tipo do combinador S traduzido em CBN}]
  '$\forall\alpha,\beta,\gamma.\ \neg\neg(\neg\neg\neg(\neg\alpha,\ \neg\neg(\neg\beta,\ \gamma)),\ \neg\neg(\neg\neg\neg(\neg\alpha,\ \beta),\ \neg\neg(\neg\alpha,\ \gamma)))$'
\end{lstlisting}
Um comportamento semelhante pode ser percebido para a tradução por \textit{call-by-value}, onde respectivamente é apresentado a seguir a tradução e o resultado da inferência.
\lstset{extendedchars=false, escapeinside=''}
\begin{lstlisting}[style=output,caption={Tradução em CBV do tipo do combinador S}]
  '$\forall\alpha,\beta,\gamma.\ \neg\neg(\neg(\alpha,\ \neg\neg(\beta,\ \neg\gamma)),\ \neg\neg(\neg(\alpha,\ \neg\beta),\ \neg\neg(\alpha,\ \neg\gamma)))$'
\end{lstlisting}
Neste caso, a substituição $S$ que satisfaz a condição de subtipagem é tal que $S = \{\ \gamma \mapsto \neg\gamma\ \}$, tornando válida assim a inferência para este termo.
\lstset{extendedchars=false, escapeinside=''}
\begin{lstlisting}[style=output,caption={Inferência do tipo do combinador S traduzido em CBV}]
  '$\forall\alpha,\beta,\gamma.\ \neg\neg(\neg(\alpha,\ \neg\neg(\beta,\ \gamma)),\ \neg\neg(\neg(\alpha,\ \neg\beta),\ \neg\neg(\alpha,\ \gamma)))$'
\end{lstlisting}

\subsection{Soma}
O próximo exemplo apresentado é a função de soma de 2 e 3 feita com os numerais de Church ($(\lambda n.\ \lambda m.\ \lambda f.\ \lambda x.\ n\ f\ (m\ f\ x))\ (\lambda a.\ \lambda b.\ a\ (a\ b))\ (\lambda c.\ \lambda d.\ c\ (c\ (c\ d)))$).
Seu tipo, é representado por $((\alpha \to \alpha) \to \alpha \to \alpha)$.
A tradução do tipo para CBN é dado por:
\lstset{extendedchars=false, escapeinside=''}
\begin{lstlisting}[style=output,caption={Tradução em CBN do tipo da função de soma}]
  '$\forall\alpha.\ \neg\neg(\neg\neg\neg(\neg\neg\alpha,\ \neg\alpha),\ \neg\neg(\neg\neg\alpha,\ \neg\alpha))$'
\end{lstlisting}
E o inferido também para CBN, onde a substituição $S$ que satisfaz a subtipagem é tal que $S = \{\ \alpha \mapsto \neg\alpha\ \}$:
\lstset{extendedchars=false, escapeinside=''}
\begin{lstlisting}[style=output,caption={Inferência do tipo da função de soma traduzido em CBN}]
  '$\forall\alpha.\ \neg\neg(\neg\neg\neg(\neg\alpha,\ \alpha),\ \neg\neg(\neg\alpha,\ \alpha))$'
\end{lstlisting}
Enquanto que para CBV, o tipo traduzido é:
\lstset{extendedchars=false, escapeinside=''}
\begin{lstlisting}[style=output,caption={Tradução em CBV do tipo da função de soma}]
  '$\forall\alpha.\ \neg\neg(\neg(\alpha,\ \neg\alpha),\ \neg\neg(\alpha,\ \neg\alpha))$'
\end{lstlisting}
O inferido portanto, sendo que a substituição $S$ que satisfaz a subtipagem neste caso é a substituição trivial $S = \{\ \alpha \mapsto \alpha\ \}$:
\lstset{extendedchars=false, escapeinside=''}
\begin{lstlisting}[style=output,caption={Inferência do tipo da função de soma traduzido em CBV}]
  '$\forall\alpha.\ \neg\neg(\neg(\alpha,\ \neg\alpha),\ \neg\neg(\alpha,\ \neg\alpha))$'
\end{lstlisting}

\TODO{Apresentar exemplo que computa (soma)}
\subsection{Multiplicação}
\TODO{Apresentar exemplo que computa (multiplicação)}
\subsection{Exponenciação}
\TODO{Apresentar exemplo que computa (exponenciação)}

\subsection{Identidade com `Let'}
\TODO{Apresentar exemplo que dá ruim}
\TODO{Explicar o problema do let no cbv.}
\TODO{Discutir brevemente a solução.}
