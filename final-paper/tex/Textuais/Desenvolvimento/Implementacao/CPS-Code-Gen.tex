\subsection{Geração de Código}\label{subsec:cps-code-gen}
Uma vez que as provas de completude e consistência não estavam no escopo deste trabalho e a validação do algoritmo se deu por meio de testes, um conjunto considerável de testes foi construído.
Para isolar os testes, ou seja, testar as funções de maneira independente para garantir que, se a inferência apresentasse algum erro, fosse certo que o erro estaria na inferência e não por conta de uma tradução incorreta, dois teoremas apresentados em~\cite{plotkin1975call} foram utilizados.

Este teorema, chamado de Teorema da Simulação, válido tanto para \textit{call-by-name} quanto para \textit{call-by-value}, afirma que, dado um programa $M$ em cálculo-$\lambda$, o resultado de sua computação há de ser o mesmo que o resultado da computação da tradução para CPS com a função identidade.
Ou seja, simulando a execução do programa feito em cálculo-$\lambda$ no cálculo de continuações.
Ou então, formalmente, $Eval(M) = Eval([M] (\lambda x.\ x))$, onde a função $Eval$ é responsável por avaliar uma expressão, efetivamente a computando.

Foi implementado a simulação do cálculo-$\lambda$ no cálculo de continuações a partir das funções de tradução para programas que computam os numerais de Church.
Isto é feito computando a função $\lambda$ onde, é incrementado o valor que representa o número de aplicações feitas (essencialmente como um numeral de Church é computado), passando como continuação a função identidade.
Fazendo com que assim, a função lambda que representa um numeral de church é simulada pelo cálculo de continuações.

\lstinputlisting[style=haskell, label=cps:code-gen, caption={Geração de código para computação de numerais de Church}]{Code/Type-Inferer/CPS_code_gen.hs}
O código Haskell gerado pode ser dividido em três partes principais, o cabeçalho de caráter informativo, que explicita o programa em cálculo-$\lambda$ de entrada para ter gerado aquele programa em CPS.
Em seguida, há a definição das funções $\mathtt{cbn}$ e $\mathtt{cbv}$, ou seja, os programas correspondente em CPS para as duas traduções daquela entrada.
Por fim, a última parte é responsável pela computação do numeral de Church, as funções definidas irão calcular o número representado pelas expressões em CPS e retornar por fim uma tupla contendo o resultado do calculado pelo \textit{call-by-name} e \textit{call-by-value} respectivamente.

Ao se traduzir o numeral de Church 0 representado em cálculo-$\lambda$ por $\lambda f.\lambda x.\ x$, para CBN e CBV, tem-se:
\lstinputlisting[style=haskell, label=cps:church-zero-cps-cbn, caption={Tradução do numeral de Church ``0'' em CBN}]{Code/Type-Inferer/CPS/church-zero-cbn.cps}
\lstinputlisting[style=haskell, label=cps:church-zero-cps-cbv, caption={Tradução do numeral de Church ``0'' em CBV}]{Code/Type-Inferer/CPS/church-zero-cbv.cps}
Desta forma, o código gerado ao se traduzir esta função, é ilustrado a seguir:
\lstinputlisting[style=haskell, label=cps:church-zero-output, caption={Código gerado ao traduzir o numeral de Church ``0''}]{Code/Type-Inferer/CPS/church-zero.hs}
Ao executar o código e chamar a função $\mathtt{main}$ deste programa, o resultado obtido é justamente a computação do numeral para as duas traduções, ou seja, $\mathtt{(0,\ 0)}$.
