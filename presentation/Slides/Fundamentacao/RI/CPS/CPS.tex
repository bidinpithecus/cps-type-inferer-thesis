\begin{frame}{Estilo de Passagem de Continuação (CPS)}
    \begin{itemize}
        \item Técnica de transformação de código que torna o fluxo de controle explícito
              \begin{itemize}
                  \item[--] Chamadas de função passam o controle para a próxima etapa explicitamente, conhecida como continuação~\cite{appel1992compiling}
                  \item[--] Ao invés das funções retornarem o resultado da computação, é invocado uma continuação, representando o próximo passo
              \end{itemize}
        \item Toda chamada de função passa então a ser uma chamada de cauda (do inglês \textit{tail-call})
    \end{itemize}
\end{frame}

\begin{frame}{Estilo de Passagem de Continuação (CPS)}
  \textbf{Chamada de cauda:}
  \begin{itemize}
    \item Última instrução executada em uma função é uma chamada a outra função, sem que restem computações adicionais a serem feitas após essa chamada~\cite{MUCHNICK1997}
          \begin{itemize}
            \item[--] Função atual pode liberar seu quadro de ativação
          \end{itemize}
  \end{itemize}
  \textbf{Chamada não de cauda:}
  \begin{itemize}
    \item Ainda restam operações, como somas ou multiplicações, após a chamada da função
          \begin{itemize}
            \item[--] Função atual precisa manter seu quadro de ativação até que as operações sejam concluídas
          \end{itemize}
  \end{itemize}
\end{frame}

\begin{frame}{Estilo de Passagem de Continuação (CPS)}
  \begin{itemize}
    \item[] \begin{figure}
            \caption{Função fatorial em Haskell com chamada não de cauda}
            \lstinputlisting[style=haskell, label=code:factorial_non_tail_call]{Code/factorial_non_tail_call.hs}
            \small{Fonte: o autor}
          \end{figure}

    \item[] \begin{figure}
            \caption{Função fatorial em Haskell com chamada de cauda}
            \lstinputlisting[style=haskell, label=code:factorial_tail_call]{Code/factorial_tail_call.hs}
            \small{Fonte: o autor}
          \end{figure}
  \end{itemize}
\end{frame}

\begin{frame}{Estilo de Passagem de Continuação (CPS)}
  \textbf{Cálculo Lambda:}
  \citeonline{church1932set} define o cálculo-$\lambda$, que é representado pela seguinte gramática:

  \begin{equation}
    e ::= x \mid \lambda x. e \mid e e\nonumber
  \end{equation}

  \begin{itemize}
    \item \textbf{Variável:} identificadores no sistema
    \item \textbf{Abstração:} função que associa um identificador $x$ a um termo $e$
    \item \textbf{Aplicação:} aplicação de um termo a outro
  \end{itemize}
\end{frame}

\begin{frame}{Estilo de Passagem de Continuação (CPS)}
  Variáveis no cálculo-$\lambda$ podem ser:
  \begin{itemize}
    \item \textbf{Livres:} quando não estão associadas a uma abstração de função
          \begin{itemize}
            \item[--] $\lambda x. y$
          \end{itemize}
    \item \textbf{Ligadas:} quando estão associadas a uma abstração de função
          \begin{itemize}
            \item[--] $(\lambda x. x) y$
          \end{itemize}
    \item[]
  \end{itemize}

  Para analisar expressões:
  \begin{itemize}
    \item \textbf{$\alpha$-redução:} Renomeação de variáveis ligadas.
          \begin{align}
            \lambda x . e[x] & \rightarrow \lambda y . e[y]\nonumber
          \end{align}

    \item \textbf{$\beta$-redução:} Aplicação de função.
          \begin{align}
            (\lambda x . e_1) e_2 & \rightarrow e_1 [e_2 / x]\nonumber
          \end{align}

    \item \textbf{$\eta$-redução:} Expansão de função.
          \begin{align}
            \lambda x . (e \, x) & \rightarrow e \quad \text{se } x \text{ não ocorre livre em } e\nonumber
          \end{align}
  \end{itemize}
\end{frame}

\begin{frame}{Estilo de Passagem de Continuação (CPS)}
    \textbf{Transformação CPS:}\\
    No cálculo-$\lambda$ tradicional, o fluxo de execução é implícito
    \begin{itemize}
        \item Funções são aplicadas e os resultados são retornados
        \item $\lambda x. x + 1$
    \end{itemize}
    Já no CPS, o fluxo de execução é explícito
    \begin{itemize}
        \item Uma série de chamadas de funções passam o resultado para um argumento extra, a continuação, indicando o próximo passo da computação
        \item $\lambda x. \lambda k. k (x + 1)$
    \end{itemize}
\end{frame}
