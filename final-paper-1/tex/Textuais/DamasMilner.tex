\chapter{Sistema Damas-Milner}\label{ch:damas-milner}

O sistema Damas-Milner, introduzido por Robin Milner e posteriormente formalizado em maior detalhe por Luis Damas \cite{MILNER1978, DAMAS1982}, é um dos sistemas de tipos mais influentes para linguagens funcionais.
Este sistema tem como principal característica a inferência automática de tipos polimórficos, sem a necessidade de anotações explícitas por parte do programador, como ocorre em linguagens como ML, Haskell e OCaml.
A sua base é o cálculo lambda com polimorfismo paramétrico, permitindo que funções possam operar sobre múltiplos tipos de maneira genérica.

A introdução do sistema Damas-Milner trouxe duas contribuições principais: a definição de um sistema de tipos que é robusto, garantindo propriedades como \textit{soundness} (consistência semântica), e a criação de um algoritmo, o Algoritmo W, capaz de inferir o tipo mais geral (também chamado de \textit{principal type-scheme}), conforme demonstrado em \citeonline{DAMAS1984}.
Como resultado, a linguagem ML e suas derivadas se tornaram notórias por fornecer ao programador a capacidade de escrever programas sem erros de tipo detectáveis durante a compilação, permitindo um desenvolvimento mais seguro e robusto \cite{MILNER1978, DAMAS1984}.

A sintaxe do sistema DM define as expressões e os tipos usados no processo de inferência.
Abaixo, segue a gramática das expressões e tipos:

\begin{equation}\label{eq:dm-syntax}
  \begin{array}{ll}
    \text{Variáveis}         & x                                                                                 \\
    \text{Expressões}        & e ::= x \mid e \ e' \mid \lambda x.e \mid \texttt{let} \ x = e \ \texttt{in} \ e' \\
    \\
    \text{Variáveis de tipo} & \alpha                                                                            \\
    \text{Tipos primitivos}  & \iota                                                                             \\
    \text{Tipos}             & \tau ::= \alpha \mid \iota \mid \tau \rightarrow \tau
    \alpha                                                                                                       \\
    \text{Schemes}           & \sigma ::= \forall \alpha. \sigma \mid \tau                                       \\
  \end{array}
\end{equation}

Na sintaxe, $x$ representa variáveis que podem ser nomes de qualquer identificador, e $e$ descreve expressões que podem ser variáveis, aplicações de função, funções anônimas ou declarações \texttt{let}, que introduzem polimorfismo através de generalização de tipos.
As variáveis de tipo $\alpha$ são usadas para representar tipos genéricos.
Os tipos primitivos $\iota$ são usados para representar tipos constantes.
Tipos $\tau$ podem ser tanto variáveis de tipo quanto funções entre tipos.
Por fim, $\sigma$ denota \textit{schemes}, ou tipos polimórficos, que podem quantificar variáveis de tipo, permitindo reutilização de tipos genéricos em diferentes contextos.

O polimorfismo no sistema Damas-Milner é introduzido pelas expressões \texttt{let}, que permitem a generalização de tipos. Ao declarar uma variável ou função usando \texttt{let}, o tipo inferido é generalizado para ser reutilizado em diferentes contextos. Isso significa que, ao declarar uma função como $\texttt{id} = \lambda x.x$, o sistema deduz o tipo mais geral $\forall \alpha. \alpha \rightarrow \alpha$, que pode ser usado com diferentes tipos conforme necessário.

A inferência de tipos envolve dois processos principais: generalização e instanciação.
A generalização ocorre quando o sistema identifica que uma expressão pode ser tipada com um tipo mais geral, permitindo que seja reutilizada de maneira polimórfica.
Já a instanciação ocorre quando um tipo polimórfico é aplicado a um tipo concreto, especializando-o para um uso específico.
Esse mecanismo garante a flexibilidade do sistema, ao mesmo tempo que mantém a segurança garantida pela inferência de tipos.

Por exemplo, considere a expressão $\texttt{let id} \ = \lambda x.x \ \texttt{in} \ (\texttt{id} \ \texttt{1}, \ \texttt{id} \ \texttt{`a'})$.
O sistema generaliza o tipo de $\texttt{id}$ para $\forall \alpha. \alpha \rightarrow \alpha$, e instancia este tipo tanto para inteiros quanto para caracteres nas duas aplicações subsequentes.

% Substituições
Outro conceito presente no sistema Damas-Milner é a substituição de tipos, onde estes são mapeados para outros tipos ou para variáveis de tipo.
Formalmente, uma substituição de tipos é representada como um mapeamento finito de variáveis de tipo para tipos, denotado por $S$, e pode ser escrito na forma $[ \alpha_1 \mapsto \tau_1, \alpha_2 \mapsto \tau_2, \ldots, \alpha_n \mapsto \tau_n ]$.
Aqui, $\alpha_i$ são variáveis de tipo distintas e $\tau_i$ são os tipos correspondentes.
Em outras palavras, $S$ associa cada variável de tipo $\alpha_i$ a um tipo $\tau_i$ específico.

A aplicação de uma substituição $S$ em um tipo $\tau$, denotada por $S\tau$, resulta na substituição de todas as ocorrências livres de $\alpha_i$ em $\tau$ por $\tau_i$.
Esse conceito de substituição é fundamental para o processo de instanciação de tipos, que será discutido a seguir.
A definição formal da aplicação de substituições é dada por:

\begin{figure}[h!]
  \caption{Regras de aplicação da substituição de tipos}
  \centering
  \[
    \begin{array}{rcl}
      S\alpha_i                    & \equiv & \tau_i,                                                                          \\[8pt]
      S\alpha                      & \equiv & \alpha, \quad \text{se } \alpha \notin \{\alpha_1, \alpha_2, \ldots, \alpha_n\}, \\[8pt]
      S(\tau_1 \rightarrow \tau_2) & \equiv & S\tau_1 \rightarrow S\tau_2,                                                     \\[8pt]
      S(\forall \alpha.\sigma)     & \equiv & S' \sigma, \quad \text{onde } S' = S \setminus [\alpha \mapsto \_].
    \end{array}
  \]
  \label{fig:substituicao-tipos}
  \small{Fonte: \cite{CASTRO2019}}
\end{figure}

onde o símbolo de subtração ($\setminus$) indica que a substituição $S'$ é a substituição $S$ restrita ao conjunto de mapeamentos que não envolvem a variável $\alpha$.

A instanciação de tipos é um processo em que um esquema de tipo $\sigma = \forall \alpha_1 \ldots \alpha_m. \tau$ é transformado em um tipo específico substituindo suas variáveis quantificadas por tipos concretos.
Se $S$ é uma substituição, então $S\sigma$ é o esquema de tipo obtido substituindo cada ocorrência livre de $\alpha_i$ em $\sigma$ por $\tau_i$, renomeando as variáveis genéricas de $\sigma$, se necessário.
O tipo resultante $S\sigma$ é chamado de uma instância de $\sigma$ \cite{DAMAS1982}.
Esse processo é essencial para adaptar esquemas de tipos polimórficos a situações específicas em um programa, mantendo a flexibilidade e segurança do sistema de tipos.

Um esquema de tipo também pode ter uma instância genérica $\sigma' = \forall \beta_1 \ldots \beta_n. \tau'$, se existir uma substituição $[ \tau_i / \alpha_i ]$ tal que $\tau' = [\tau_i / \alpha_i]\tau$, e as variáveis $\beta_j$ não aparecem livres em $\sigma$.
Nesse caso, escrevemos $\sigma > \sigma'$, indicando que $\sigma$ é mais geral do que $\sigma'$.
Vale notar que a instanciação atua sobre variáveis livres, enquanto a instanciação genérica lida com variáveis ligadas.

O sistema de tipos de Damas e Milner conta com um conjunto de regras de inferência de tipos, apresentadas na Figura \ref{eq:type-inference}, que são usadas para determinar os tipos das expressões no sistema.
Essas regras são julgamentos de tipos da forma $\Gamma \vdash e: \sigma$, onde $\Gamma$ é o contexto, um conjunto de suposições com pares de variáveis e seus tipos: $(e, \sigma)$, $e$ é a expressão sendo tipada, e $\sigma$ é o tipo inferido no contexto.

\begin{figure}[h!]
  \caption{Regras de Inferência do sistema Damas-Milner}
  \centering
  \[
    \text{TAUT:} \quad \frac{x : \sigma \in \Gamma}{\Gamma \vdash x : \sigma}
  \]

  \[
    \text{ABS:} \quad \frac{\Gamma, x : \tau \vdash e : \tau'}{\Gamma \vdash (\lambda x. e) : \tau \to \tau'}
  \]

  \[
    \text{APP:} \quad \frac{\Gamma \vdash e : \tau' \to \tau \quad \Gamma \vdash e' : \tau'}{\Gamma \vdash (e \ e') : \tau}
  \]

  \[
    \text{LET:} \quad \frac{\Gamma \vdash e : \sigma \quad \Gamma, x : \sigma \vdash e' : \tau}{\Gamma \vdash (\texttt{let } x = e \ \texttt{in} \ e') : \tau}
  \]

  \[
    \text{INST:} \quad \frac{\Gamma \vdash e : \sigma}{\Gamma \vdash e : \sigma'} \quad \scriptstyle (\sigma\ >\ \sigma')
  \]

  \[
    \text{GEN:} \quad \frac{\Gamma \vdash e : \sigma}{\Gamma \vdash e : \forall \alpha \sigma} \quad \scriptstyle (\alpha \text{ não livre em } \Gamma)
  \]

  \small{Fonte: o autor. Adaptado de \cite{DAMAS1982}}
  \label{eq:type-inference}
\end{figure}

As regras de inferência são interpretadas de baixo para cima.
Por exemplo, na regra da tautologia (TAUT):
\[
  \frac{x : \sigma \in \Gamma}{\Gamma \vdash x : \sigma}
\]
significa que, se em um contexto $\Gamma$, a variável $x$ possui o tipo $\sigma$, então podemos concluir que $x$ tem o tipo $\sigma$ no mesmo contexto.
Isso reflete o fato de que a associação de tipos no contexto é preservada.

Na regra de generalização (GEN), a condição de que $\alpha$ não seja livre em $\Gamma$ assegura que o tipo generalizado não dependa de nenhum tipo específico presente no contexto.
Isso permite que o tipo $\forall \alpha. \sigma$ seja usado de forma polimórfica em diferentes partes do programa.

Essas regras garantem a solidez do sistema, preservando a segurança dos tipos ao inferir automaticamente os tipos mais gerais possíveis para as expressões.

Antes de apresentar o Algoritmo W, é importante observar que ele é uma implementação prática das regras de inferência aqui descritas, usando o conceito de unificação para resolver as equações de tipo geradas durante a inferência.
A seguir, será discutido em detalhes o funcionamento do Algoritmo W.

\section{Algoritmo W}\label{sec:w-algo}

O Algoritmo W, introduzido em \citeonline{DAMAS1982}, é um algoritmo eficiente para inferência de tipos em linguagens de programação funcional.
Ele se baseia no processo de unificação para solucionar equações de tipos geradas durante a análise de expressões, atribuindo os tipos mais gerais possíveis, ou seja, os tipos mais polimórficos que ainda garantem a consistência do sistema.

% Mostrar e explicar algoritmo de unificação
A unificação é o processo de encontrar uma substituição de variáveis de tipo que torna dois tipos dados equivalentes.
Formalmente, dados dois tipos $\tau_1$ e $\tau_2$, a unificação procura uma substituição $S$ tal que $S\tau_1 = S\tau_2$.
Se tal substituição existe, os tipos são considerados unificáveis e $S$ é chamada de solução unificadora.
Caso contrário, os tipos são incompatíveis.

O algoritmo de unificação pode ser descrito pelas seguintes regras:

\begin{figure}[h!]
  \caption{Algoritmo de unificação utilizado no sistema de tipos Damas-Milner}
  \centering
  \begin{align*}
    \text{unificar}(C) =
    \begin{cases}
      \text{se } C = \emptyset, \text{ então } [\ ]                                      \\
      \text{senão, seja } \{S = T\} \cup C' = C \text{ tal que}                          \\
      \quad \text{se } S = T                                                             \\
      \quad \quad \text{então } \text{unificar}(C')                                      \\
      \quad \text{senão, se } S = X \text{ e } X \notin \text{FV}(T)                     \\
      \quad \quad \text{então } \text{unificar}([X \mapsto T]C') \circ [X \mapsto T]     \\
      \quad \text{senão, se } T = X \text{ e } X \notin \text{FV}(S)                     \\
      \quad \quad \text{então } \text{unificar}([X \mapsto S]C') \circ [X \mapsto S]     \\
      \quad \text{senão, se } S = S_1 \rightarrow S_2 \text{ e } T = T_1 \rightarrow T_2 \\
      \quad \quad \text{então } \text{unificar}(C' \cup \{S_1 = T_1, S_2 = T_2\})        \\
      \text{senão}                                                                       \\
      \quad \text{falha}                                                                 \\
    \end{cases}
  \end{align*}
  \small{Fonte: autor. Adaptado de \cite{PIERCE2002}}
  \label{eq:unify-algo}
\end{figure}

Neste algoritmo, $C$ representa um conjunto de equações de tipos a serem resolvidas, $S$ e $T$ são tipos que se tentam unificar, e $X$ representa uma variável de tipo.
A função $\text{FV}$ calcula as variáveis livres em um tipo.

O algoritmo trabalha iterativamente, tentando resolver a primeira equação do conjunto e, em seguida, aplicando recursivamente a unificação para o restante.
Ele segue as seguintes etapas:

1. Se o conjunto de equações $C$ estiver vazio, então a unificação é bem-sucedida e retorna a substituição vazia.

2. Se a equação atual $S = T$ for tal que os tipos $S$ e $T$ são idênticos, a equação é considerada resolvida, e o algoritmo prossegue com a unificação do restante do conjunto.

3. Se $S$ é uma variável de tipo $X$ que não ocorre livre em $T$, então substitui-se $X$ por $T$ em todas as equações restantes e prossegue-se com a unificação.

4. Similarmente, se $T$ é uma variável de tipo $X$ que não ocorre livre em $S$, então substitui-se $X$ por $S$ e prossegue-se com a unificação.

5. Se $S$ e $T$ são ambos tipos função, tenta-se unificar seus domínios e co-domínios.

6. Caso nenhuma dessas condições seja satisfeita, a unificação falha.

Essas regras são aplicadas iterativamente para resolver um conjunto de equações de tipos, formando a base para o funcionamento do Algoritmo W.

% Mostrar e explicar algoritmo W
