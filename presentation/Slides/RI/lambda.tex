\begin{frame}{Cálculo Lambda}
    Definido por \citeonline{church1932set}, o cálculo-$\lambda$ puro é representado pela gramática a seguir:

    \begin{flushleft}
        \qquad\qquad M := (M M) \hspace{20ex} (Aplicação)\\
        \qquad\qquad\qquad | V  \\
        
        \qquad\qquad V :=  x  \hspace{26ex} (Variável)\\
        \qquad\qquad\qquad | ($\lambda$x.M) \hspace{20ex} (Abstração)\\
    \end{flushleft}
\end{frame}

\begin{frame}{Cálculo Lambda}
    \begin{itemize}
        \item \textbf{Substituição:} Uma substituição é representada como $M[x:=N]$, de forma que toda ocorrência de $x$ em $M$ será substituída por $N$.

        \item \textbf{$\alpha$-equivalência:} É possível demonstrar a equivalência entre dois termos a partir da mudança de variáveis que estão ligadas em uma abstração, e.g., $\lambda x.x$ $\equiv$ $\lambda y.y$.

        \item \textbf{$\beta$-redução:} ($\lambda$x.M) N := M[x:=N]

        \item \textbf{Contexto:} 
            \begin{enumerate}
                \item \textit{Filling:} seja C = ($\lambda x. y []$) o \textit{filling} C[$\lambda z.x$], é o termo ($\lambda x. y (\lambda z.x)$).
            \end{enumerate}
    \end{itemize}
\end{frame}