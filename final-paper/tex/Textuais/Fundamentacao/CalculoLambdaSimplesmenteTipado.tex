\section{Cálculo Lambda Simplesmente Tipado}\label{sec:simply-typed-lambda-calulus}

O Cálculo Lambda Simplesmente Tipado é uma das primeiras e mais simples variantes do Cálculo Lambda que incorpora tipos em sua estrutura~\cite{church1940formulation}.
Enquanto o cálculo lambda original não faz distinção entre diferentes tipos de dados, no Cálculo Lambda Simplesmente Tipado os termos são anotados com tipos.
Cada função recebe e retorna valores de tipos específicos, o que permite prevenir uma série de erros comuns em programas, como a aplicação de funções a argumentos incorretos.
Além disso, o sistema de tipos serve como uma ferramenta de verificação durante a compilação de programas, assegurando que erros de tipo sejam detectados antes da execução.
Dessa forma, ele não apenas facilita a criação de software mais robusto, mas também oferece uma base formal para o estudo de linguagens de programação~\cite{pierce2002types}.

A sintaxe básica do Cálculo Lambda Simplesmente Tipado inclui:

\begin{itemize}
  \item Variáveis: $x, y, z, \ldots$
  \item Tipos: $T ::= \mathbf{Int} \mid \mathbf{Bool} \mid T \to T$
  \item Termos: $\lambda x:T. \tau \mid \tau_1 \tau_2 \mid x$
\end{itemize}

No Cálculo Lambda Simplesmente Tipado, cada variável possui um tipo atribuído e os termos são construídos com base nesses tipos.
Por exemplo, a abstração de função $\lambda x:T. \tau$ define uma função onde a variável $x$ é de tipo $T$ e o corpo da função, $\tau$, é um termo.
A aplicação de função $\tau_1 \tau_2$ indica que $\tau_1$ é uma função que é aplicada ao argumento $\tau_2$, o qual deve ter um tipo compatível com o tipo esperado por $\tau_1$.
Essa formalização facilita a composição de funções e o raciocínio sobre a estrutura dos programas, pois cada termo pode ser avaliado dentro de um contexto de tipagem.
A sintaxe dos tipos, como $T \to T$, define uma função que aceita um argumento do tipo $T$ e retorna um valor também do tipo $T$.

A inferência de tipos no Cálculo Lambda Simplesmente Tipado assegura que cada expressão tenha um tipo bem-definido, baseado nas regras de tipagem.
A tipagem de termos é feita através de um conjunto de regras formais que garantem a consistência dos tipos no programa.
Por exemplo, a regra de tipagem para abstrações lambda é a seguinte:

\[
  \frac{\Gamma, x:T_1 \vdash \tau:T_2}{\Gamma \vdash (\lambda x:T_1. \tau): T_1 \to T_2}
\]

Isso significa que, se o termo $\tau$ possui o tipo $T_2$ sob o contexto onde $x$ possui o tipo $T_1$, então a abstração $\lambda x:T_1. \tau$ tem o tipo $T_1 \to T_2$.
Essa verificação de tipo garante que, ao aplicar a função, o tipo do argumento corresponde ao tipo esperado pela função.

O Cálculo Lambda Simplesmente Tipado está intimamente relacionado com a lógica intuicionista proposicional.
Esse vínculo é formalizado pela Correspondência Curry-Howard, que estabelece uma correspondência direta entre proposições lógicas e tipos, e entre provas e programas.
Em outras palavras, tipos podem ser interpretados como proposições lógicas, e termos tipados como provas dessas proposições~\cite{pierce2002types}.
Por exemplo, o tipo $A \to B$ no Cálculo Lambda Simplesmente Tipado pode ser visto como a implicação lógica ``se $A$, então $B$''.
Assim, uma função que aceita um argumento do tipo $A$ e retorna um valor do tipo $B$ é equivalente a uma prova de que $A$ implica em $B$.
Esse princípio permite usar ferramentas da teoria de tipos para construir provas formais de teoremas em lógica intuicionista, fornecendo uma base teórica robusta para assistentes de prova automatizados, como o Coq~\cite{coquand1988calculus}.

Além disso, a Correspondência Curry-Howard não apenas conecta tipos e lógica, mas também oferece um método sistemático para projetar e raciocinar sobre sistemas de inferência de tipos, garantindo que programas tipados sejam corretos em relação às especificações lógicas.
A inferência de tipos desempenha um papel fundamental na programação funcional moderna, sendo inicialmente introduzida com a linguagem ML por~\citeonline{damas1982principal}, com o algoritmo W.
A linguagem Haskell extende o sistema Damas-Milner, adicionando principalmente o suporte a sobrecarga de funções.
