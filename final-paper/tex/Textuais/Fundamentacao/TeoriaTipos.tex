\section{Teoria de Tipos}\label{sec:type-theory}

A Teoria de Tipos, conforme apresentada por~\cite{coquand2022type}, foi introduzida por Russell em 1908 ao encontrar um paradoxo na Teoria de Conjuntos, conhecido atualmente como o Paradoxo de Russell:

\begin{equation}\label{eq:russell-paradox}
  \text{Seja } R = \{ x \mid x \notin x \}, \text{ então } R \in R \iff R \notin R
\end{equation}

Ou seja, considere $R$ como o conjunto dos conjuntos que não contêm a si mesmos.
A contradição surge ao observar que, se o conjunto $R$ contém a si mesmo, isso implica que $R$ não contém a si mesmo, e vice-versa.

Outra maneira de descrever esse paradoxo é através do Paradoxo do Barbeiro: imagine uma cidade com apenas um barbeiro, onde ele somente barbeia aqueles que não se barbeiam.
O paradoxo surge quando perguntamos: ``Quem barbeia o barbeiro?''
Ele não pode fazer sua própria barba, pois barbeia apenas aqueles que não fazem a própria barba.
No entanto, se ele não faz sua própria barba, então pertence ao grupo daqueles que devem ser barbeados pelo barbeiro, logo, ele deveria barbear-se.
Essa situação gera uma contradição semelhante ao Paradoxo de Russell.

Assim como os paradoxos na Teoria de Conjuntos expuseram a necessidade de fundamentos mais rigorosos para a matemática, a Teoria de Tipos surgiu como uma estrutura lógica para evitar inconsistências.
Na computação, essa ideia se reflete nos sistemas de tipos modernos, que impedem comportamentos paradoxais ou indefinidos em programas.
Por exemplo, ao restringir operações a tipos específicos, evita-se que funções sejam aplicadas a entidades incompatíveis — análogo a evitar que o barbeiro pertença ao conjunto que gera a contradição.

Atualmente, a principal aplicação da Teoria de Tipos está na formalização de sistemas de tipos para linguagens de programação.
Um sistema de tipos garante a ausência de certos comportamentos dos programas classificando os valores computados em cada uma de suas sentenças~\cite{pierce2002types}.
Além disso, atribuir e verificar tipos para cada construção presente nos programas têm várias utilidades, como fornecer informações para auxiliar na modularização de programas, otimização de código executada pelo compilador, além de poder ser usada como documentação do código.

No contexto das linguagens de programação, podemos distinguir três categorias principais de tipos: tipos simples, tipos polimórficos e tipos dependentes~\cite{pierce2002types}.
Tipos simples atribuem um tipo fixo a cada termo, enquanto tipos polimórficos introduzem a noção de generalidade, permitindo que funções possam ser aplicadas a argumentos de diferentes tipos sem a necessidade de serem redefinidas para cada um.
Já os tipos dependentes permitem que tipos dependam de valores.

Um exemplo de tipo simples é uma função que opera sobre números inteiros.
Esta função recebe um número inteiro e retorna outro número inteiro.
Seu tipo, portanto, é representado como $Int \rightarrow Int$, indicando que tanto a entrada quanto a saída são do tipo inteiro.

Um exemplo de polimorfismo é a função identidade, que recebe um elemento de qualquer tipo e retorna o mesmo elemento.
Seu tipo é expresso como $a \rightarrow a$, onde $a$ pode ser qualquer tipo, caracterizando o polimorfismo paramétrico.
Nesse caso, a função mantém o mesmo comportamento para todos os tipos, sem necessidade de reimplementação.

Em linguagens com suporte a tipos dependentes, um exemplo seria o de um vetor cujo comprimento (número de elementos) faz parte de seu tipo.
Nesse caso, uma função de concatenação de vetores deve garantir que somente vetores com tipos compatíveis em relação ao comprimento possam ser concatenados.
O tipo da função de concatenação seria algo como\footnote{A notação exata pode variar entre diferentes linguagens de programação que suportam tipos dependentes. A estrutura apresentada serve apenas como uma ilustração conceitual do comportamento esperado.} $Vector(n) \rightarrow Vector(m) \rightarrow Vector(n+m)$, onde $n$ e $m$ são valores que representam os comprimentos dos vetores e fazem parte da definição de tipo.

No contexto do polimorfismo,~\citeonline{pierce2002types} define duas principais variedades: o polimorfismo paramétrico, (como é o caso da função identidade), e o polimorfismo com sobrecarga.
No primeiro, uma única definição opera genericamente, mantendo o mesmo comportamento para todos os tipos. 
Já no segundo, o comportamento varia conforme o tipo dos argumentos, permitindo múltiplas implementações — como na sobrecarga de operadores, onde a função selecionada depende dos tipos dos operandos.

O polimorfismo desempenha um papel crucial na inferência de tipos. 
Em linguagens como Haskell, o sistema de tipos deduz automaticamente tipos genéricos sempre que possível, permitindo que funções como a identidade ($\lambda x.x$, tipada como $\forall \alpha. \alpha \to \alpha$) sejam usadas de forma universal. 
Por outro lado, funções com restrições específicas ilustram o polimorfismo com sobrecarga. 
Por exemplo, a função de soma no Listing~\ref{code:add}, originalmente definida como $Int \rightarrow Int \rightarrow Int$, pode ser generalizada para operar sobre quaisquer tipos numéricos utilizando classes de tipos. 
Isso é alcançado ao substituir a tipagem explícita por uma restrição como \texttt{Num a => a -> a -> a}, onde \texttt{Num a} indica que \texttt{a} deve pertencer à classe de tipos numéricos.

Em Haskell, classes de tipos são mecanismos que habilitam o polimorfismo de sobrecarga. 
A função \texttt{sumList}, por exemplo, (Listing~\ref{code:sum_list}) é definida com a restrição \texttt{Num a}, permitindo que opere sobre listas de inteiros, \texttt{Float}, \texttt{Double} ou outros tipos numéricos. 
Essa abordagem combina flexibilidade e segurança: o polimorfismo com sobrecarga garante que a função generalize seu comportamento apenas dentro de um domínio específico (\textit{e.g.}, números), evitando inconsistências. 

\lstinputlisting[style=haskell, label=code:sum_list, caption={Função somatório de elementos de lista em Haskell}]{Code/sum_list.hs}
