\newcommand{\Mgu}{\ensuremath{\textit{mgu}}}
\newcommand{\MguList}{\ensuremath{\textit{mguList}}}
\newcommand{\UnifyVar}{\ensuremath{\textit{varBind}}}
\newcommand{\HeadSep}{\ensuremath{\textit{:}}}
\newcommand{\Length}{\ensuremath{\textit{length}}}
\newcommand{\List}{\ensuremath{\textit{list}}}

\section{Formalização}\label{sec:formalizacao}

O sistema de tipos formalizado aqui foi fortemente inspirado no sistema de Damas e Milner, explicado na Seção~\ref{sec:damas-milner}, onde suas regras foram adaptadas de modo que elas se enquadrem no sistema polimórfico baseado em continuações.
Em particular, o contexto $\Gamma$ associa variáveis a tipos polimórfico e define julgamentos distintos para representar átomos e comandos.
A distinção destes se mostra necessária uma vez que é levado em consideração o comportamento não retornável das continuações. 

A sintaxe do sistema conta com expressões e tipos usados no processo de tipagem e de inferência de tipos.
Abaixo, segue a gramática das expressões e tipos presentes:

\phantom{Newline}

\begin{tabular}{lccl}
  Átomos & $a$ & $\Coloneqq$ & $x \enspace|\enspace n$ \\
  Comandos & $b$ & $\Coloneqq$ & $x(\vv{a}) \enspace|\enspace \mathtt{let}\ x(\vv{x}) = b\ \mathtt{in}\ b$ \\
  \\
  Tipos monomórficos & $\tau$ & $\Coloneqq$ & $\alpha \enspace|\enspace \mathtt{int} \enspace|\enspace \neg\vv{\tau}$ \\
  Tipos polimórficos & $\sigma$ & $\Coloneqq$ & $\forall\vv{\alpha}.\tau$ \\
  Contexto & $\Gamma$ & $\Coloneqq$ & $\cdot \enspace|\enspace \Gamma,\ x{:}\ \sigma$ \\
\end{tabular}\label{cps-type-system}

\phantom{Newline}

\noindent Na sintaxe apresentada, $a$ representa os átomos. Isto é, variáveis do programa ($x$) e literais inteiros ($n$) formam os elementos primitivos do sistema.
Os comandos $b$, por sua vez, são as expressões, explicadas com mais detalhes na Seção~\ref{subsec:cps}, sendo a primeira o \textit{jump}, e a segunda o \textit{bind}.
Três elementos distintos compõem os tipos presentes neste sistema.
Os tipos monomórficos ($\tau$), são os tipos que não possuem quantificação (monomórficos), podendo ser variáveis de tipo ($\alpha$), tipos numéricos inteiros ($\mathtt{int}$), ou ainda, tipos negados ($\neg\vv{\tau}$), usados para representar funções que retornam absurdos.
Já os tipos polimórficos ($\sigma$), são responsáveis por garantir a quantificação universal de variáveis de tipos (polimórficos).
Por fim, o contexto ($\Gamma$) contém o mapeamento de cada variável para um tipo polimórfico ($\sigma$).

As regras sintáticas de tipagem do sistema de tipos, inspiradas no Sistema Damas-Milner são ilustradas a seguir:

\phantom{Newline}

\fbox{$\Gamma\vdash a{:}\ \tau$}

\begin{prooftree}
    \RightLabel{$\mathtt{[Var]}$}
    \AxiomC{$x{:}\ \sigma \in \Gamma$}
    \AxiomC{$\sigma > \tau$}
    \BinaryInfC{$\Gamma\vdash x{:}\ \tau$}
\end{prooftree}
\begin{prooftree}
    \RightLabel{$\mathtt{[Int]}$}
    \AxiomC{}
    \UnaryInfC{$\Gamma\vdash n{:}\ \mathtt{int}$}
\end{prooftree}

\phantom{Newline}

\fbox{$\Gamma\vdash b$}

\begin{prooftree}
    \RightLabel{$\mathtt{[Jump]}$}
    \AxiomC{$\Gamma\vdash k{:}\ \neg\vv\tau$}
    \AxiomC{$\Gamma\vdash \vv{a}{:}\ \vv{\tau}$}
    \BinaryInfC{$\Gamma\vdash k(\vv{a})$}
\end{prooftree}

\begin{prooftree}
    \RightLabel{$\mathtt{[Bind]}$}
    \AxiomC{$\Gamma, \vv{x}{:}\ \vv{\tau}\vdash c$}
    \AxiomC{$\Gamma,\ k{:}\ \overline{\Gamma}(\neg\vv\tau)\vdash b$}
    \BinaryInfC{$\Gamma\vdash\mathtt{let}\ k(\vv{x}) = c\ \mathtt{in}\ b$}
\end{prooftree}

Aqui, a regra $\mathtt{[Var]}$ define como tipo de uma variável uma instância do tipo (possivelmente polimórfico) que está associado a variável no contexto de tipos.
O símbolo $>$ denota essa relação de ordem, indicando que o tipo $\sigma$ é mais geral que $\tau$.
Assim, em um contexto $\Gamma$, uma variável $x$ terá tipo $\tau$ caso esta esteja presente no contexto.
A regra $\mathtt{[Int]}$, é direta.
Em um contexto $\Gamma$, um literal inteiro terá um tipo $\mathtt{int}$.
Por exemplo, se $x{:}\ \forall\vv\alpha.\tau\in\Gamma$, então $\Gamma\vdash x{:}\ \tau$ (após instanciação adequada das variáveis de tipo).

As continuações, como discutido na Seção~\ref{subsec:cps}, representam fluxos de controle que não retornam valores.
Uma vez que a continuação pode ser interpretada como o próximo passo de uma computação, e a computação se dá por contradições, a continuação em si não possui um tipo, ela representa um absurdo.
Então, pode se dizer que a continuação é uma testemunha de que aquilo é um absurdo.

A regra $[\mathtt{Jump}]$ portanto, diz que sob um contexto $\Gamma$, se $k{:}\ \neg(\tau_1,\dots,\tau_n)$ com $n$ argumentos e cada argumento $a_i$ tiver um tipo correspondente $\tau_i$, então $k(\vv{a})$ é válido, ou seja, o salto $k$ com os argumentos $\vv{a}$ é testemunha de uma contradição.
De modo semelhante para o $\mathtt{[Bind]}$, as premissas para $c$ e $b$ onde $c$ é derivável a partir do contexto $\Gamma\ \cup\ \{\ \vv{x}{:}\ \vv{\tau}\ \}$, e $b$ sob o contexto $\Gamma\ \cup\ \{\ k{:}\ \overline{\Gamma}(\neg\vv\tau)\ \}$, são testemunhas de que o comando $\mathtt{let}\ k(\vv{x}) = c\ \mathtt{in}\ b$ é uma contradição.
Assim como o `let' introduz o polimorfismo no sistema Damas-Milner, a generalização $\overline{\Gamma}(\neg\vv{\tau})$ presente na premissa do $\mathtt{[Bind]}$ quantifica as variáveis livres de $\vv{\tau}$ em $\Gamma$, estendendo o polimorfismo também ao CPS.

O algoritmo de inferência de tipos segue o mesmo esquema de Damas-Milner (algoritmo W) adaptado ao CPS.
Assim como o W, este faz uso do unificador mais geral, retornando sempre que existir o tipo mais genérico das expressões pertencentes a este sistema.
Abaixo, tem-se sua definição:

\phantom{Newline}

\fbox{$\Gamma\vdash_W a{:}\ \tau$}

\begin{prooftree}
    \AxiomC{$x{:}\ \sigma \in \Gamma$}
    \AxiomC{$\tau = \mathit{inst}(\sigma)$}
    \BinaryInfC{$\Gamma \vdash_W x{:}\ \tau$}
\end{prooftree}

\begin{prooftree}
    \AxiomC{}
    \UnaryInfC{$\Gamma \vdash_W n{:}\ \mathtt{int}$}
\end{prooftree}

\phantom{Newline}

\fbox{$\Gamma\vdash_W b \Rightarrow S$}

\begin{prooftree}
    \AxiomC{$\Gamma \vdash_W k{:}\ \tau_1$}
    \AxiomC{$\Gamma \vdash_W \vv{a}{:}\ \vv{\tau_2}$}
    \AxiomC{$S = \mathit{mgu}(\tau_1, \neg\vv{\tau_2})$}
    \TrinaryInfC{$\Gamma \vdash_W k(\vv{a}) \Rightarrow S$}
\end{prooftree}

\begin{prooftree}
    \AxiomC{$\vv{\tau} = \vv{\mathit{newvar}}$}
    \AxiomC{$\Gamma, \vv{x}{:}\ \vv{\tau} \vdash_W c \Rightarrow S_1$}
    \AxiomC{$\sigma = \overline{S_1\Gamma}(S_1 \neg\vv{\tau})$}
    \AxiomC{$S_1\Gamma, k{:}\ \sigma\vdash_W b \Rightarrow S_2$}
    \QuaternaryInfC{$\Gamma \vdash_W \mathtt{let}\ k(\vv{x}) = c\ \mathtt{in}\ b \Rightarrow S_2 \circ S_1$}
\end{prooftree}
Para as regras de inferência dos átomos, tal qual o sistema de tipos definido anteriormente, o algoritmo com uma variável $x$ de tipo polimórfico $\sigma$ pertencente ao contexto $\Gamma$ retornará um tipo monomórfico $\tau$ sob o mesmo contexto onde $\tau$ será a instanciação deste tipo $\sigma$.
Para o tipo numérico, a regra não exige condições adicionais, e o algoritmo inferirá diretamente o tipo $\mathtt{int}$ para o literal $n$.
\begin{prooftree}
    \AxiomC{$a{:}\ \forall\alpha.\alpha \in \Gamma$}
    \AxiomC{$\alpha = \mathit{inst}(\forall\alpha.\alpha)$}
    \BinaryInfC{$\Gamma \vdash_W a{:}\ \alpha$}
\end{prooftree}
Por exemplo, esteja a variável $a$ com tipo $\forall\alpha.\alpha$ no contexto, ou seja, $a{:}\ \forall\alpha.\alpha\ \in\ \Gamma$.
O algoritmo então inferirá, que a variável $a$ terá tipo $\alpha$, após as devidas normalizações (redução-$\alpha$).

Os comandos serão inferidos com substituições, onde o algoritmo as retornará representando o absurdo para qual esses testemunham.
Para o $\mathtt{[Jump]}$, partindo das premissas onde sob um contexto $\Gamma$, a chamada $k$ terá um tipo monomórfico $\tau_1$, os $n$ argumentos em $\vv{a}$ terão $n$ tipos monomórficos $\tau_2$, e ainda, $S$ é a unificação mais geral entre $\tau_1$ e $\neg\vv{\tau_2}$, o algoritmo irá então retornar esta substituição $S$ para o salto $k(\vv{a})$.

\begin{prooftree}
    \AxiomC{$\Gamma \vdash_W k{:}\ \alpha$}
    \AxiomC{$\Gamma \vdash_W x{:}\ \beta$}
    \AxiomC{$S = \mathit{mgu}(\alpha, \neg\beta)$}
    \TrinaryInfC{$\Gamma \vdash_W k(x) \Rightarrow \{\ \alpha \mapsto \neg\beta\ \}$}
\end{prooftree}
Por exemplo, em determinado contexto $\Gamma$, seja a chamada $k$ com tipo $\alpha$, ou seja, $\Gamma \vdash_W k{:}\ \alpha$, e ainda sob o mesmo contexto, o argumento $x$ com tipo $\beta$, ou seja, $\Gamma \vdash_W x{:}\ \beta$.
A partir da unificação mais geral entre $\alpha$ e $\neg\beta$ é obtida a substituição $S$, ou seja, $S = \mathit{mgu}(\alpha, \neg\beta)$.
O algoritmo então, irá inferir que a substituição para que o salto represente uma contradição é $\{\alpha \mapsto \neg\beta\}$.

Vale destacar que o algoritmo de unificação apresentado na Figura~\ref{algo:unify}, ainda que retorne a substituição que representa a unificação mais geral, não é o mesmo que o $\mathit{mgu}$ utilizado na regra $[\mathtt{Jump}]$.
Suas definições variam conforme os sistemas de tipos que elas atendem.
Enquanto que a função \Unify\ é definida para os tipos do Sistema Damas-Milner, a \Mgu apresentada na Figura~\ref{algo:mgu-cps} é definida para os tipos do Sistema baseado em continuações.

\begin{figure}[ht!]
  \centering
  \begin{align*}
    & \texttt{\Mgu($\alpha,\Whitespace\tau$) = } \\
    & \qquad{} \texttt{\Return\Whitespace\UnifyVar($\alpha,\Whitespace\tau$)} \\
    & \texttt{\Mgu($\tau,\Whitespace\alpha$) = } \\
    & \qquad{} \texttt{\Return\Whitespace\UnifyVar($\alpha,\Whitespace\tau$)} \\
    & \texttt{\Mgu(\texttt{Int},\Whitespace\texttt{Int}) = } \\
    & \qquad{} \texttt{\Return\Whitespace[$\Whitespace$]} \\
    & \texttt{\Mgu(\texttt{Neg}\Whitespace ${list}_{1}$,\Whitespace\texttt{Neg} ${list}_{2}$) = } \\
    & \qquad{} \texttt{\If \Whitespace \Length\Whitespace${list}_{1}$ $\neq$ \Length\Whitespace${list}_{2}$, \Then \Whitespace \Fail} \\
    & \qquad{} \texttt{\Else\Whitespace\Return\Whitespace\MguList(${list}_{1},\Whitespace{list}_{2}$)} \\
    & \texttt{\Mgu($\tau_1,\Whitespace\tau_2$) = } \\
    & \qquad{} \texttt{\Fail} \\
    \\
    & \texttt{\MguList($[\Whitespace],\Whitespace[\Whitespace]$) = } \\
    & \qquad{} \texttt{\Return\Whitespace[$\Whitespace$]} \\
    & \texttt{\MguList($[\tau\HeadSep\Whitespace\tau_s],\Whitespace[\tau'\HeadSep\Whitespace\tau_s']$) = } \\
    & \qquad{} \texttt{$S_1$\Whitespace:=\Whitespace\Mgu($\tau,\Whitespace\tau'$)} \\
    & \qquad{} \texttt{$S_2$\Whitespace:=\Whitespace\MguList($S_1(\tau_s),\Whitespace S_1(\tau_s')$)} \\
    & \qquad{} \texttt{\Return\Whitespace$S_2\Whitespace \circ\Whitespace S_1$} \\
    & \texttt{\MguList($\_,\Whitespace\_$) = } \\
    & \qquad{} \texttt{\Fail}
  \end{align*}
  \caption{Algoritmo de unificação para o Sistema baseado em continuações no formato de função.}
  \small{Fonte: autor.}
  \label{algo:mgu-cps}
\end{figure}

A principal diferença deste algoritmo em relação ao anterior é que, neste sistema de tipos, há listas de tipos que também precisam ser unificadas.
Para realizar essa unificação, verifica-se inicialmente se as duas listas possuem o mesmo tamanho, ou seja, se $\texttt{\Length}\ list_1 = \texttt{\Length}\ list_2$.
Caso essa condição seja satisfeita, o algoritmo \Mgu\ é aplicado recursivamente a cada par de tipos correspondentes nos mesmos índices das listas, acumulando as substituições parciais ao longo do processo.
A função \texttt{\UnifyVar} é responsável por realizar a unificação entre variáveis de tipo e outros tipos, retornando a substituição correspondente ou um erro, caso a verificação de ocorrência via \texttt{\Occurs} falhe.

Para a regra que garante o polimorfismo do sistema, o $[\mathtt{Bind}]$, a chamada $k$ recebe $\vv{x}$ argumentos, onde estes terão $\vv{\tau}$ tipos como sendo variáveis de tipo, ou seja, $\vv{\tau} = \vv{\mathit{newvar}}$.
O $c$, por se tratar de um comando, é uma substituição $S_1$, onde recursivamente será inferida com o contexto inicial $\Gamma$ unido com os $\vv{x}$ argumentos tipados com suas $\vv{\tau}$ variáveis de tipo frescas, ou seja, $\{\ \Gamma\ \cup\ \{\ \vv{x}{:}\ \vv{\tau}\ \}\ \} \vdash_W c \Rightarrow S_1$.
Um ponto de atenção é necessário na função de generalização $\sigma = \overline{S_1\Gamma}(S_1 \neg\vv{\tau})$.
A subsituição $S_1$ aplicada no contexto garante que este esteja atualizado com a descoberta do comando $c$ na premissa anterior.
Como as continuações não retornam e sim somente passam o resultado da computação adiante, é necessário também que $S_1$ seja aplicado no tipo do argumento $S_1 \neg\vv{\tau}$, para garantir que a substituição obtida no comando anterior seja utilizada nos tipos.
De maneira semelhante ao primeiro comando, o comando $b$ é inferido recursivamente com $S_1$ aplicado no contexto unido com o salto $k$ tendo o tipo polimórfico $\sigma$ produzido na premissa anterior, sendo atribuido a esta inferência a substituição $S_2$, ou seja, $\{\ S_1\Gamma\ \cup\ \{\ k{:}\ \sigma \ \} \ \} \Rightarrow S_2$.
O algoritmo portanto, para o comando $\mathtt{let}\ k(\vv{x}) = c\ \mathtt{in}\ b$, irá produzir a substituição resultante da composição entre as substituições de $b$ e $c$, ou seja, $S_2 \circ S_1$.
