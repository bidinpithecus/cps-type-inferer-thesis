\begin{frame}{Estilo de Passagem de Continuação (CPS)}
    \begin{itemize}
        \item Técnica de transformação de código que torna o fluxo de controle explícito
              \begin{itemize}
                  \item[--] Chamadas de função passam o controle para a próxima etapa explicitamente, conhecida como continuação~\cite{appel1992compiling}
                  \item[--] Ao invés das funções retornarem o resultado da computação, é invocado uma continuação, representando o próximo passo
              \end{itemize}
        \item Toda chamada de função passa então a ser uma chamada de cauda (do inglês \textit{tail-call})
    \end{itemize}
\end{frame}

\begin{frame}{Estilo de Passagem de Continuação (CPS)}
    \textbf{Chamada de cauda:}
    \begin{itemize}
        \item Última instrução executada em uma função é uma chamada a outra função, sem que restem computações adicionais a serem feitas após essa chamada~\cite{MUCHNICK1997}
              \begin{itemize}
                  \item[--] Função atual pode liberar seu quadro de ativação
              \end{itemize}
    \end{itemize}
    \textbf{Chamada não de cauda:}
    \begin{itemize}
        \item Ainda restam operações, como somas ou multiplicações, após a chamada da função
              \begin{itemize}
                  \item[--] Função atual precisa manter seu quadro de ativação até que as operações sejam concluídas
              \end{itemize}
    \end{itemize}
\end{frame}

\begin{frame}{Estilo de Passagem de Continuação (CPS)}
    \begin{itemize}
        \item[] \begin{figure}
                  \caption{Função fatorial em Haskell com chamada não de cauda}
                  \lstinputlisting[style=haskell, label=code:factorial_non_tail_call]{Code/factorial_non_tail_call.hs}
                  \small{Fonte: o autor}
              \end{figure}

        \item[] \begin{figure}
                  \caption{Função fatorial em Haskell com chamada de cauda}
                  \lstinputlisting[style=haskell, label=code:factorial_tail_call]{Code/factorial_tail_call.hs}
                  \small{Fonte: o autor}
              \end{figure}
    \end{itemize}
\end{frame}
