\section{Formalização}\label{sec:formalizacao}

Em razão da natureza mais prática deste trabalho, a notação utilizada para representar o cálculo de continuações será a mesma utilizada por~\cite{appel1997shrinking}, a sintaxe do `let'.
O sistema de tipos formalizado aqui foi fortemente inspirado no sistema de Damas e Milner, explicado na Seção~\ref{sec:damas-milner}, tendo as regras adaptadas de modo que elas se enquadrem no sistema polimórfico baseado em continuações.

A sintaxe do sistema conta com expressões e tipos usados no processo de tipagem e de inferência de tipos.
Abaixo\footnote{Vale destacar que, todas as formalizações presentes aqui nesta seção, foram feitas pelo coorientador em reunião juntamente do autor, onde esse explicava suas motivações para atingir o resultado. Ainda, no momento da produção deste trabalho, não foi feita uma publicação contendo estas formalizações para que haja refências o apontando.}, segue a gramática das expressões e tipos presentes:

\phantom{Newline}

\begin{tabular}{lccl}\label{cps-type-system}
  Átomos & $a$ & $\Coloneqq$ & $x \enspace|\enspace n$ \\
  Comandos & $b$ & $\Coloneqq$ & $x(\vv{a}) \enspace|\enspace \mathtt{let}\ x(\vv{x}) = b\ \mathtt{in}\ b$ \\
  \\
  Monotipo & $\tau$ & $\Coloneqq$ & $\alpha \enspace|\enspace \mathtt{int} \enspace|\enspace \neg\vv{\tau}$ \\
  Politipo & $\sigma$ & $\Coloneqq$ & $\forall\vv{\alpha}.\tau$ \\
  Contexto & $\Gamma$ & $\Coloneqq$ & $\cdot \enspace|\enspace \Gamma, x{:}\ \sigma$ \\
\end{tabular}

\phantom{Newline}

Na sintaxe apresentada, $a$ representa os átomos. Isto é, variáveis ($x$) literais inteiros ($n$) formam os elementos primitivos do sistema.
Os comandos $b$, por sua vez, são as expressões, explicadas com mais detalhes na Seção~\ref{subsec:cps}, sendo a primeira o \textit{jump}, e a segunda o \textit{binding}.
Três elementos distintos compõem os tipos presentes neste sistema.
Os monotipos ($\tau$), são os tipos que não possuem quantificação, ou seja, são monomórficos, podendo ser variáveis de tipo ($\alpha$), tipos numéricos inteiros ($\mathtt{int}$), ou ainda, tipos negados ($\neg\vv{\tau}$).
Já os politipos ($\sigma$), são responsáveis por garantir a quantificação universal de variáveis de tipos, ou seja, os tipos polimórficos.
Por fim, o contexto ($\Gamma$) contém o mapeamento de cada variável para um politipo ($\sigma$).

As regras sintáticas de tipagem do sistema de tipos, contando com elementos também presentes no Sistema Damas-Milner são ilustrados a seguir:

\phantom{Newline}

\fbox{$\Gamma\vdash a : \tau$}

\begin{prooftree}
    \RightLabel{$\mathtt{[Var]}$}
    \AxiomC{$x : \sigma \in \Gamma$}
    \AxiomC{$\sigma \sqsubseteq \tau$}
    \BinaryInfC{$\Gamma\vdash x : \tau$}
\end{prooftree}
\begin{prooftree}
    \RightLabel{$\mathtt{[Int]}$}
    \AxiomC{}
    \UnaryInfC{$\Gamma\vdash n : \mathtt{int}$}
\end{prooftree}

Aqui, para os átomos do sistema, a partir da regra $\mathtt{[Var]}$ é esperado que as variáveis, como vêm do contexto, tenham um tipo polimórfico $\sigma$.
Existe ainda um tipo monomórfico $\tau$ que, a partir da relação de ordem $\sqsubseteq$, será originado a partir do $\sigma$.
Assim, em um contexto $\Gamma$, uma variável $x$ terá tipo $\tau$ caso esta tenha um tipo $sigma$ neste contexto.

A regra $\mathtt{[Int]}$, é bem direta ao ponto.
Em um contexto $Gamma$, um literal inteiro terá um tipo $\mathtt{int}$.

\phantom{Newline}

\fbox{$\Gamma\vdash b$}

\begin{prooftree}
    \RightLabel{$\mathtt{[Jump]}$}
    \AxiomC{$\Gamma\vdash k : \neg\vv\tau$}
    \AxiomC{$\Gamma\vdash \vv{a} : \vv{\tau}$}
    \BinaryInfC{$\Gamma\vdash k(\vv{a})$}
\end{prooftree}

\begin{prooftree}
    \RightLabel{$\mathtt{[Bind]}$}
    \AxiomC{$\Gamma, \vv{x}{:}\ \vv{\tau}\vdash c$}
    \AxiomC{$\Gamma,\ k{:}\ \overline{\Gamma}(\neg\vv\tau)\vdash b$}
    \BinaryInfC{$\mathtt{let}\ k(\vv{x}) = c\ \mathtt{in}\ b$}
\end{prooftree}

% Explicar aqui regra do jump e regra do bind.

\phantom{Newline}
