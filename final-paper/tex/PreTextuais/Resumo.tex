% ---
% RESUMOS
% ---

\setlength{\absparsep}{18pt}    % ajusta o espaçamento dos parágrafos do resumo
\begin{resumo}
  No contexto de compiladores, as representações intermediárias desempenham um papel fundamental, especialmente em ambientes de produção.
  O \textit{Continuation Passing Style} (CPS) é uma dessas representações, notável para linguagens funcionais devido às otimizações avançadas que permite.
  Contudo, há uma escassez de pesquisas e implementações dessa IR com termos devidamente tipados, o que impede a detecção de uma gama de erros durante as etapas de compilação.
  Este trabalho, portanto, explora a formalização de um sistema de tipos para o CPS e o desenvolvimento de um algoritmo de inferência de tipos para essa representação, juntamente com sua implementação na linguagem de programação Haskell. 
  
  \textbf{Palavras-chave}: Inferência de Tipos, Estilo de Passagem de Continuação (CPS), Damas-Milner, Haskell, Sistema de Tipos.
\end{resumo}
