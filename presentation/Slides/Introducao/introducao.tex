\begin{frame}{Introdução}
    \textbf{Compilação:}
    \begin{itemize}
        \item Tradução de código de uma linguagem para outra
              \begin{itemize}
                  \item[--] Geralmente do código-fonte para o de máquina
              \end{itemize}
        \item Composta por diferentes etapas como:
              \begin{itemize}
                  \item[--] Análise léxica
                  \item[--] Análise sintática
                  \item[--] Análise semântica
                  \item[--] Otimizações
                  \item[--] Geração de código
              \end{itemize}
        \item Ligadas por Representações Intermediárias
              \begin{itemize}
                  \item[--] Principalmente nas otimizações~\cite{PLOTKIN1975125}
              \end{itemize}
    \end{itemize}
\end{frame}

\begin{frame}{Introdução}
    \textbf{Representações Intermediárias:}
    \begin{itemize}
        \item Linguagens imperativas
              \begin{itemize}
                  \item[--] Atribuição Única Estática (SSA)
              \end{itemize}
        \item Linguagens funcionais
              \begin{itemize}
                  \item[--] Forma Normal Administrativa (ANF)
                  \item[--] Estilo de Passagem de Continuação (CPS)
              \end{itemize}
        \item CPS
              \begin{itemize}
                  \item[--] Continuações explícitas
                        \begin{itemize}
                            \item[--] Parâmetro extra na função
                            \item[--] Funções sem retorno
                        \end{itemize}
                  \item[--] Otimizações
                        \begin{itemize}
                            \item[--] Eliminação da pilha de chamadas
                            \item[--] Eliminação de chamadas de cauda
                        \end{itemize}
              \end{itemize}
    \end{itemize}
\end{frame}
