\newcommand{\defas}{\ensuremath{\overset{def}{=}}}
\newcommand{\fv}{\ensuremath{\text{FV}}}
\newcommand{\fvc}{\ensuremath{\text{FVC}}}
\newcommand{\eeq}{\ensuremath{\overset{e}{=}}}
\newcommand{\append}{\ensuremath{\texttt{ ++ }}}
\newcommand{\V}{\ensuremath{\mathcal{V}}}
\newcommand{\C}{\ensuremath{\mathbb{C}}}
\newcommand{\wf}{\ensuremath{\textit{ wf\/}}}
\newcommand{\If}{\ensuremath{\text{if }}}
\newcommand{\Let}{\ensuremath{\text{let }}}
\newcommand{\In}{\ensuremath{\text{in }}}
\newcommand{\Then}{\ensuremath{\text{ then }}}
\newcommand{\Else}{\ensuremath{\text{ else }}}
\newcommand{\fail}{\ensuremath{\text{fail}}}
\newcommand{\unify}{\ensuremath{\textit{ unify\/}}}
\newcommand{\occurs}{\ensuremath{\textit{ ocurrs\/}}}
\newcommand{\dom}{\ensuremath{\textit{ dom\/}}}
\newcommand{\size}{\ensuremath{\textit{ size\/}}}
\newcommand{\unifier}{\ensuremath{\textit{ unifier\/}}}
\newcommand{\True}{\ensuremath{\texttt{ True}}}
\newcommand{\False}{\ensuremath{\texttt{ False}}}

\section{Sistema Damas-Milner}\label{sec:damas-milner}

O sistema Damas-Milner, introduzido por Robin Milner e posteriormente formalizado em maior detalhe por Luis Damas~\cite{MILNER1978, DAMAS1982}, é um dos sistemas de tipos mais influentes para linguagens funcionais.
Este sistema tem como principal característica a inferência automática de tipos polimórficos, sem a necessidade de anotações explícitas por parte do programador, como ocorre em linguagens como ML, Haskell e OCaml.
A sua base é o cálculo lambda com polimorfismo paramétrico, introduzido via \texttt{let}, permitindo que funções possam operar sobre múltiplos tipos de maneira genérica.

A introdução do sistema Damas-Milner trouxe duas contribuições principais: a definição de um sistema de tipos robusto e a criação de um algoritmo, o Algoritmo W, capaz de inferir o tipo mais geral (também chamado de \textit{principal type-scheme}), conforme demonstrado em~\citeonline{DAMAS1984}.
O algoritmo é consistente e completo em relação ao sistema de tipos: a consistência assegura que todo tipo inferido é correto, ou seja, pode ser derivado pelo sistema de tipos; já a completude garante que qualquer tipo derivado pelo sistema será uma instância do tipo inferido pelo algoritmo.
Como resultado, a linguagem ML e suas derivadas se tornaram notórias por fornecer ao programador a capacidade de escrever programas sem erros de tipo detectáveis durante a compilação, permitindo um desenvolvimento mais seguro e robusto~\cite{MILNER1978, DAMAS1984}.

A sintaxe do sistema Damas-Milner define as expressões e os tipos usados no processo de inferência.
Abaixo, segue a gramática das expressões e tipos:

\begin{equation}\label{eq:dm-syntax}
  \begin{array}{ll}
    \text{Variáveis}         & x                                                                                 \\
    \text{Expressões}        & e ::= x \mid e \ e' \mid \lambda x.e \mid \texttt{let} \ x = e \ \texttt{in} \ e' \\
    \\
    \text{Variáveis de tipo} & \alpha                                                                            \\
    \text{Tipos primitivos}  & \iota                                                                             \\
    \text{Tipos}             & \tau ::= \alpha \mid \iota \mid \tau \rightarrow \tau
    \alpha                                                                                                       \\
    \text{Schemes}           & \sigma ::= \forall \alpha. \sigma \mid \tau                                       \\
  \end{array}
\end{equation}

Na sintaxe, $x$ representa variáveis que podem ser nomes de qualquer identificador, e $e$ descreve expressões que podem ser variáveis, aplicações de função, funções anônimas ou declarações \texttt{let}, que introduzem polimorfismo através de generalização de tipos.
$\alpha$ é usado para representar variáveis de tipos.
Os tipos primitivos $\iota$ são usados para representar tipos constantes.
Tipos $\tau$ podem ser tanto variáveis de tipo quanto funções entre tipos.
Por fim, $\sigma$ denota \textit{schemes}, ou tipos polimórficos, que podem quantificar variáveis de tipo, permitindo reutilização de variáveis de tipos em diferentes contextos.

O polimorfismo no sistema Damas-Milner é introduzido pelas expressões \texttt{let}, que permitem a generalização de tipos.
Ao declarar uma variável ou função usando \texttt{let}, o tipo inferido é generalizado para ser utilizado de maneira polimórfica na expressão que ocorre após o \texttt{in}.
Isso significa que, ao declarar uma função como $\texttt{let id} = \lambda x.x$, o sistema deduz o tipo mais geral $\forall \alpha. \alpha \rightarrow \alpha$, que pode ter sua variável de tipo $\alpha$ instanciada para diferentes tipos conforme for necessário.

A inferência de tipos envolve dois processos principais: generalização e instanciação.
A generalização ocorre quando o sistema identifica que uma expressão pode ser tipada com um tipo mais geral, permitindo que seja reutilizada de maneira polimórfica.
Já a instanciação ocorre quando um tipo polimórfico é aplicado a um tipo concreto, especializando-o para um uso específico.
Esse mecanismo garante a flexibilidade do sistema, ao mesmo tempo que mantém a segurança garantida pela inferência de tipos.

Por exemplo, considere a expressão $\texttt{let id} \ = \lambda x.x \ \texttt{in} \ (\texttt{id} \ \texttt{1}, \ \texttt{id} \ \texttt{`a'})$.
O sistema generaliza o tipo de $\texttt{id}$ para $\forall \alpha. \alpha \rightarrow \alpha$, e instancia este tipo tanto para inteiros quanto para caracteres nas duas aplicações subsequentes.

Outro conceito presente no sistema Damas-Milner é a substituição de tipos, onde estes são mapeados para outros tipos ou para variáveis de tipo.
Formalmente, uma substituição de tipos é representada como um mapeamento finito de variáveis de tipo para tipos, denotado por $S$, e pode ser escrito na forma $[ \alpha_1 \mapsto \tau_1, \alpha_2 \mapsto \tau_2, \ldots, \alpha_n \mapsto \tau_n ]$.
Aqui, $\alpha_i$ são variáveis de tipo distintas e $\tau_i$ são os tipos correspondentes.
Em outras palavras, $S$ associa cada variável de tipo $\alpha_i$ a um tipo $\tau_i$ específico.

A aplicação de uma substituição $S$ em um tipo $\tau$, denotada por $S\tau$, resulta na substituição de todas as ocorrências livres de $\alpha_i$ em $\tau$ por $\tau_i$.
Esse conceito de substituição é fundamental para o processo de instanciação de tipos, que será discutido a seguir.
A definição formal da aplicação de substituições é dada por:

\begin{figure}[ht!]
  \caption{Regras de aplicação da substituição de tipos}
  \centering
  \[
    \begin{array}{rcl}
      S\alpha_i                    & \equiv & \tau_i,                                                                          \\[8pt]
      S\alpha                      & \equiv & \alpha, \quad \text{se } \alpha \notin \{\alpha_1, \alpha_2, \ldots, \alpha_n\}, \\[8pt]
      S(\tau_1 \rightarrow \tau_2) & \equiv & S\tau_1 \rightarrow S\tau_2,                                                     \\[8pt]
      S(\forall \alpha.\sigma)     & \equiv & S' \sigma, \quad \text{onde } S' = S \setminus [\alpha \mapsto \_].
    \end{array}
  \]\label{fig:substituicao-tipos}
  \small{Fonte:~\cite{CASTRO2019}}
\end{figure}

onde o símbolo de subtração ($\setminus$) indica que a substituição $S'$ é a substituição $S$ restrita ao conjunto de mapeamentos que não envolvem a variável $\alpha$.

A instanciação de tipos é um processo em que um esquema de tipo $\sigma = \forall \alpha_1 \ldots \alpha_m. \tau$ é transformado em um tipo específico substituindo suas variáveis quantificadas por tipos concretos.
Se $S$ é uma substituição, então $S\sigma$ é o esquema de tipo obtido substituindo cada ocorrência livre de $\alpha_i$ em $\sigma$ por $\tau_i$, renomeando as variáveis genéricas de $\sigma$, se necessário.
O tipo resultante $S\sigma$ é chamado de uma instância de $\sigma$~\cite{DAMAS1982}.
Esse processo é essencial para adaptar esquemas de tipos polimórficos a situações específicas em um programa, mantendo a flexibilidade e segurança do sistema de tipos.

Um esquema de tipo também pode ter uma instância genérica $\sigma' = \forall \beta_1 \ldots \beta_n. \tau'$, se existir uma substituição $[ \tau_i / \alpha_i ]$ tal que $\tau' = [\tau_i / \alpha_i]\tau$, e as variáveis $\beta_j$ não aparecem livres em $\sigma$.
Nesse caso, escrevemos $\sigma > \sigma'$, indicando que $\sigma$ é mais geral do que $\sigma'$.
Vale notar que a instanciação atua sobre variáveis livres, enquanto a instanciação genérica lida com variáveis ligadas.

O sistema de tipos de Damas e Milner é definido por um conjunto de regras de inferência de tipos, apresentadas na Figura~\ref{eq:type-inference}, que são usadas para determinar os tipos das expressões no sistema.
Essas regras são representadas por meio de julgamentos de tipos da forma $\Gamma \vdash e: \sigma$, onde $\Gamma$ é o contexto, um conjunto de suposições com pares de variáveis e seus tipos: $(e, \sigma)$, $e$ é a expressão sendo tipada, e $\sigma$ ou $\tau$ são o tipo inferido no contexto.

\begin{figure}[ht!]
  \caption{Regras de Inferência do sistema Damas-Milner}
  \centering
  \[
    \text{TAUT:} \quad \frac{x : \sigma \in \Gamma}{\Gamma \vdash x : \sigma}
  \]

  \[
    \text{ABS:} \quad \frac{\Gamma, x : \tau \vdash e : \tau'}{\Gamma \vdash (\lambda x. e) : \tau \to \tau'}
  \]

  \[
    \text{APP:} \quad \frac{\Gamma \vdash e : \tau' \to \tau \quad \Gamma \vdash e' : \tau'}{\Gamma \vdash (e \ e') : \tau}
  \]

  \[
    \text{LET:} \quad \frac{\Gamma \vdash e : \sigma \quad \Gamma, x : \sigma \vdash e' : \tau}{\Gamma \vdash (\texttt{let } x = e \ \texttt{in} \ e') : \tau}
  \]

  \[
    \text{INST:} \quad \frac{\Gamma \vdash e : \sigma}{\Gamma \vdash e : \sigma'} \quad \scriptstyle (\sigma\ >\ \sigma')
  \]

  \[
    \text{GEN:} \quad \frac{\Gamma \vdash e : \sigma}{\Gamma \vdash e : \forall \alpha \sigma} \quad \scriptstyle (\alpha \text{ não livre em } \Gamma)
  \]

  \small{Fonte: o autor. Adaptado de~\cite{DAMAS1982}}\label{eq:type-inference}
\end{figure}

As regras de inferência são interpretadas de baixo para cima.
Por exemplo, na regra da tautologia (TAUT):
\[
  \frac{x : \sigma \in \Gamma}{\Gamma \vdash x : \sigma}
\]
significa que, se em um contexto $\Gamma$, a variável $x$ possui o tipo $\sigma$, então podemos concluir que $x$ tem o tipo $\sigma$ no mesmo contexto.
Isso reflete o fato de que a associação de tipos no contexto é preservada.

Na regra de generalização (GEN), a condição de que $\alpha$ não seja livre em $\Gamma$ assegura que o tipo generalizado não dependa de nenhum tipo específico presente no contexto.
Isso permite que o tipo $\forall \alpha. \sigma$ seja usado de forma polimórfica em diferentes partes do programa.

Essas regras garantem a solidez do sistema, preservando a segurança dos tipos ao inferir automaticamente os tipos mais gerais possíveis para as expressões.

Antes de apresentar o Algoritmo W, é importante observar que ele é uma implementação prática das regras de inferência aqui descritas, usando o conceito de unificação para resolver as equações de tipo geradas durante a inferência.
A seguir, será discutido em detalhes o funcionamento do Algoritmo W.

\subsection{Algoritmo W}\label{subsec:w-algo}

O Algoritmo W, introduzido em~\citeonline{DAMAS1982}, é um algoritmo eficiente\footnote{Embora seja eficiente na grande maioria dos casos, há situações em que o Algoritmo W apresenta desempenho exponencial, como ocorre em expressões com recursão polimórfica, conforme discutido em~\cite{CRISTIANO2004}.} para inferência de tipos em linguagens de programação funcional.
Ele se baseia no processo de unificação para solucionar equações de tipos geradas durante a análise de expressões, atribuindo os tipos mais gerais possíveis, ou seja, os tipos mais polimórficos que ainda garantem a consistência do sistema.

A unificação é o processo de encontrar uma substituição de variáveis de tipo que torna dois tipos dados equivalentes.
Formalmente, dados dois tipos $\tau_1$ e $\tau_2$, a unificação procura uma substituição $S$ tal que $S\tau_1 = S\tau_2$.
Se tal substituição existe, os tipos são considerados unificáveis e $S$ é chamada de solução unificadora.
Caso contrário, os tipos são incompatíveis.

Seja o algoritmo de unificação, $unify$, que recebe como argumento um conjunto de tipos simples e retorna uma substituição, melhor explicada na Figura~\ref{fig:substituicao-tipos}, pode ser descrito pelas seguintes regras:

% \begin{figure}
%   \caption{Algoritmo de unificação no formato de função.}
%   \[
%   \begin{array}{ll}
%   (1) & \unify([\: ]) = [\: ]\\
%   (2) & \unify((\alpha \eeq \alpha) :: \mathbb{C}) = \unify(\mathbb{C})\\
%   (3) & \unify((\alpha \eeq\tau) :: \mathbb{C}) = 
%         \If \occurs(\alpha,\tau) \Then \fail \Else \unify([\alpha\mapsto\tau]\mathbb{C})\circ [\alpha \mapsto \tau]\\
%   (4) & \unify((\tau \eeq\alpha) :: \mathbb{C}) = 
%         \If \occurs(\alpha,\tau) \Then \fail \Else \unify([\alpha\mapsto\tau]\mathbb{C})\circ [\alpha \mapsto \tau]\\
%   (5) & \unify((\tau_1\to\tau_2 \eeq \tau\to\tau')::\mathbb{C}) = 
%         \unify((\tau_1\eeq\tau)::(\tau_2\eeq\tau')::\mathbb{C}) \\
%   (6) & \unify((\tau \eeq \tau')::\mathbb{C}) = \If \tau\equiv\tau' \Then \unify(\mathbb{C}) \Else \fail 
%   \end{array}
%   \]
%       \small{Fonte: autor. Adaptado de~\cite{RIBEIRO2016}}\label{algo:unify}
%   \end{figure}

\begin{figure}
\caption{Algoritmo de unificação no formato de função.}
\[
\begin{array}{ll}
  \unify(\alpha, \alpha) = [\: ]\\
  \unify(\alpha, \tau) = \If \occurs(\alpha,\tau) \Then \fail \Else [\alpha\mapsto\tau]\\
  \unify(\tau, \alpha) = \If \occurs(\alpha,\tau) \Then \fail \Else [\alpha\mapsto\tau]\\
  \unify(\tau_1 \to \tau_2, \tau_1' \to \tau_2') = \unify(\tau_1, \tau_1') \circ \unify(\tau_2, \tau_2')\\
  % \unify(\tau, \tau') = \If \tau\equiv\tau' \Then [\: ] \Else \fail\\
\end{array}
\]
    \small{Fonte: autor. Adaptado de~\cite{RIBEIRO2016}}\label{algo:unify}
\end{figure}

Note que o algoritmo de unificação utiliza o algoritmo de verificação de
ocorrência apresentado na Figura~\ref{algo:occurs}, responsável por
garantir que uma variável de tipo não apareça em um tipo, evitando assim,
definições circulares em uma substituição, como por exemplo em
$[\alpha\mapsto\alpha\to\alpha]$~\cite{RIBEIRO2016}.

\begin{figure}
    \caption{Algoritmo \textit{occurs} no formato de função.}
\[ \begin{array}{llp{.7\textwidth}}
      \occurs (\alpha,\tau_1\to\tau_2) & = & $\occurs(\alpha,\tau_1) \lor occurs(\alpha,\tau_2)$ \\
      \occurs (\alpha,\alpha)         & = & $\True$ \\
      \occurs (\alpha, \tau)          & = & $\False$ \\
  \end{array}
\]
\small{Fonte: autor. Adaptado de~\cite{RIBEIRO2016}}\label{algo:occurs}
\end{figure}

!!!ADICIONAR DESCRIÇÃO ALGORITMO UNIFICAÇÃO!!!

O Algoritmo W, descrito na Figura~\ref{algo:w} é um método utilizado para inferência de tipos em expressões de linguagens funcionais.
Ele atribui os tipos mais gerais possíveis a cada subexpressão, garantindo a consistência com as operações definidas.
O algoritmo combina a unificação com regras de inferência de tipos para deduzir o tipo de uma expressão, explorando o polimorfismo de forma eficiente.

\begin{figure}[ht!]
  \caption{Algoritmo W no formato de função.}
  \centering
  \begin{align*}
     & \texttt{$W$($\Gamma$, $x$) = }                                                                                     \\
     & \qquad{}\qquad{}\texttt{se $\Gamma(x) = \forall$ $\alpha_1 \ldots \alpha_n$ $.$ $\tau$, então}                     \\
     & \qquad{}\qquad{}\qquad{}\texttt{retorna ([$\alpha_i$ $\mapsto$ $\tau_i$]$\tau$, Id), onde $\alpha_i'$ fresh}       \\
     & \qquad{}\qquad{}\texttt{senão}                                                                                     \\
     & \qquad{}\qquad{}\qquad{}\texttt{falha}                                                                             \\
    \\
     & \texttt{$W$($\Gamma$, $e$ $e'$) = }                                                                                \\
     & \qquad{}\qquad{}\texttt{($\tau$, $S_1$) $\leftarrow$ $W$($\Gamma$, $e$)}                                           \\
     & \qquad{}\qquad{}\texttt{($\tau'$, $S_2$) $\leftarrow$ $W$($S_1\Gamma$, $e'$)}                                      \\
     & \qquad{}\qquad{}\texttt{S $\leftarrow$ unify($S_2\tau$, $\tau'$ $\to$ $\alpha$), onde $\alpha$ fresh}              \\
     & \qquad{}\qquad{}\texttt{retorna (S$\alpha$, S $\circ$ $S_2$ $\circ$ $S_1$)}                                        \\
    \\
     & \texttt{$W$($\Gamma$, $\lambda x.e$) = }                                                                           \\
     & \qquad{}\qquad{}\texttt{($\tau$, $S$) $\leftarrow$ $W$($\Gamma$[$x$ $:$ $\alpha$], $e$), onde $\alpha$ fresh}      \\
     & \qquad{}\qquad{}\texttt{retorna ($S$($\alpha$ $\to$ $\tau$), $S$)}                                                 \\
    \\
     & \texttt{$W$($\Gamma$, let $x = e$ in $e'$) = }                                                                     \\
     & \qquad{}\qquad{}\texttt{($\tau$, $S_1$) $\leftarrow$ $W$($\Gamma$, $e$)}                                           \\
     & \qquad{}\qquad{}\texttt{($\tau'$, $S_2$) $\leftarrow$ $W$($S_1$ $\Gamma$[$x$ $:$ gen($S_1\Gamma$, $\tau$)], $e'$)} \\
     & \qquad{}\qquad{}\texttt{retorna ($\tau'$, $S_2$ $\circ$ $S_1$)}                                                    \\
  \end{align*}
  \small{Fonte:~\cite{CASTRO2019}}\label{algo:w}
\end{figure}

O funcionamento deste pode ser analisado para os diferentes tipos de expressões a seguir.
Para uma variável $x$, o algoritmo verifica se existe uma atribuição de tipo para $x$ no contexto $\Gamma$.
Se $x$ estiver associada a um tipo polimórfico da forma $\forall \alpha_1 \ldots \alpha_n . \tau$, realiza-se a substituição das variáveis ligadas $\alpha_i$ por novos tipos frescos, que não aparecem em outros lugares do contexto, e retorna-se o tipo resultante, juntamente com a substituição identidade.
Caso $x$ não esteja no contexto, a inferência falha.

Para uma aplicação de função $e\ e'$, o algoritmo infere recursivamente os tipos das subexpressões $e$ e $e'$.
A partir desses tipos, unifica o tipo de $e$ com um tipo função $\tau' \to \alpha$, onde $\alpha$ é um novo tipo variável introduzido durante a unificação.
A substituição resultante é então aplicada ao tipo inferido de $e$ e o algoritmo retorna o tipo correspondente à aplicação.

Quando a expressão é uma abstração $\lambda x . e$, o algoritmo atualiza o contexto adicionando uma nova variável de tipo para $x$ e procede inferindo o tipo de $e$.
O tipo função resultante, $\alpha \to \tau$, é então retornado como o tipo inferido para a abstração.

No caso de expressões do tipo \texttt{let}, onde uma variável é definida localmente, o algoritmo primeiro infere o tipo da expressão vinculada, seguido pelo tipo do corpo da expressão.
O contexto é atualizado para incluir a variável definida com um tipo generalizado, permitindo polimorfismo na expressão resultante.
A generalização é aplicada ao tipo inferido, de forma que as variáveis de tipo que não estão presentes no contexto sejam quantificadas, garantindo assim um nível adequado de polimorfismo na inferência de tipos.

Essas etapas garantem que o Algoritmo W seja capaz de inferir tipos de forma eficiente, atribuindo os tipos mais polimórficos possíveis para expressões em linguagens funcionais e explorando as capacidades do sistema de tipos.
