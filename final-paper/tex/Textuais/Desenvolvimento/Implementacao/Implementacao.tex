\section{Implementação}\label{sec:implementacao}

Partindo para a parte prática do trabalho, os módulos e funções serão apresentados de maneira gradual, de modo a facilitar o entendimento do fluxo inteiro do programa.
Vale destacar aqui que é feito também a implementação do sistema Hindley-Milner, porém, como este foi feito somente para poder usar o sistema de continuações de maneira mais cômoda, não será entrado em detalhes nos que dizem respeito à esse sistema de tipos.
Inicialmente, na Subseção~\ref{subsec:cps-adt}, será discutido sobre a maneira como foi representado o sistema de tipos.
Em sequência, a Subseção~\ref{subsec:cps-translations} trará detalhes sobre a implementação das funções de tradução de cálculo lambda simplesmente tipado para CPS.
Posteriormente, a Subseção~\ref{subsec:cps-inferer} irá tratar da inferência em si, juntamente da verificação do tipo.
Por fim, a geração de código abordada na Subseção~\ref{subsec:cps-code-gen} serve como uma maneira de se testar a tradução do código.

\subsection{Tipos de Dados}\label{subsec:cps-adt}

Para representar os comandos, bem como os tipos do sistema, foram utilizados tipos de dados algébricos (ADTs, do inglês \textit{algebraic data types}), disponíveis no Listing~\ref{code:cps-adt}.

\lstinputlisting[style=haskell, label=code:cps-adt, caption={Definição dos tipos de dados}]{Code/Type-Inferer/CPSTyping.hs}
Para os comandos, dois construtores podem ser observados, o \texttt{Jump} e o \texttt{Bind}, sendo responsáveis por construir respectivamente os comandos de \textit{Jump}, onde há um salto \texttt{Id} com \texttt{[Id]} parâmetros.
Sendo assim, o salto $k(x)$ seria representado por este ADT: $\mathtt{Jump\ k\ [x]}$.
E o comando \textit{Bind}, onde há outro \texttt{Command} definido recursivamente, a função \texttt{Id} com argumentos \texttt{[Id]} e por fim outro \texttt{Command} também definido recursivamente.
O \textit{bind} $\mathtt{let}\ k(x) = k(x)\ \mathtt{in}\ k(x)$, portanto seria definido pelo seguinte ADT: $\mathtt{Bind\ (Jump\ k\ [x])\ k\ [x]\ (Jump\ k\ [x])}$.

Os tipos, foram representados com dois tipos algébricos diferentes, um para os monotipos $\mathtt{CPSMonoType}$, e uma para os politipos $\mathtt{CPSPolyType}$.
As variáveis do monotipo são construídas a partir dos construtores $\mathtt{TVar\ Id}$ e $\mathtt{TInt}$, onde no primeiro, o $\mathtt{Id}$ serve para obter a variável de tipo atribuída àquela variável, enquanto que as funções que não retornam são representadas as partir do construtor de negação $\mathtt{TNeg\ [CPSMonoType]}$ sendo os argumentos dela definidos recursivamente sobre si.
O contexto por sua vez, é um tipo que utiliza o $\mathtt{Data.Map}$ disponível no pacote \textit{containers} para mapear uma variável para um tipo polimórfico.
As substituições são representadas utilizando o mesmo $\mathtt{Data.Map}$, onde desta vez é mapeado uma variável de tipo para um monotipo.

\subsection{Traduções}\label{subsec:cps-translations}
Para facilitar os testes efetuados e ainda tornar o fluxo de execução mais direto, foram implementas as funções de tradução de expressões e de tipos de acordo com as definições das Seções~\ref{subsec:cps-translation} e~\ref{subsec:typed-cps-translation}.

\lstinputlisting[style=haskell, label=cps:cbn-initial-cont, caption={Continuação inicial}]{Code/Type-Inferer/CPS_initial_cont.hs}
Para todas as computações, um contexto inicial precisa conter a continuação inicial.
Este será o objeto a ter seu tipo inferido.
Afim de praticidade, esta continuação será sempre a mesma, dada por `k', como mostrado no Código~\ref{cps:cbn-initial-cont}.

\lstinputlisting[style=haskell, label=cps:cbn-expr-translation, caption={Tradução das expressões para CBN}]{Code/Type-Inferer/CPS_CBN_expr_translation.hs}
No Código~\ref{cps:cbn-expr-translation}, são apresentadas as funções responsáveis para traduzir as expressões para CBN.
O ponto de partida desta computação será a função $\mathtt{cbnExprTrans \dblcolon Expr \to Command}$, onde ela irá receber a expressão em cálculo-$\lambda$ e retornará o comando traduzido.
Sua responsabilidade é criar as mônadas de estado que farão o controle do índice das variáveis frescas necessárias e chamar as funções de tradução passando a continuação inicial.

Tomando como exemplo a função identidade em cálculo-$\lambda$ ($\lambda x.\ x$), é possível perceber como os termos crescem em CPS.
Isto torna o desenvolvimento diretamente neste cálculo não apropriado, mas ainda, nota-se que a implementação da função de tradução é bastante direta em relação a sua definição formal.
Os únicos pontos de atenção são em relação à geração das variáveis frescas, mas que como foi dito anteriormente, não eram completamente necessários, visto que eles são ligados imediatamente.
Ao traduzir então a função, tem-se que o equivalente em CPS é o mostrado no Código~\ref{cps:id-cps-cbn}.
\lstinputlisting[style=haskell, label=cps:id-cps-cbn, caption={Tradução da função identidade em CBN}]{Code/Type-Inferer/CPS/id-cbn.cps}
Este comportamento fica ainda mais visível quando uma função um pouco maior é traduzida, por exemplo o numeral de Church dois ($\lambda f.\ \lambda x.\ f\ (f\ x)$).
Sua tradução portanto é mostrada no Código~\ref{cps:church-two-cps-cbn}.
\lstinputlisting[style=haskell, label=cps:church-two-cps-cbn, caption={Tradução do numeral de Church ``2'' em CBN}]{Code/Type-Inferer/CPS/church-two-cbn.cps}
De maneira semelhante, foram feitas as mesmas funções utilizadas no \textit{call-by-name}, porém adaptadas para o CBV, respeitando as diferenças presentes na definicão formal da função.
\lstinputlisting[style=haskell, label=cps:cbv-expr-translation, caption={Tradução das expressões para CBV}]{Code/Type-Inferer/CPS_CBV_expr_translation.hs}
A partir dessas diferenças nas definições, nota-se também particularidades nas traduções destas, por exemplo ao traduzir a mesma função identidade, em CBV, é obtido o resultado apresentado no Código~\ref{cps:id-cps-cbv}.
\lstinputlisting[style=haskell, label=cps:id-cps-cbv, caption={Tradução da função identidade em CBV}]{Code/Type-Inferer/CPS/id-cbv.cps}
Neste exemplo da função identidade, pouca diferença entre as duas estratégias de avaliação pode ser notada.
Isso se deve não ao tamanho da expressão, e sim dos elementos desta.
Neste caso, há somente uma abstração lambda com uma variável.
A seguir, é exibido novamente o numeral de Church dois, porém para o CBV, onde mais diferenças podem ser observadas.
O motivo disto é os elementos da função, que diferentemente da identidade, conta com mais construtores para representá-la, exibido no Código~\ref{cps:church-two-cps-cbv}.
\lstinputlisting[style=haskell, label=cps:church-two-cps-cbv, caption={Tradução do numeral de Church ``2'' em CBV}]{Code/Type-Inferer/CPS/church-two-cbv.cps}

Mostrado anteriormente no Código~\ref{cps:id-cps-cbn}, a expressão CPS resultante da tradução da função identidade em CBN difere da mesma traduzida em CBV, presente no Código~\ref{cps:id-cps-cbv}.
O mesmo pode ser observado ao traduzir a função identidade tipada.
Por exemplo, ao executar a função para a identidade em CBN, o tipo obtido é $\neg\neg(\neg\neg\alpha,\ \neg\alpha)$.
Já em CBV, para a mesma função, tem-se que o tipo traduzido é $(\alpha,\ \neg\alpha)$.

Para raciocinar sobre os tipos do sistema, tem que ser levado em consideração o que estes representam, contradições.
Ao tomar como exemplo o tipo resultante da função identidade em CPS a partir da tradução por valor (CBV), isto é, $(\alpha,\ \neg\alpha)$\footnote{Apesar de estar sendo traduzido o tipo da função identidade, como esta tradução é feita para os tipos simples, note que aqui, está sendo representado monomorficamente pois não há a generalização do $\alpha$.}, deve-se pensar que este representa o absurdo de ter $\alpha$ como argumento e $\neg\alpha$ como continuação, ao mesmo tempo.

\lstinputlisting[style=haskell, label=cps:cbn-type-translation, caption={Tradução dos tipos para CBN}]{Code/Type-Inferer/CPS_CBN_type_translation.hs}
Aqui no Código~\ref{cps:cbn-type-translation}, a função responsável por traduzir um tipo utilizando a estratégia \textit{call-by-name}, a função $\mathtt{cbvTypeTranslation \dblcolon LambdaMonoType \to CPSPolyType}$ irá receber um tipo simples no cálculo-$\lambda$ simplesmente tipado, e retornar um tipo polimórfico em CPS.
Perceba que na definição, o retorno era um tipo simples em CPS. 
Essa diferença é justificada ao se observar o corpo desta função, onde há a definição da função $\mathtt{cbvTrans}$.
Esta é quem efetivamente faz a computação, pois é nela que acontece o casamento de padrões para determinar o tipo sendo traduzido.
Ainda, esta função retorna um tipo polimórfico pelo fato de que em momento posterior a essa tradução, a subtipagem do tipo traduzido e do tipo inferido precisa ser verificada.

\lstinputlisting[style=haskell, label=cps:cbv-type-translation, caption={Tradução dos tipos para CBV}]{Code/Type-Inferer/CPS_CBV_type_translation.hs}
Para que essas duas funções de tradução de tipos (CBN e CBV) sejam usadas do mesmo modo, elas seguem a messma assinatura. 
Seus comportamentos são o mesmo, a diferir somente nas diferenças das definições da função, isto é, como é feita a tradução.

O sistema de tipos proposto neste trabalho, entretanto, é polimórfico, ou seja, aqui há a adição de variáveis de tipos quantificadas.
Sendo assim, esta função de tradução, se aplicando neste caso de uso, não necessariamente retornará o tipo mais geral de uma expressão, mas sempre um subtipo deste.

\subsection{Inferência}\label{subsec:cps-inferer}
A inferência de tipos é a etapa onde, em uma linguagem onde tipos são presentes, um termo tem seu tipado inferido sem anotação prévia.
Isto é, sem explicitar o tipo de um termo, este tem seu tipo deduzido.
Em um ambiente tipado com polimorfismo, a maior utilidade do inferidor de tipos é que este seja sempre o mais geral possível, tal que possa ser especializado para cada uso.
O algoritmo de inferência proposto para este sistema de tipos é capaz de inferir o tipo mais geral possível, conforme explicado mais detalhadamente na Seção~\ref{sec:formalizacao}.
Seu desenvolvimento foi direto como pode ser visto abaixo, no Código~\ref{cps:infer-command}.

\lstinputlisting[style=haskell, label=cps:infer-command, caption={Função principal de Inferência}]{Code/Type-Inferer/CPS_infer_command.hs}
Muitas das funções necessárias para a inferência do CPS são iguais as do Damas-Milner.
Por não ser o foco deste trabalho, explicações sobre estas serão omitidos.
Desta forma, o processamento para se inferir o tipo das continuações foi dividido em três principais funções.

A função $\mathtt{inferAtom \dblcolon Context \to Id \to TI\ CPSMonoType}$, como seu nome e assinatura indica, é a função responsável por inferir os átomos do termo, isto é, buscar as variáveis do contexto e retorná-las caso sejam literais, instanciá-las se forem variáveis polimórficas, ou então retornar erro caso esta não esteja presente.
Ao executar a função passando o contexto $\{\ x{:}\ \forall\alpha.\alpha \ \}$ e a variável $x$, a função irá instanciar uma nova variável de tipo $\beta$ (se esta for a próxima ainda não utilizada) e retornar este tipo monomórfico.
Ou ainda, caso a função seja chamada com o contexto $\{\ x{:}\ \forall\alpha.\alpha \ \}$ e a variável $k$, um erro $\mathtt{UnboundVariable\ ``k"\ \{\ x{:}\ \forall\alpha.\alpha \ \}}$ será exibido.

Já a função $\mathtt{inferCommand \dblcolon Context \to Command \to TI\ Substitution}$ é quem, a partir do contexto e comando, irá retornar a substituição que indica o tipo da continuação.
Tal qual a função anterior, nenhuma dificuldade grande foi encontrada aqui.
A inferência do salto foi a mais simples delas, onde cada linha da função se refere às premissas na definição, retornando então a unificação mais geral entre $\tau_1$ e $\vv{\tau_2}$.
Já para o \textit{binding}, algumas funções auxiliares foram necessárias principalmente para manipular o contexto ao extender este em diferentes momentos.
Além destas, funções para a generalização e por fim para a composição das subsituições também foram precisas.

O ponto de partida da inferência, $\mathtt{inferWithCtx \dblcolon Command \to TI\ CPSPolyType}$, além de definir o contexto inicial com a continuação inicial $k$, recebendo um tipo $\alpha$ qualquer, esta função aplicará a substituição obtida na inferência e a aplicará no contexto, de modo que o contexto final esteja atualizado com o tipo inferido da continuacão $k$.
Após isto, é feita a normalização do tipo polimórfico da substituição, isto é, limpar as variáveis de tipo utilizadas durante o processo que não são mais necessárias, afim de promover consistência e facilitar o entendimento.
Por exemplo, no caso onde o tipo polimórfico final seja $\forall\delta.\delta$, o retornado seria $\forall\alpha.\alpha$.

Uma etapa que não está diretamente relacionada com a inferência, e sim com os tipos em si, é a verificação da subtipagem do tipo traduzido e do tipo inferido.
Isto é, uma verificação se o tipo traduzido é um subtipo do tipo inferido, indicando diretamente se o algoritmo de inferência foi implementado corretamente.
\lstinputlisting[style=haskell, label=cps:subtyping, caption={Verificação de Subtipagem}]{Code/Type-Inferer/CPS_subtyping.hs}
O algoritmo de verificação de subtipos do Código~\ref{cps:subtyping} procura uma substituição $S$ tal que, ao aplicá-la em um dos tipos, ele se torne o outro.
Por exemplo, ao analisar os tipos $\mathtt{\tau_1 = \alpha \to \alpha}$ e $\tau_2 = \neg\alpha \to \neg\alpha$, é possível encontrar uma substituição $S = \{\ \alpha \mapsto \neg\alpha \ \}$, tal que $S\tau_1$ resulte em $\tau_2$.
A partir disto, pode ser dito que o tipo $\tau_2$ é um subtipo de $\tau_1$, onde o algoritmo retornaria com sucesso a substituição $S$.
Já olhando para outro exemplo, onde $\mathtt{\tau_1 = \alpha \to \neg\alpha}$ e $\mathtt{\tau_2 = \alpha \to \alpha}$, não é possível encontrar uma substituição $S$ tal que $S\tau_1$ seja $\tau_2$.
Desta forma, o algoritmo não retornaria uma substituição, falhando assim a verificação de subtipos.

\subsection{Geração de Código}\label{subsec:cps-code-gen}
% Explicar geração de código
% \lstinputlisting[style=haskell, label=cps:typing, caption={Definição dos tipos de dados}]{Code/Type-Inferer/CPSTyping.hs}
% Explicar o que faz

\subsection{Fluxo Principal}\label{subsec:cps-main-program}
O fluxo completo de execução do programa principal contempla todas as funções apresentadas nesta seção, com a adição de funções auxiliares.
Essas são aplicadas em sequência, de modo a realizar uma série de ações.

Inicialmente, é passado o caminho de um arquivo contendo um programa em cálculo-$\lambda$ com adição do `let'.
O conteúdo então é processado pelo \textit{parser} e representado pelos tipos de dados algébricos para o cálculo lambda.
Uma vez que o programa já está sendo representado pelos ADTs, e ainda tem seu tipo inferido, é possível iniciar o processamento descrito pelas funções apresentadas.
A primeira delas é a tradução para CPS tanto em \textit{call-by-name} quanto em \textit{call-by-value}, as exibindo logo em seguida.
Com as traduções da expressão feitas, o código Haskell já pode ser gerado, salvando assim no diretório \texttt{output} com mesmo nome do arquivo de entrada.
Como passo posterior, tem-se a tradução dos tipos para ambas as estratégias de avaliação.
Os passos finais envolvem a inferência de ambas as traduções, juntamente da verificação de subtipagem, onde esta informará se o tipo traduzido é um subtipo do inferido.

\lstset{extendedchars=false, escapeinside=''}
\lstinputlisting[style=output, label=cps:main-execution, caption={Execução do programa principal}]{Code/Type-Inferer/CPS_main_execution.cps}
Ao executar o programa do Código~\ref{cps:main-execution} com o comando \texttt{cabal run}, passando também o arquivo de entrada \texttt{input/church-zero.in}, é processado e exibida todas as informações que foram citadas anteriormente, inclusive a geração do código Haskell em \texttt{output/church-zero.hs}.
É possível perceber que na saída do programa, é mostrado o tipo traduzido (na sequência da mensagem ``\textit{Expected Continuation Type:}'') e o tipo inferido (que sucede a mensagem ``\textit{Inferred Continuation Type:}'').
Logo em seguida, o questionamento ``\textit{Do the types match?}'' é o trecho da saída que compete à subtipagem, retornando ``\textit{Yes}'' caso esse seja um subtipo deste, o que indica uma inferência compatível com a tradução ou ``\textit{No}'' caso a verificação falhe, indicando uma inferência incorreta.

\lstset{extendedchars=false, escapeinside=''}
\lstinputlisting[style=output, label=cps:execution-generated, caption={Execução do programa gerado}]{Code/Type-Inferer/CPS_execution_generated.cps}
Ainda, a compilação e execução do código Haskell gerado pode ser conferida no Código~\ref{cps:execution-generated} acima, ilustrando exatamente o comportamento detalhado anteriormente.

