\chapter{Desenvolvimento}\label{ch:desenvolvimento}

Será abordada, neste capítulo, a contribuição prática do trabalho, apresentando a formalização do sistema de tipos proposto, sua implementação na linguagem Haskell, e os resultados experimentais obtidos.
No que diz respeito ao sistema de tipos, em virtude da limitação de tempo e do escopo deste trabalho, não foi feita a prova de que este é correto ou completo.
Ao invés disso, há evidências empíricas, baseadas em testes, para suportar o correção do sistema de tipos proposto.
Uma explicação mais detalhada sobre esta parte será dada na Seção~\ref{sec:formalizacao}.

Na Seção~\ref{sec:implementacao}, serão descritas as decisões de projeto que orientaram a implementação. 
O ambiente de desenvolvimento utilizado foi o compilador \textit{The Glorious Glasgow Haskell Compilation System} (GHC), na versão 9.12.2 em conjunto com o gerenciador de projetos Cabal na versão 3.14.1.
O código-fonte completo está disponível publicamente no repositório \textit{cps-type-inferer}\footnote{\url{https://github.com/bidinpithecus/cps-type-inferer}} do GitHub.
A Seção~\ref{sec:implementacao} terá um aprofundamento maior na implementação.

\newcommand{\Mgu}{\ensuremath{\textit{mgu}}}
\newcommand{\MguList}{\ensuremath{\textit{mguList}}}
\newcommand{\UnifyVar}{\ensuremath{\textit{varBind}}}
\newcommand{\HeadSep}{\ensuremath{\textit{:}}}
\newcommand{\Length}{\ensuremath{\textit{length}}}
\newcommand{\List}{\ensuremath{\textit{list}}}


\section{Formalização}\label{sec:formalizacao}

Em razão da natureza mais prática deste trabalho, a notação utilizada para representar o cálculo de continuações será a mesma utilizada por~\citeonline{appel1997shrinking}, a sintaxe do `let'.
O sistema de tipos formalizado aqui foi fortemente inspirado no sistema de Damas e Milner, explicado na Seção~\ref{sec:damas-milner}, onde suas regras foram adaptadas de modo que elas se enquadrem no sistema polimórfico baseado em continuações.
Em particular, o contexto $\Gamma$ associa variáveis a tipos polimórfico e define julgamentos distintos para representar átomos e comandos.
A distinção destes se mostra necessária uma vez que é levado em consideração o comportamento não retornável das continuações. 

A sintaxe do sistema conta com expressões e tipos usados no processo de tipagem e de inferência de tipos.
Abaixo, segue a gramática das expressões e tipos presentes\footnote{Vale destacar que, todas as formalizações presentes aqui nesta seção, foram feitas pelo coorientador em reunião juntamente do autor, onde esse explicava suas motivações para atingir o resultado. Ainda, no momento da produção deste trabalho, não foi feita uma publicação contendo estas formalizações para que seja devidamente referenciada.}:

\phantom{Newline}

\begin{tabular}{lccl}
  Átomos & $a$ & $\Coloneqq$ & $x \enspace|\enspace n$ \\
  Comandos & $b$ & $\Coloneqq$ & $x(\vv{a}) \enspace|\enspace \mathtt{let}\ x(\vv{x}) = b\ \mathtt{in}\ b$ \\
  \\
  Tipos monomórficos & $\tau$ & $\Coloneqq$ & $\alpha \enspace|\enspace \mathtt{int} \enspace|\enspace \neg\vv{\tau}$ \\
  Tipos polimórficos & $\sigma$ & $\Coloneqq$ & $\forall\vv{\alpha}.\tau$ \\
  Contexto & $\Gamma$ & $\Coloneqq$ & $\cdot \enspace|\enspace \Gamma,\ x{:}\ \sigma$ \\
\end{tabular}\label{cps-type-system}

\phantom{Newline}

\noindent Na sintaxe apresentada, $a$ representa os átomos. Isto é, variáveis do programa ($x$) e literais inteiros ($n$) formam os elementos primitivos do sistema.
Os comandos $b$, por sua vez, são as expressões, explicadas com mais detalhes na Seção~\ref{subsec:cps}, sendo a primeira o \textit{jump}, e a segunda o \textit{bind}.
Três elementos distintos compõem os tipos presentes neste sistema.
Os tipos monomórficos ($\tau$), são os tipos que não possuem quantificação (monomórficos), podendo ser variáveis de tipo ($\alpha$), tipos numéricos inteiros ($\mathtt{int}$), ou ainda, tipos negados ($\neg\vv{\tau}$), usados para representar funções que retornam absurdos.
Já os tipos polimórfico ($\sigma$), são responsáveis por garantir a quantificação universal de variáveis de tipos (polimórficos).
Por fim, o contexto ($\Gamma$) contém o mapeamento de cada variável para um tipo polimórfico ($\sigma$).

As regras sintáticas de tipagem do sistema de tipos, inspiradas no Sistema Damas-Milner são ilustradas a seguir:

\phantom{Newline}

\fbox{$\Gamma\vdash a{:}\ \tau$}

\begin{prooftree}
    \RightLabel{$\mathtt{[Var]}$}
    \AxiomC{$x{:}\ \sigma \in \Gamma$}
    \AxiomC{$\sigma \sqsubseteq \tau$}
    \BinaryInfC{$\Gamma\vdash x{:}\ \tau$}
\end{prooftree}
\begin{prooftree}
    \RightLabel{$\mathtt{[Int]}$}
    \AxiomC{}
    \UnaryInfC{$\Gamma\vdash n{:}\ \mathtt{int}$}
\end{prooftree}

\phantom{Newline}

\fbox{$\Gamma\vdash b$}

\begin{prooftree}
    \RightLabel{$\mathtt{[Jump]}$}
    \AxiomC{$\Gamma\vdash k{:}\ \neg\vv\tau$}
    \AxiomC{$\Gamma\vdash \vv{a}{:}\ \vv{\tau}$}
    \BinaryInfC{$\Gamma\vdash k(\vv{a})$}
\end{prooftree}

\begin{prooftree}
    \RightLabel{$\mathtt{[Bind]}$}
    \AxiomC{$\Gamma, \vv{x}{:}\ \vv{\tau}\vdash c$}
    \AxiomC{$\Gamma,\ k{:}\ \overline{\Gamma}(\neg\vv\tau)\vdash b$}
    \BinaryInfC{$\Gamma\vdash\mathtt{let}\ k(\vv{x}) = c\ \mathtt{in}\ b$}
\end{prooftree}

Aqui, a regra $\mathtt{[Var]}$ define como tipo de uma variável uma instância do tipo (possivelmente polimórfico) que está associado a variável no contexto de tipos.
O símbolo $\sqsubseteq$ denota essa relação de ordem, indicando que o tipo $\tau$ é menos geral que $\sigma$.
Assim, em um contexto $\Gamma$, uma variável $x$ terá tipo $\tau$ caso esta esteja presente no contexto.
A regra $\mathtt{[Int]}$, é direta.
Em um contexto $\Gamma$, um literal inteiro terá um tipo $\mathtt{int}$.
Por exemplo, se $x{:}\ \forall\vv\alpha.\tau\in\Gamma$, então $\Gamma\vdash x{:}\ \tau$ (após instanciação adequada das variáveis de tipo).

As continuações, como discutido na Seção~\ref{subsec:cps}, representam fluxos de controle que não retornam valores.
Uma vez que a continuação pode ser interpretada como o próximo passo de uma computação, e a computação se dá por contradições, a continuação em si não possui um tipo, ela representa um absurdo.
Então, pode se dizer que a continuação é uma testemunha de que aquilo é um absurdo.

A regra $[\mathtt{Jump}]$ portanto, diz que sob um contexto $\Gamma$, se $k{:}\ \neg(\tau_1,\dots,\tau_n)$ com $n$ argumentos e cada argumento $a_i$ tiver um tipo correspondente $\tau_i$, então $k(\vv{a})$ é válido, ou seja, o salto $k$ com os argumentos $\vv{a}$ é testemunha de uma contradição.
De modo semelhante para o $\mathtt{[Bind]}$, as premissas $c$ e $b$ onde $c$ está sob o contexto $\{\ \Gamma\ \cup\ \{\ \vv{x}{:}\ \vv{\tau}\ \}\ \}$, e $b$ sob o contexto $\{\ \Gamma\ \cup\ \{\ k{:}\ \overline{\Gamma}(\neg\vv\tau)\ \}\ \}$, são testemunhas de que o comando $\mathtt{let}\ k(\vv{x}) = c\ \mathtt{in}\ b$ é uma contradição.
Assim como o `let' introduz o polimorfismo no sistema Damas-Milner, a generalização $\overline{\Gamma}(\neg\vv{\tau})$ presente na premissa do $\mathtt{[Bind]}$ quantifica as variáveis livres de $\vv{\tau}$ em $\Gamma$, estendendo o polimorfismo também ao CPS.

O algoritmo de inferência de tipos segue o mesmo esquema de Damas-Milner (algoritmo W) adaptado ao CPS.
Assim como o W, este faz uso do unificador mais geral, retornando sempre que existir o tipo mais genérico das expressões pertencentes a este sistema.
Abaixo, tem-se sua definição:

\phantom{Newline}

\fbox{$\Gamma\vdash_W a{:}\ \tau$}

\begin{prooftree}
    \AxiomC{$x{:}\ \sigma \in \Gamma$}
    \AxiomC{$\tau = \mathit{inst}(\sigma)$}
    \BinaryInfC{$\Gamma \vdash_W x{:}\ \tau$}
\end{prooftree}

\begin{prooftree}
    \AxiomC{}
    \UnaryInfC{$\Gamma \vdash_W n{:}\ \mathtt{int}$}
\end{prooftree}

\phantom{Newline}

\fbox{$\Gamma\vdash_W b \Rightarrow S$}

\begin{prooftree}
    \AxiomC{$\Gamma \vdash_W k{:}\ \tau_1$}
    \AxiomC{$\Gamma \vdash_W \vv{a}{:}\ \vv{\tau_2}$}
    \AxiomC{$S = \mathit{mgu}(\tau_1, \neg\vv{\tau_2})$}
    \TrinaryInfC{$\Gamma \vdash_W k(\vv{a}) \Rightarrow S$}
\end{prooftree}

\begin{prooftree}
    \AxiomC{$\vv{\tau} = \vv{\mathit{newvar}}$}
    \AxiomC{$\Gamma, \vv{x}{:}\ \vv{\tau} \vdash_W c \Rightarrow S_1$}
    \AxiomC{$\sigma = \overline{S_1\Gamma}(S_1 \neg\vv{\tau})$}
    \AxiomC{$S_1\Gamma, k{:}\ \sigma\vdash_W b \Rightarrow S_2$}
    \QuaternaryInfC{$\Gamma \vdash_W \mathtt{let}\ k(\vv{x}) = c\ \mathtt{in}\ b \Rightarrow S_2 \circ S_1$}
\end{prooftree}
Para as regras de inferência dos átomos, tal qual o sistema de tipos definido anteriormente, o algoritmo com uma variável $x$ de tipo polimórfico $\sigma$ pertencente ao contexto $\Gamma$ retornará um tipo monomórfico $\tau$ sob o mesmo contexto onde $\tau$ será a instanciação deste tipo $\sigma$.
Para o tipo numérico, não são necessárias premissas, algoritmo simplesmente devolve o tipo $\mathtt{int}$.
\begin{prooftree}
    \AxiomC{$a{:}\ \forall\alpha.\alpha \in \Gamma$}
    \AxiomC{$\alpha = \mathit{inst}(\forall\alpha.\alpha)$}
    \BinaryInfC{$\Gamma \vdash_W x{:}\ \alpha$}
\end{prooftree}
Por exemplo, esteja a variável $a$ com tipo $\forall\alpha.\alpha$ no contexto, ou seja, $a{:}\ \forall\alpha.\alpha\ \in\ \Gamma$.
O algoritmo então inferirá, que a variável $a$ terá tipo $\alpha$, após as devidas normalizações (redução-$\alpha$).

Os comandos serão inferidos a partir de substituições, onde o algoritmo as retornará representando o absurdo para qual esses testemunham.
Para o $\mathtt{[Jump]}$, partindo das premissas onde sob um contexto $\Gamma$, a chamada $k$ terá um tipo monomórfico $\tau_1$, os $n$ argumentos em $\vv{a}$ terão $n$ tipos monomórficos $\tau_2$, e ainda, $S$ é a unificação mais geral entre $\tau_1$ e $\neg\vv{\tau_2}$, o algoritmo irá então retornar esta substituição $S$ para o salto $k(\vv{a})$.

\begin{prooftree}
    \AxiomC{$\Gamma \vdash_W k{:}\ \alpha$}
    \AxiomC{$\Gamma \vdash_W x{:}\ \beta$}
    \AxiomC{$S = \mathit{mgu}(\alpha, \neg\beta)$}
    \TrinaryInfC{$\Gamma \vdash_W k(x) \Rightarrow \{\ \alpha \mapsto \neg\beta\ \}$}
\end{prooftree}
Por exemplo, em determinado contexto $\Gamma$, seja a chamada $k$ com tipo $\alpha$, ou seja, $\Gamma \vdash_W k{:}\ \alpha$, e ainda sob o mesmo contexto, o argumento $x$ com tipo $\beta$, ou seja, $\Gamma \vdash_W x{:}\ \beta$.
A partir da unificação mais geral entre $\alpha$ e $\neg\beta$ é obtida a substituição $S$, ou seja, $S = \mathit{mgu}(\alpha, \neg\beta)$.
O algoritmo então, irá inferir que a substituição para que o salto represente uma contradição é $\{\alpha \mapsto \neg\beta\}$.

Vale destacar que o algoritmo de unificação apresentado na Figura~\ref{algo:unify}, ainda que retorne a substituição que representa a unificação mais geral, não é o mesmo que o $\mathit{mgu}$ utilizado na regra $[\mathtt{Jump}]$.
Suas definições variam conforme os sistemas de tipos que elas atendem.
Enquanto que a função \Unify\ é definida para os tipos do Sistema Damas-Milner, a \Mgu, apresentada na Figura~\ref{algo:mgu-cps} é definida para os tipos do Sistema baseado em continuações.

\begin{figure}[ht!]
  \caption{Algoritmo de unificação para o Sistema baseado em continuações no formato de função.}
  \centering
  \begin{align*}
    & \texttt{\Mgu($\alpha,\Whitespace\tau$) = } \\
    & \qquad{} \texttt{\Return\Whitespace\UnifyVar($\alpha,\Whitespace\tau$)} \\
    & \texttt{\Mgu($\tau,\Whitespace\alpha$) = } \\
    & \qquad{} \texttt{\Return\Whitespace\UnifyVar($\alpha,\Whitespace\tau$)} \\
    & \texttt{\Mgu(\texttt{Int},\Whitespace\texttt{Int}) = } \\
    & \qquad{} \texttt{\Return\Whitespace[$\Whitespace$]} \\
    & \texttt{\Mgu(\texttt{Neg}\Whitespace ${list}_{1}$,\Whitespace\texttt{Neg} ${list}_{2}$) = } \\
    & \qquad{} \texttt{\If \Whitespace \Length\Whitespace${list}_{1}$ $\neq$ \Length\Whitespace${list}_{2}$, \Then \Whitespace \Fail} \\
    & \qquad{} \texttt{\Else\Whitespace\Return\Whitespace\MguList(${list}_{1},\Whitespace{list}_{2}$)} \\
    & \texttt{\Mgu($\tau_1,\Whitespace\tau_2$) = } \\
    & \qquad{} \texttt{\Fail} \\
    \\
    & \texttt{\MguList($[\Whitespace],\Whitespace[\Whitespace]$) = } \\
    & \qquad{} \texttt{\Return\Whitespace[$\Whitespace$]} \\
    & \texttt{\MguList($[\tau\HeadSep\Whitespace\tau_s],\Whitespace[\tau'\HeadSep\Whitespace\tau_s']$) = } \\
    & \qquad{} \texttt{$S_1$\Whitespace:=\Whitespace\Mgu($\tau,\Whitespace\tau'$)} \\
    & \qquad{} \texttt{$S_2$\Whitespace:=\Whitespace\MguList($S_1(\tau_s),\Whitespace S_1(\tau_s')$)} \\
    & \qquad{} \texttt{\Return\Whitespace$S_2\Whitespace \circ\Whitespace S_1$} \\
    & \texttt{\MguList($\_,\Whitespace\_$) = } \\
    & \qquad{} \texttt{\Fail}
  \end{align*}
  \small{Fonte: autor.}
  \label{algo:mgu-cps}
\end{figure}

A principal diferença deste algoritmo em relação ao anterior é que, neste sistema de tipos, há listas de tipos que também precisam ser unificadas.
Para realizar essa unificação, verifica-se inicialmente se as duas listas possuem o mesmo tamanho, ou seja, se $\texttt{\Length}\ list_1 = \texttt{\Length}\ list_2$.
Caso essa condição seja satisfeita, o algoritmo \Mgu\ é aplicado recursivamente a cada par de tipos correspondentes nos mesmos índices das listas, acumulando as substituições parciais ao longo do processo.
A função \texttt{\UnifyVar} é responsável por realizar a unificação entre variáveis de tipo e outros tipos, retornando a substituição correspondente ou um erro, caso a verificação de ocorrência via \texttt{\Occurs} falhe.

Para a regra que garante o polimorfismo do sistema, o $[\mathtt{Bind}]$, a chamada $k$ recebe $\vv{x}$ argumentos, onde estes terão $\vv{\tau}$ tipos como sendo variáveis de tipo, ou seja, $\vv{\tau} = \vv{\mathit{newvar}}$.
O $c$, por se tratar de um comando, será uma substituição $S_1$, onde recursivamente será inferida com o contexto inicial $\Gamma$ unido com os $\vv{x}$ argumentos tipados com suas $\vv{\tau}$ variáveis de tipo frescas, ou seja, $\{\ \Gamma\ \cup\ \{\ \vv{x}{:}\ \vv{\tau}\ \}\ \} \vdash_W c \Rightarrow S_1$.
Um ponto de atenção é necessário na função de generalização $\sigma = \overline{S_1\Gamma}(S_1 \neg\vv{\tau})$.
A subsituição $S_1$ aplicada no contexto garante que este esteja atualizado com a descoberta do comando $c$ na premissa anterior.
Como as continuações não retornam e sim somente passam o resultado da computação adiante, é necessário também que $S_1$ seja aplicado no tipo do argumento $S_1 \neg\vv{\tau}$, para garantir que a substituição obtida no comando anterior seja utilizada nos tipos.
De maneira semelhante ao primeiro comando, o comando $b$ é inferido recursivamente com $S_1$ aplicado no contexto unido com o salto $k$ tendo o tipo polimórfico $\sigma$ produzido na premissa anterior, sendo atribuido a esta inferência a substituição $S_2$, ou seja, $\{\ S_1\Gamma\ \cup\ \{\ k{:}\ \sigma \ \} \ \} \Rightarrow S_2$.
O algoritmo portanto, para o comando $\mathtt{let}\ k(\vv{x}) = c\ \mathtt{in}\ b$, irá produzir a substituição resultante da composição entre as substituições de $b$ e $c$, ou seja, $S_2 \circ S_2$.

\section{Implementação}\label{sec:implementacao}
\chapter{Resultados}\label{ch:resultados}
Uma vez que todo o fluxo do programa foi exibido, seus resultados podem ser apresentados e compreendidos.
Diversos testes foram executados onde, a partir da função de tradução de tipos, puderam ter seus tipos verificados para validar a implementação.
Para tal, estão disponibilizados no repositório do projeto, arquivos de entrada com funções contendo diferentes características.
Afim de observar o comportamento do programa abrangendo uma maior gama de opções, algumas das funções testadas são extensas e contam com combinações de regras do cálculo-$\lambda$ simplesmente tipado.

\section{Combinador S}
Um desses é o combinador S ($\lambda x.\lambda y.\lambda z.\ x\ z\ (y\ z)$), que apesar de não ser um termo extenso, este utiliza de uma combinação dos três construtores (variáveis, abstrações e aplicações) para que o termo seja obtido.
O tipo deste, é representado por $(\alpha \to \beta \to \gamma) \to (\alpha \to \beta) \to \alpha \to \gamma$ e suas traduções, tanto em CBN quanto em CBV são grandes demais para serem aqui colocadas, mas estas podem ser encontradas no código fonte do repositório mencionado.
A tradução do seu tipo entretanto, para CBN, é apresentada a seguir.
\lstset{extendedchars=false, escapeinside=''}
\begin{lstlisting}[style=output,caption={Tradução em CBN do tipo do combinador S}]
  '$\forall\alpha,\beta,\gamma.\ \neg\neg(\neg\neg\neg(\neg\neg\alpha,\ \neg\neg(\neg\neg\beta,\ \neg\gamma)),\ \neg\neg(\neg\neg\neg(\neg\neg\alpha,\ \neg\beta),\ \neg\neg(\neg\neg\alpha,\ \neg\gamma)))$'
\end{lstlisting}
Enquanto que, a partir do tipo inferido a seguir para a mesma estratégia de avaliação, uma substituição $S$ tal que ao aplicá-la no tipo traduzido torne-se o inferido, é $S = \{\ \alpha \mapsto \neg\alpha,\ \beta \mapsto \neg\beta,\ \gamma \mapsto \neg\gamma\ \}$, validando assim a inferência para este termo.
\lstset{extendedchars=false, escapeinside=''}
\begin{lstlisting}[style=output,caption={Inferência do tipo do combinador S traduzido em CBN}]
  '$\forall\alpha,\beta,\gamma.\ \neg\neg(\neg\neg\neg(\neg\alpha,\ \neg\neg(\neg\beta,\ \gamma)),\ \neg\neg(\neg\neg\neg(\neg\alpha,\ \beta),\ \neg\neg(\neg\alpha,\ \gamma)))$'
\end{lstlisting}
Um comportamento semelhante pode ser percebido para a tradução por \textit{call-by-value}, onde respectivamente é apresentado a seguir a tradução e o resultado da inferência.
\lstset{extendedchars=false, escapeinside=''}
\begin{lstlisting}[style=output,caption={Tradução em CBV do tipo do combinador S}]
  '$\forall\alpha,\beta,\gamma.\ \neg\neg(\neg(\alpha,\ \neg\neg(\beta,\ \neg\gamma)),\ \neg\neg(\neg(\alpha,\ \neg\beta),\ \neg\neg(\alpha,\ \neg\gamma)))$'
\end{lstlisting}
Neste caso, a substituição $S$ que satisfaz a condição de subtipagem é tal que $S = \{\ \gamma \mapsto \neg\gamma\ \}$, tornando válida assim a inferência para este termo.
\lstset{extendedchars=false, escapeinside=''}
\begin{lstlisting}[style=output,caption={Inferência do tipo do combinador S traduzido em CBV}]
  '$\forall\alpha,\beta,\gamma.\ \neg\neg(\neg(\alpha,\ \neg\neg(\beta,\ \gamma)),\ \neg\neg(\neg(\alpha,\ \neg\beta),\ \neg\neg(\alpha,\ \gamma)))$'
\end{lstlisting}

\section{Soma}
O próximo exemplo apresentado é a função de soma de 2 e 3 feita com os numerais de Church ($(\lambda n.\ \lambda m.\ \lambda f.\ \lambda x.\ n\ f\ (m\ f\ x))\ (\lambda a.\ \lambda b.\ a\ (a\ b))\ (\lambda c.\ \lambda d.\ c\ (c\ (c\ d)))$).
Seu tipo, é representado por $((\alpha \to \alpha) \to \alpha \to \alpha)$.
A tradução do tipo para CBN é dado por:
\lstset{extendedchars=false, escapeinside=''}
\begin{lstlisting}[style=output,caption={Tradução em CBN do tipo da função de soma}]
  '$\forall\alpha.\ \neg\neg(\neg\neg\neg(\neg\neg\alpha,\ \neg\alpha),\ \neg\neg(\neg\neg\alpha,\ \neg\alpha))$'
\end{lstlisting}
E o inferido também para CBN, onde a substituição $S$ que satisfaz a subtipagem é tal que $S = \{\ \alpha \mapsto \neg\alpha\ \}$:
\lstset{extendedchars=false, escapeinside=''}
\begin{lstlisting}[style=output,caption={Inferência do tipo da função de soma traduzido em CBN}]
  '$\forall\alpha.\ \neg\neg(\neg\neg\neg(\neg\alpha,\ \alpha),\ \neg\neg(\neg\alpha,\ \alpha))$'
\end{lstlisting}
Enquanto que para CBV, o tipo traduzido é:
\lstset{extendedchars=false, escapeinside=''}
\begin{lstlisting}[style=output,caption={Tradução em CBV do tipo da função de soma}]
  '$\forall\alpha.\ \neg\neg(\neg(\alpha,\ \neg\alpha),\ \neg\neg(\alpha,\ \neg\alpha))$'
\end{lstlisting}
O inferido portanto, sendo que a substituição $S$ que satisfaz a subtipagem neste caso é a substituição trivial $S = \{\ \alpha \mapsto \alpha\ \}$:
\lstset{extendedchars=false, escapeinside=''}
\begin{lstlisting}[style=output,caption={Inferência do tipo da função de soma traduzido em CBV}]
  '$\forall\alpha.\ \neg\neg(\neg(\alpha,\ \neg\alpha),\ \neg\neg(\alpha,\ \neg\alpha))$'
\end{lstlisting}
Por fim, ao executar o código gerado, obtém-se o resultado $(5,\ 5)$, indicando uma correta tradução da expressão de entrada.

\section{Multiplicação}
Outro exemplo é o da multiplicação de 6 e 8, que o termo lambda por si é grande demais para ser exibido aqui.
A função lambda responsável pela multiplicação de dois argumentos, ($\lambda m.\ \lambda n.\ \lambda f.\ \lambda x.\ m\ (n\ f)\ x$) é extenso o bastante para seu sucesso dar uma noção boa de que o código está correto e que o algoritmo é capaz de inferir o tipo corretamente das mais diversas expressões.
Assim como o exemplo da função de soma, o tipo desta expressão é: $((\alpha \to \alpha) \to \alpha \to \alpha)$.
A tradução do tipo para CBN é dado por:
\lstset{extendedchars=false, escapeinside=''}
\begin{lstlisting}[style=output,caption={Tradução em CBN do tipo da função de soma}]
  '$\forall\alpha.\ \neg\neg(\neg\neg\neg(\neg\neg\alpha,\ \neg\alpha),\ \neg\neg(\neg\neg\alpha,\ \neg\alpha))$'
\end{lstlisting}
E o inferido também para CBN, onde a substituição $S$ que satisfaz a subtipagem é tal que $S = \{\ \alpha \mapsto \neg\alpha\ \}$:
\lstset{extendedchars=false, escapeinside=''}
\begin{lstlisting}[style=output,caption={Inferência do tipo da função de soma traduzido em CBN}]
  '$\forall\alpha.\ \neg\neg(\neg\neg\neg(\neg\alpha,\ \alpha),\ \neg\neg(\neg\alpha,\ \alpha))$'
\end{lstlisting}
Enquanto que para CBV, o tipo traduzido é:
\lstset{extendedchars=false, escapeinside=''}
\begin{lstlisting}[style=output,caption={Tradução em CBV do tipo da função de soma}]
  '$\forall\alpha.\ \neg\neg(\neg(\alpha,\ \neg\alpha),\ \neg\neg(\alpha,\ \neg\alpha))$'
\end{lstlisting}
O inferido portanto, sendo que a substituição $S$ que satisfaz a subtipagem neste caso é a substituição trivial $S = \{\ \alpha \mapsto \alpha\ \}$:
\lstset{extendedchars=false, escapeinside=''}
\begin{lstlisting}[style=output,caption={Inferência do tipo da função de soma traduzido em CBV}]
  '$\forall\alpha.\ \neg\neg(\neg(\alpha,\ \neg\alpha),\ \neg\neg(\alpha,\ \neg\alpha))$'
\end{lstlisting}
É possível perceber que, os resultados obtidos são os mesmos da função de soma.
Isso se deve ao fato de que, por mais que a expressão seja diferente, as duas possuem o mesmo tipo.
Então, o algoritmo infere que as duas possuem o mesmo tipo.

Por fim, ao executar o código gerado, obtém-se o resultado $(48,\ 48)$, indicando uma correta tradução da expressão de entrada.
\section{Identidade com `Let'}
Ainda que a tradução do `let' tenha sido apresentada e discutida, esta possui um problema relacionada à tipagem quanto ao \textit{call-by-value}.
Para melhor compreender esta questão, é necessário entender fundamentalmente a diferença entre as estratégias de avaliação por nome e por valor.
A por nome, também conhecida como avaliação preguiçosa, só irá avaliar a expressão no momento em que esta for necessária, desta forma, caso tenha alguma expressão que não seja utilizada, esta nem mesmo será computada.
Enquanto que a por valor não, ao invés disso, ela avalia toda expressão no início, sem se importar se será utilizada ou não, desta forma, ainda que uma expressão não seja utilizada, ela será calculada.

A função em questão é a identidade utilizando o `let', ($\mathtt{let\ id\ =\ \lambda x.\ x\ in\ id\ id}$).
Seu tipo é $\alpha \to \alpha$, enquanto que sua tradução em CBN é dada por:
\lstinputlisting[style=haskell, label=cps:let-id-cps-cbn, caption={Tradução em CBN da identidade com `let'}]{Code/Type-Inferer/CPS/let-id-cbn.cps}
A tradução do tipo para CBN é dada por:
\lstset{extendedchars=false, escapeinside=''}
\begin{lstlisting}[style=output,caption={Tradução em CBN da identidade com `let'}]
  '$\forall\alpha.\ \neg\neg(\neg\neg\alpha,\ \neg\alpha)$'
\end{lstlisting}
E o inferido também para CBN, onde a substituição $S$ que satisfaz a subtipagem é tal que $S = \{\ \alpha \mapsto \neg\alpha \ \}$:
\lstset{extendedchars=false, escapeinside=''}
\begin{lstlisting}[style=output,caption={Inferência da identidade com `let' traduzido em CBN}]
  '$\forall\alpha.\ \neg\neg(\neg\alpha,\ \alpha)$'
\end{lstlisting}
Já ao olhar para o resultado em CBV, temos um problema de tipagem.
A tradução da expressão, que num primeiro momento não há nenhum problema aparente, é dada por:
\lstinputlisting[style=haskell, label=cps:let-id-cps-cbv, caption={Tradução em CBV da identidade com `let'}]{Code/Type-Inferer/CPS/let-id-cbv.cps}
Ao investigar mais a fundo porém, é possível ser notado na linha 1 que, como o $\mathtt{id}$ é passado por parâmetro para a continuação $\mathtt{k}$, durante o processo de inferência, esta será inserida no contexto como sendo um tipo polimórfico.
Em momento posterior entretanto, na inferência da expressão $\mathtt{id\ id}$, a expressão $\mathtt{id}$ é assumida como sendo um tipo monomórfico.
É neste instante então que é feita a verificação do \textit{occurs check} para garantir que tipos cíclicos (ou seja, um tipo estar contido em outro, tornando assim impossível a unificação destes) não sejam permitidos, onde esta falha, retornando assim o erro.
Desta forma, o algoritmo falha, identificando um erro de \textit{OccursCheck}, que será apresentado a seguir:
\lstset{extendedchars=false, escapeinside=''}
\begin{lstlisting}[style=output,caption={Erro de Inferência da identidade com `let' traduzido em CBV}]
  '$OccursCheck{:}\ \beta\ in\ \neg(\beta,\ \alpha)$'
\end{lstlisting}
Este erro indica que, ao tentar encontrar uma unificação para as variáveis de tipo, foi encontrado que uma delas já estava presente na outra, neste caso, que o $\beta$ já ocorria em $\neg(\beta,\ \alpha)$.
Como isto torna impossível que uma substituição seja suficiente, de modo a gerar um loop infinito caso seja tentado, um erro é retornado.

\section{Autoaplicação Mal-Tipada}\label{sec:autoap-mal-tipada}
Este exemplo demonstra um caso clássico de termo não tipável no cálculo lambda simplesmente tipado: a autoaplicação ($\lambda x.\ x\ x$).
No sistema de tipos original, a expressão requer que o tipo de $x$ seja ao mesmo tempo uma função (para ser aplicada) e o argumento dessa função, levando a uma contradição.
O erro de tipo gerado é:

\lstset{extendedchars=false, escapeinside=''}
\begin{lstlisting}[style=output,caption={Erro de tipo no cálculo lambda original}]
  Cannot unify { '$\alpha$' } with { '$\alpha \to \beta$' }
\end{lstlisting}
A tradução para CPS em \textit{call-by-name} é dada por:
\lstinputlisting[style=haskell, caption={Tradução em CBN da autoaplicação}]{Code/Type-Inferer/CPS/ill-typed-cbn.cps}
Durante a inferência de tipos no sistema CPS, o algoritmo falha com:
\begin{lstlisting}[style=output,caption={Erro de inferência em CBN}]
  OccursCheck: '$\epsilon$' in '$\neg\neg(\neg\epsilon,\ \gamma)$'
\end{lstlisting}
Já na estratégia \textit{call-by-value}, a tradução é:
\lstinputlisting[style=haskell, caption={Tradução em CBV da autoaplicação}]{Code/Type-Inferer/CPS/ill-typed-cbv.cps}
E o erro de inferência correspondente:
\begin{lstlisting}[style=output,caption={Erro de inferência em CBV}]
  OccursCheck: '$\beta$' in '$\neg(\beta,\ \gamma)$'
\end{lstlisting}
Em ambos os casos, o sistema CPS apresenta erro na tipagem, refletindo a não tipagem do termo original.

\section{Combinador Y}\label{sec:y-combinator}
O combinador Y ($\lambda f.\ (\lambda x.\ f\ (x\ x))\ (\lambda x.\ f\ (x\ x))$) é outro termo não tipável no cálculo lambda simplesmente tipado, essencial para expressar recursão em cálculos não tipados.
O erro de unificação é semelhante ao caso anterior:

\lstset{extendedchars=false, escapeinside=''}
\begin{lstlisting}[style=output,caption={Erro de tipo no cálculo lambda original}]
  Cannot unify { '$\beta$' } with { '$\beta \to \gamma$' }
\end{lstlisting}
A tradução em \textit{call-by-name} é:
\lstinputlisting[style=haskell, caption={Tradução em CBN do combinador Y}]{Code/Type-Inferer/CPS/y-combinator-cbn.cps}
A inferência de tipos no sistema CPS para CBN gera:
\begin{lstlisting}[style=output,caption={Erro de inferência em CBN}]
  OccursCheck: '$\lambda$' in '$\neg\neg(\neg\lambda,\ \iota)$'
\end{lstlisting}
Para \textit{call-by-value}, a tradução é a seguinte:
\lstinputlisting[style=haskell, caption={Tradução em CBV do combinador Y}]{Code/Type-Inferer/CPS/y-combinator-cbv.cps}
Com o seguinte erro de inferência:
\begin{lstlisting}[style=output,caption={Erro de inferência em CBV}]
  OccursCheck: '$\zeta$' in '$\neg(\zeta,\ \neg\iota)$'
\end{lstlisting}
Estes resultados demonstram que o sistema de tipos proposto é consistente com o cálculo lambda tradicional, identificando corretamente termos não tipáveis através de erros de ocorrência durante a unificação.

\begin{frame}{Conclusões Parciais}
    \begin{itemize}
        \item CPS é uma escolha interessante para IR
              \begin{itemize}
                  \item Otimizações
              \end{itemize}
        \item Sistema de tipos
              \begin{itemize}
                  \item Correção de transformações e otimizações
                  \item Ausência de certos comportamentos indesejados
              \end{itemize}
    \end{itemize}
\end{frame}

