\begin{frame}{Estilo de Passagem de Continuação (CPS)}
    \begin{itemize}
        \item Técnica de transformação de código que torna o fluxo de controle explícito
              \begin{itemize}
                  \item[--] Chamadas de função passam o controle para a próxima etapa explicitamente, conhecida como continuação~\cite{appel1992compiling}
                  \item[--] Ao invés das funções retornarem o resultado da computação, é invocado uma continuação, representando o próximo passo
              \end{itemize}
        \item Toda chamada de função passa então a ser uma chamada de cauda (\textit{tail-call})
    \end{itemize}
\end{frame}

\begin{frame}{Estilo de Passagem de Continuação (CPS)}
  \textbf{Chamada de cauda:}
  \begin{itemize}
    \item Última instrução executada em uma função é uma chamada a outra função, sem que restem computações adicionais a serem feitas após essa chamada~\cite{MUCHNICK1997}
          \begin{itemize}
            \item[--] Função atual pode liberar seu quadro de ativação
          \end{itemize}
  \end{itemize}
  \textbf{Chamada não de cauda:}
  \begin{itemize}
    \item Ainda restam operações, como somas ou multiplicações, após a chamada da função
          \begin{itemize}
            \item[--] Função atual precisa manter seu quadro de ativação até que as operações sejam concluídas
          \end{itemize}
  \end{itemize}
\end{frame}

\begin{frame}{Estilo de Passagem de Continuação (CPS)}
  \begin{itemize}
    \item[] \begin{figure}
            \caption{Função fatorial em Haskell com chamada não de cauda}
            \lstinputlisting[style=haskell, label=code:factorial_non_tail_call]{Code/factorial_non_tail_call.hs}
            \small{Fonte: o autor}
          \end{figure}

    \item[] \begin{figure}
            \caption{Função fatorial em Haskell com chamada de cauda}
            \lstinputlisting[style=haskell, label=code:factorial_tail_call]{Code/factorial_tail_call.hs}
            \small{Fonte: o autor}
          \end{figure}
  \end{itemize}
\end{frame}

\begin{frame}{Estilo de Passagem de Continuação (CPS)}
  \textbf{Cálculo Lambda:}\\
  \citeonline{church1932set} define o cálculo-$\lambda$, que é representado pela seguinte gramática:

  \begin{equation}
    e ::= x \mid \lambda x. e \mid e e\nonumber
  \end{equation}

  \begin{itemize}
    \item \textbf{Variável:} identificadores no sistema
    \item \textbf{Abstração:} função que associa um identificador $x$ a um termo $e$
    \item \textbf{Aplicação:} aplicação de um termo a outro
  \end{itemize}
\end{frame}

\begin{frame}{Estilo de Passagem de Continuação (CPS)}
  Variáveis no cálculo-$\lambda$ podem ser:
  \begin{itemize}
    \item \textbf{Livres:} quando não estão associadas a uma abstração de função
          \begin{itemize}
            \item[--] $\lambda x. y$
          \end{itemize}
    \item \textbf{Ligadas:} quando estão associadas a uma abstração de função
          \begin{itemize}
            \item[--] $(\lambda x. x) y$
          \end{itemize}
    \item[]
  \end{itemize}

  Para analisar expressões:
  \begin{itemize}
    \item \textbf{$\alpha$-redução:} Renomeação de variáveis ligadas.
          \begin{align}
            \lambda x . e[x] & \rightarrow \lambda y . e[y]\nonumber
          \end{align}

    \item \textbf{$\beta$-redução:} Aplicação de função.
          \begin{align}
            (\lambda x . e_1) e_2 & \rightarrow e_1 [e_2 / x]\nonumber
          \end{align}

    \item \textbf{$\eta$-redução:} Expansão de função.
          \begin{align}
            \lambda x . (e \, x) & \rightarrow e \quad \text{se } x \text{ não ocorre livre em } e\nonumber
          \end{align}
  \end{itemize}
\end{frame}

\begin{frame}{Estilo de Passagem de Continuação (CPS)}
    \textbf{Transformação CPS:}\\
    No cálculo-$\lambda$ tradicional, o fluxo de execução é implícito
    \begin{itemize}
        \item Funções são aplicadas e os resultados são retornados
        \item $\lambda x. x + 1$
    \end{itemize}
    Já no CPS, o fluxo de execução é explícito
    \begin{itemize}
        \item Uma série de chamadas de funções passam o resultado para um argumento extra, a continuação, indicando o próximo passo da computação
        \item $\lambda x. \lambda k. k (x + 1)$
    \end{itemize}
\end{frame}

\begin{frame}{Estilo de Passagem de Continuação (CPS)}
    \textbf{Cálculo de Continuações (\textit{CPS-calculus}):}\\
    \cite{thielecke1997} define como sendo um sistema formal que trata o CPS como um modelo computacional por si só\\
    Seus termos são:
    \begin{equation}
        M ::= x\langle \vec{x} \rangle \mid M\{x\langle \vec{x} \rangle = M\}\nonumber
    \end{equation}
    \begin{itemize}
        \item \textbf{Salto (\textit{jump}):} uma chamada para a continuação $x$ com os parâmetros $\vec{x}$
        \item \textbf{Vínculo (\textit{binding}):} uma chamada onde o corpo $M$ está vinculado à continuação $x$ com os parâmetros $\vec{x}$
    \end{itemize}
\end{frame}

\begin{frame}{Estilo de Passagem de Continuação (CPS)}
    \textbf{Tradução CPS:}\\
    Converte um código escrito em estilo direto para CPS~\cite{FLANAGAN1993}
    \begin{itemize}
        \item Modificar as funções para elas não retornarem um valor, mas sim, passarem o resultado para uma continuação
    \end{itemize}
\end{frame}

\begin{frame}{Estilo de Passagem de Continuação (CPS)}
    \begin{figure}
        \caption{Função soma em Haskell em Estilo Direto}
        \lstinputlisting[style=haskell, label=code:add]{Code/add.hs}
        \small{Fonte: o autor}
    \end{figure}

    \begin{figure}
        \caption{Função soma em Haskell em CPS}
        \lstinputlisting[style=haskell, label=code:add_cps]{Code/add_cps.hs}
        \small{Fonte: o autor}
    \end{figure}
\end{frame}

\begin{frame}{Estilo de Passagem de Continuação (CPS)}
    \begin{figure}
        \caption{Função fatorial em Haskell em Estilo Direto}
        \lstinputlisting[style=haskell, label=code:factorial]{Code/fat.hs}
        \small{Fonte: o autor}
    \end{figure}

    \begin{figure}
        \caption{Função fatorial em Haskell em CPS}
        \lstinputlisting[style=haskell, label=code:factorial_cps]{Code/fat_cps.hs}
        \small{Fonte: o autor}
    \end{figure}
\end{frame}

% \begin{frame}{Estilo de Passagem de Continuação (CPS)}
%     \textbf{Tranformação dos Tipos:}\\
%     Função em Estilo direto
%     \begin{itemize}
%         \item $A \rightarrow B$
%     \end{itemize}
%     Função em CPS
%     \begin{itemize}
%         \item $A \rightarrow (B \rightarrow \perp) \rightarrow \perp$
%     \end{itemize}

%     % Na função de soma, definida na Figura~\ref{code:add}, a função tem tipo $Int \rightarrow Int \rightarrow Int$, ou seja, ela recebe dois inteiros e retorna um inteiro.
%     % Já a função de soma em CPS, definido na Figura~\ref{code:add_cps}, possui o tipo $Int \rightarrow Int \rightarrow (Int \rightarrow r) \rightarrow r$. Isso significa que a função recebe dois inteiros e uma continuação, que é uma função de tipo $Int \rightarrow r$, onde \texttt{r} pode ser qualquer tipo, e retorna esse mesmo tipo \texttt{r}.

%     % Essa transformação de tipo reflete a diferença fundamental entre o estilo direto e o CPS\@: em vez de retornar um valor diretamente, a função em CPS recebe uma continuação que especifica o próximo passo da computação.
%     % O mesmo padrão pode ser observado nas funções para o cálculo do fatorial nas Figuras~\ref{code:factorial} e~\ref{code:factorial_cps}.
%     % No estilo direto, a função \texttt{factorial} tem o tipo $Int \rightarrow Int$, enquanto na versão CPS, a função \texttt{factorialCps} tem o tipo $Int \rightarrow (Int \rightarrow r) \rightarrow r$.

%     % Essa correspondência entre os tipos não é uma coincidência.
%     % Como discutido por~\cite{TORRENS2019}, uma função em estilo direto com tipo $A \rightarrow B$ pode ser transformada em uma função em CPS com o tipo $A \rightarrow (B \rightarrow \perp) \rightarrow \perp$.
%     % Aqui, $\perp$ representa o tipo dos valores que nunca retornam, uma característica associada ao estilo de passagem de continuações, onde as funções são compostas de forma a encadear continuações até que a execução termine de maneira explícita.
% \end{frame}
