\subsection{Inferência}\label{subsec:cps-inferer}
A inferência de tipos é a etapa onde, em uma linguagem onde tipos são presentes, um termo tem seu tipado inferido sem anotação prévia.
Isto é, sem explicitar o tipo de um termo, este tem seu tipo deduzido.
Em um ambiente tipado com polimorfismo, a maior utilidade do inferidor de tipos é que este seja sempre o mais geral possível, tal que possa ser especializado para cada uso.
O algoritmo de inferência proposto para este sistema de tipos é capaz de inferir o tipo mais geral possível, conforme explicado mais detalhadamente na Seção~\ref{sec:formalizacao}.
Seu desenvolvimento foi direto como pode ser visto abaixo, no Código~\ref{cps:infer-command}.

\lstinputlisting[style=haskell, label=cps:infer-command, caption={Função principal de Inferência}]{Code/Type-Inferer/CPS_infer_command.hs}
Muitas das funções necessárias para a inferência do CPS são iguais as do Damas-Milner.
Por não ser o foco deste trabalho, explicações sobre estas serão omitidos.
Desta forma, o processamento para se inferir o tipo das continuações foi dividido em três principais funções.

A função $\mathtt{inferAtom \dblcolon Context \to Id \to TI\ CPSMonoType}$, como seu nome e assinatura indica, é a função responsável por inferir os átomos do termo, isto é, buscar as variáveis do contexto e retorná-las caso sejam literais, instanciá-las se forem variáveis polimórficas, ou então retornar erro caso esta não esteja presente.
Ao executar a função passando o contexto $\{\ x{:}\ \forall\alpha.\alpha \ \}$ e a variável $x$, a função irá instanciar uma nova variável de tipo $\beta$ (se esta for a próxima ainda não utilizada) e retornar este tipo monomórfico.
Ou ainda, caso a função seja chamada com o contexto $\{\ x{:}\ \forall\alpha.\alpha \ \}$ e a variável $k$, um erro $\mathtt{UnboundVariable\ ``k"\ \{\ x{:}\ \forall\alpha.\alpha \ \}}$ será exibido.

Já a função $\mathtt{inferCommand \dblcolon Context \to Command \to TI\ Substitution}$ é quem, a partir do contexto e comando, irá retornar a substituição que indica o tipo da continuação.
Tal qual a função anterior, nenhuma dificuldade grande foi encontrada aqui.
A inferência do salto foi a mais simples delas, onde cada linha da função se refere às premissas na definição, retornando então a unificação mais geral entre $\tau_1$ e $\vv{\tau_2}$.
Já para o \textit{binding}, algumas funções auxiliares foram necessárias principalmente para manipular o contexto ao extender este em diferentes momentos.
Além destas, funções para a generalização e por fim para a composição das subsituições também foram precisas.

O ponto de partida da inferência, $\mathtt{inferWithCtx \dblcolon Command \to TI\ CPSPolyType}$, além de definir o contexto inicial com a continuação inicial $k$, recebendo um tipo $\alpha$ qualquer, esta função aplicará a substituição obtida na inferência e a aplicará no contexto, de modo que o contexto final esteja atualizado com o tipo inferido da continuacão $k$.
Após isto, é feita a normalização do tipo polimórfico da substituição, isto é, limpar as variáveis de tipo utilizadas durante o processo que não são mais necessárias, afim de promover consistência e facilitar o entendimento.
Por exemplo, no caso onde o tipo polimórfico final seja $\forall\delta.\delta$, o retornado seria $\forall\alpha.\alpha$.

Uma etapa que não está diretamente relacionada com a inferência, e sim com os tipos em si, é a verificação da subtipagem do tipo traduzido e do tipo inferido.
Isto é, uma verificação se o tipo traduzido é um subtipo do tipo inferido, indicando diretamente se o algoritmo de inferência foi implementado corretamente.
\lstinputlisting[style=haskell, label=cps:subtyping, caption={Verificação de Subtipagem}]{Code/Type-Inferer/CPS_subtyping.hs}
O algoritmo de verificação de subtipos do Código~\ref{cps:subtyping} procura uma substituição $S$ tal que, ao aplicá-la em um dos tipos, ele se torne o outro.
Por exemplo, ao analisar os tipos $\mathtt{\tau_1 = \alpha \to \alpha}$ e $\tau_2 = \neg\alpha \to \neg\alpha$, é possível encontrar uma substituição $S = \{\ \alpha \mapsto \neg\alpha \ \}$, tal que $S\tau_1$ resulte em $\tau_2$.
A partir disto, pode ser dito que o tipo $\tau_2$ é um subtipo de $\tau_1$, onde o algoritmo retornaria com sucesso a substituição $S$.
Já olhando para outro exemplo, onde $\mathtt{\tau_1 = \alpha \to \neg\alpha}$ e $\mathtt{\tau_2 = \alpha \to \alpha}$, não é possível encontrar uma substituição $S$ tal que $S\tau_1$ seja $\tau_2$.
Desta forma, o algoritmo não retornaria uma substituição, falhando assim a verificação de subtipos.
