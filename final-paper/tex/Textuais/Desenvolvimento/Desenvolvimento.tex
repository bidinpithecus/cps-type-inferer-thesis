\chapter{Desenvolvimento}\label{ch:desenvolvimento}

Será abordado, neste capítulo, a contribuição prática do trabalho, apresentando formalização do sistema de tipos proposto, sua implementação na linguagem Haskell, e os resultados experimentais obtidos.
No que diz respeito ao sistema de tipos, em virtude da limitação de tempo e da grandeza deste trabalho, não foi feito a prova de que este é correto ou completo, há somente indícios que levam a crer que o trabalho feito está correto, junto de resultados corretos em testes efetuados.
Uma explicação mais detalhada sobre esta parte será dada na Seção~\ref{sec:formalizacao}.

Na Seção~\ref{sec:implementacao}, será descrito as decisões de projeto que orientaram a implementação. 
O ambiente de execução utilizado foi o compilador \textit{The Glorious Glascow Haskell Compilation System} (GHC), na versão 9.12.2 em conjunto com o gerenciador de projetos Cabal na versão 3.14.1.
O código-fonte completo está disponível publicamente no repositório \textit{cps-type-inferer}\footnote{\url{https://github.com/bidinpithecus/cps-type-inferer}} do GitHub.
Na Seção~\ref{sec:implementacao}, terá um aprofundamento maior na implementação.

Considerações, como resultados e descobertas importantes, serão entradas em detalhes na Seção~\ref{sec:resultados}, onde será discutido sobre a relevância das contribuições e próximos passos a serem tomados em relação a esta pesquisa na seção~\ref{sec:conclusao}.

\section{Formalização}\label{sec:formalizacao}
\section{Implementação}\label{sec:implementacao}

Partindo para a parte prática do trabalho, os módulos e funções serão apresentados de maneira gradual, de modo a facilitar o entendimento do fluxo inteiro do programa.
Vale destacar aqui que é feito também a implementação do sistema Hindley-Milner, porém, como este foi feito somente para poder usar o sistema de continuações de maneira mais cômoda, não será entrado em detalhes nos que dizem respeito à esse sistema de tipos.
Inicialmente, na Subseção~\ref{subsec:cps-adt}, será discutido sobre a maneira como foi representado o sistema de tipos.
Em sequência, a Subseção~\ref{subsec:cps-translations} trará detalhes sobre a implementação das funções de tradução de cálculo lambda simplesmente tipado para CPS.
Posteriormente, a Subseção~\ref{subsec:cps-inferer} irá tratar da inferência em si, juntamente da verificação do tipo.
Por fim, a geração de código abordada na Subseção~\ref{subsec:cps-code-gen} serve como uma maneira de se testar a tradução do código.

\subsection{Tipos de Dados}\label{subsec:cps-adt}

Para representar os comandos, bem como os tipos do sistema, foram utilizados tipos de dados algébricos (ADTs, do inglês \textit{algebraic data types}), disponíveis no Listing~\ref{code:cps-adt}.

\lstinputlisting[style=haskell, label=code:cps-adt, caption={Definição dos tipos de dados}]{Code/Type-Inferer/CPSTyping.hs}
Para os comandos, dois construtores podem ser observados, o \texttt{Jump} e o \texttt{Bind}, sendo responsáveis por construir respectivamente os comandos de \textit{Jump}, onde há um salto \texttt{Id} com \texttt{[Id]} parâmetros.
Sendo assim, o salto $k(x)$ seria representado por este ADT: $\mathtt{Jump\ k\ [x]}$.
E o comando \textit{Bind}, onde há outro \texttt{Command} definido recursivamente, a função \texttt{Id} com argumentos \texttt{[Id]} e por fim outro \texttt{Command} também definido recursivamente.
O \textit{bind} $\mathtt{let}\ k(x) = k(x)\ \mathtt{in}\ k(x)$, portanto seria definido pelo seguinte ADT: $\mathtt{Bind\ (Jump\ k\ [x])\ k\ [x]\ (Jump\ k\ [x])}$.

Os tipos, foram representados com dois tipos algébricos diferentes, um para os monotipos $\mathtt{CPSMonoType}$, e uma para os politipos $\mathtt{CPSPolyType}$.
As variáveis do monotipo são construídas a partir dos construtores $\mathtt{TVar\ Id}$ e $\mathtt{TInt}$, onde no primeiro, o $\mathtt{Id}$ serve para obter a variável de tipo atribuída àquela variável, enquanto que as funções que não retornam são representadas as partir do construtor de negação $\mathtt{TNeg\ [CPSMonoType]}$ sendo os argumentos dela definidos recursivamente sobre si.
O contexto por sua vez, é um tipo que utiliza o $\mathtt{Data.Map}$ disponível no pacote \textit{containers} para mapear uma variável para um tipo polimórfico.
As substituições são representadas utilizando o mesmo $\mathtt{Data.Map}$, onde desta vez é mapeado uma variável de tipo para um monotipo.

\subsection{Traduções}\label{subsec:cps-translations}
Para facilitar os testes efetuados e ainda tornar o fluxo de execução mais direto, foram implementas as funções de tradução de expressões e de tipos de acordo com as definições das Seções~\ref{subsec:cps-translation} e~\ref{subsec:typed-cps-translation}.

\lstinputlisting[style=haskell, label=cps:cbn-initial-cont, caption={Continuação inicial}]{Code/Type-Inferer/CPS_initial_cont.hs}
Para todas as computações, um contexto inicial precisa conter a continuação inicial.
Este será o objeto a ter seu tipo inferido.
Afim de praticidade, esta continuação será sempre a mesma, dada por `k', como mostrado no Código~\ref{cps:cbn-initial-cont}.

\lstinputlisting[style=haskell, label=cps:cbn-expr-translation, caption={Tradução das expressões para CBN}]{Code/Type-Inferer/CPS_CBN_expr_translation.hs}
No Código~\ref{cps:cbn-expr-translation}, são apresentadas as funções responsáveis para traduzir as expressões para CBN.
O ponto de partida desta computação será a função $\mathtt{cbnExprTrans \dblcolon Expr \to Command}$, onde ela irá receber a expressão em cálculo-$\lambda$ e retornará o comando traduzido.
Sua responsabilidade é criar as mônadas de estado que farão o controle do índice das variáveis frescas necessárias e chamar as funções de tradução passando a continuação inicial.

Tomando como exemplo a função identidade em cálculo-$\lambda$ ($\lambda x.\ x$), é possível perceber como os termos crescem em CPS.
Isto torna o desenvolvimento diretamente neste cálculo não apropriado, mas ainda, nota-se que a implementação da função de tradução é bastante direta em relação a sua definição formal.
Os únicos pontos de atenção são em relação à geração das variáveis frescas, mas que como foi dito anteriormente, não eram completamente necessários, visto que eles são ligados imediatamente.
Ao traduzir então a função, tem-se que o equivalente em CPS é o mostrado no Código~\ref{cps:id-cps-cbn}.
\lstinputlisting[style=haskell, label=cps:id-cps-cbn, caption={Tradução da função identidade em CBN}]{Code/Type-Inferer/CPS/id-cbn.cps}
Este comportamento fica ainda mais visível quando uma função um pouco maior é traduzida, por exemplo o numeral de Church dois ($\lambda f.\ \lambda x.\ f\ (f\ x)$).
Sua tradução portanto é mostrada no Código~\ref{cps:church-two-cps-cbn}.
\lstinputlisting[style=haskell, label=cps:church-two-cps-cbn, caption={Tradução do numeral de Church ``2'' em CBN}]{Code/Type-Inferer/CPS/church-two-cbn.cps}
De maneira semelhante, foram feitas as mesmas funções utilizadas no \textit{call-by-name}, porém adaptadas para o CBV, respeitando as diferenças presentes na definicão formal da função.
\lstinputlisting[style=haskell, label=cps:cbv-expr-translation, caption={Tradução das expressões para CBV}]{Code/Type-Inferer/CPS_CBV_expr_translation.hs}
A partir dessas diferenças nas definições, nota-se também particularidades nas traduções destas, por exemplo ao traduzir a mesma função identidade, em CBV, é obtido o resultado apresentado no Código~\ref{cps:id-cps-cbv}.
\lstinputlisting[style=haskell, label=cps:id-cps-cbv, caption={Tradução da função identidade em CBV}]{Code/Type-Inferer/CPS/id-cbv.cps}
Neste exemplo da função identidade, pouca diferença entre as duas estratégias de avaliação pode ser notada.
Isso se deve não ao tamanho da expressão, e sim dos elementos desta.
Neste caso, há somente uma abstração lambda com uma variável.
A seguir, é exibido novamente o numeral de Church dois, porém para o CBV, onde mais diferenças podem ser observadas.
O motivo disto é os elementos da função, que diferentemente da identidade, conta com mais construtores para representá-la, exibido no Código~\ref{cps:church-two-cps-cbv}.
\lstinputlisting[style=haskell, label=cps:church-two-cps-cbv, caption={Tradução do numeral de Church ``2'' em CBV}]{Code/Type-Inferer/CPS/church-two-cbv.cps}

Mostrado anteriormente no Código~\ref{cps:id-cps-cbn}, a expressão CPS resultante da tradução da função identidade em CBN difere da mesma traduzida em CBV, presente no Código~\ref{cps:id-cps-cbv}.
O mesmo pode ser observado ao traduzir a função identidade tipada.
Por exemplo, ao executar a função para a identidade em CBN, o tipo obtido é $\neg\neg(\neg\neg\alpha,\ \neg\alpha)$.
Já em CBV, para a mesma função, tem-se que o tipo traduzido é $(\alpha,\ \neg\alpha)$.

Para raciocinar sobre os tipos do sistema, tem que ser levado em consideração o que estes representam, contradições.
Ao tomar como exemplo o tipo resultante da função identidade em CPS a partir da tradução por valor (CBV), isto é, $(\alpha,\ \neg\alpha)$\footnote{Apesar de estar sendo traduzido o tipo da função identidade, como esta tradução é feita para os tipos simples, note que aqui, está sendo representado monomorficamente pois não há a generalização do $\alpha$.}, deve-se pensar que este representa o absurdo de ter $\alpha$ como argumento e $\neg\alpha$ como continuação, ao mesmo tempo.

\lstinputlisting[style=haskell, label=cps:cbn-type-translation, caption={Tradução dos tipos para CBN}]{Code/Type-Inferer/CPS_CBN_type_translation.hs}
Aqui no Código~\ref{cps:cbn-type-translation}, a função responsável por traduzir um tipo utilizando a estratégia \textit{call-by-name}, a função $\mathtt{cbvTypeTranslation \dblcolon LambdaMonoType \to CPSPolyType}$ irá receber um tipo simples no cálculo-$\lambda$ simplesmente tipado, e retornar um tipo polimórfico em CPS.
Perceba que na definição, o retorno era um tipo simples em CPS. 
Essa diferença é justificada ao se observar o corpo desta função, onde há a definição da função $\mathtt{cbvTrans}$.
Esta é quem efetivamente faz a computação, pois é nela que acontece o casamento de padrões para determinar o tipo sendo traduzido.
Ainda, esta função retorna um tipo polimórfico pelo fato de que em momento posterior a essa tradução, a subtipagem do tipo traduzido e do tipo inferido precisa ser verificada.

\lstinputlisting[style=haskell, label=cps:cbv-type-translation, caption={Tradução dos tipos para CBV}]{Code/Type-Inferer/CPS_CBV_type_translation.hs}
Para que essas duas funções de tradução de tipos (CBN e CBV) sejam usadas do mesmo modo, elas seguem a messma assinatura. 
Seus comportamentos são o mesmo, a diferir somente nas diferenças das definições da função, isto é, como é feita a tradução.

O sistema de tipos proposto neste trabalho, entretanto, é polimórfico, ou seja, aqui há a adição de variáveis de tipos quantificadas.
Sendo assim, esta função de tradução, se aplicando neste caso de uso, não necessariamente retornará o tipo mais geral de uma expressão, mas sempre um subtipo deste.

\subsection{Inferência}\label{subsec:cps-inferer}
% Explicar inferência
% \lstinputlisting[style=haskell, label=cps:typing, caption={Definição dos tipos de dados}]{Code/Type-Inferer/CPSTyping.hs}
% Explicar código

\subsection{Geração de Código}\label{subsec:cps-code-gen}
Uma vez que as provas de completude e consistência não estavam no escopo deste trabalho e a validação do algoritmo se deu por meio de testes, um conjunto considerável de testes foi construído.
Para isolar os testes, ou seja, testar as funções de maneira independente para garantir que, se a inferência apresentasse algum erro, fosse certo que o erro estaria na inferência e não por conta de uma tradução incorreta, dois teoremas apresentados em~\cite{plotkin1975call} foram utilizados.

Este teorema, chamado de Teorema da Simulação, válido tanto para \textit{call-by-name} quanto para \textit{call-by-value}, afirma que, dado um programa $M$ em cálculo-$\lambda$, o resultado de sua computação há de ser o mesmo que o resultado da computação da tradução para CPS com a função identidade.
Ou seja, simulando a execução do programa feito em cálculo-$\lambda$ no cálculo de continuações.
Ou então, formalmente, $Eval(M) = Eval([M] (\lambda x.\ x))$, onde a função $Eval$ é responsável por avaliar uma expressão, efetivamente a computando.

Foi implementada a simulação do cálculo-$\lambda$ no cálculo de continuações a partir das funções de tradução para programas que computam os numerais de Church.
Isto é feito computando a função $\lambda$ onde, é incrementado o valor que representa o número de aplicações feitas (essencialmente como um numeral de Church é computado), passando como continuação a função identidade.
Fazendo com que assim, a função lambda que representa um numeral de church é simulada pelo cálculo de continuações.

\lstinputlisting[style=haskell, label=cps:code-gen, caption={Geração de código para computação de numerais de Church}]{Code/Type-Inferer/CPS_code_gen.hs}
O código Haskell gerado pode ser dividido em três partes principais, o cabeçalho de caráter informativo, que explicita o programa em cálculo-$\lambda$ de entrada para ter gerado aquele programa em CPS.
Em seguida, há a definição das funções $\mathtt{cbn}$ e $\mathtt{cbv}$, ou seja, os programas correspondente em CPS para as duas traduções daquela entrada.
Por fim, a última parte é responsável pela computação do numeral de Church, as funções definidas irão calcular o número representado pelas expressões em CPS e retornar por fim uma tupla contendo o resultado do calculado pelo \textit{call-by-name} e \textit{call-by-value} respectivamente.

Ao se traduzir o numeral de Church 0 representado em cálculo-$\lambda$ por $\lambda f.\lambda x.\ x$, para CBN e CBV, tem-se:
\lstinputlisting[style=haskell, label=cps:church-zero-cps-cbn, caption={Tradução do numeral de Church ``0'' em CBN}]{Code/Type-Inferer/CPS/church-zero-cbn.cps}
\lstinputlisting[style=haskell, label=cps:church-zero-cps-cbv, caption={Tradução do numeral de Church ``0'' em CBV}]{Code/Type-Inferer/CPS/church-zero-cbv.cps}
Desta forma, o código gerado ao se traduzir esta função, é ilustrado a seguir:
\lstinputlisting[style=haskell, label=cps:church-zero-output, caption={Código gerado ao traduzir o numeral de Church ``0''}]{Code/Type-Inferer/CPS/church-zero.hs}
Ao executar o código e chamar a função $\mathtt{main}$ deste programa, o resultado obtido é justamente a computação do numeral para as duas traduções, ou seja, $\mathtt{(0,\ 0)}$.

\subsection{Fluxo Principal}\label{subsec:cps-main-program}
O fluxo completo de execução do programa principal contempla todas as funções apresentadas nesta seção, com a adição de funções auxiliares.
Essas são aplicadas em sequência, de modo a realizar uma série de ações.

Inicialmente, é passado o caminho de um arquivo contendo um programa em cálculo-$\lambda$ com adição do `let'.
O conteúdo então é processado pelo \textit{parser} e representado pelos tipos de dados algébricos para o cálculo lambda.
Uma vez que o programa já está sendo representado pelos ADTs, e ainda tem seu tipo inferido, é possível iniciar o processamento descrito pelas funções apresentadas.
A primeira delas é a tradução para CPS tanto em \textit{call-by-name} quanto em \textit{call-by-value}, as exibindo logo em seguida.
Com as traduções da expressão feitas o código Haskell já pode ser gerado, salvando assim no diretório \texttt{output} com mesmo nome do arquivo de entrada.
Como passo posterior, tem-se a tradução dos tipos para ambas as estratégias de avaliação.
Os passos finais envolvem a inferência de ambas as traduções, juntamente da verificação de subtipagem, onde esta informará se o tipo traduzido é um subtipo do inferido.

\lstset{extendedchars=false, escapeinside=''}
\lstinputlisting[style=output, label=cps:main-execution, caption={Execução do programa principal}]{Code/Type-Inferer/CPS_main_execution.cps}
Ao executar o programa com o comando \texttt{cabal run}, passando também o arquivo de entrada \texttt{input/church-zero.in}, é processado e exibida todas as informações que foi citada anteriormente, inclusive a geração do código Haskell em \texttt{output/church-zero.hs}.
É possível perceber que na saída do programa, é mostrado o tipo traduzido (na sequência da mensagem ``\textit{Expected Continuation Type:}'') e o tipo inferido (que sucede a mensagem ``\textit{Inferred Continuation Type:}'').
Logo em seguida, o questionamento ``\textit{Do the types match?}'' é o trecho da saída que compete à subtipagem, retornando ``\textit{Yes}'' caso esse seja um subtipo deste, o que indica uma inferência compatível com a tradução ou ``\textit{No}'' caso a verificação falhe, indicando uma inferência incorreta.

\lstset{extendedchars=false, escapeinside=''}
\lstinputlisting[style=output, label=cps:execution-generated, caption={Execução do programa gerado}]{Code/Type-Inferer/CPS_execution_generated.cps}
Ainda, a compilação e execução do código Haskell gerado pode ser conferida no Código~\ref{cps:execution-generated} acima, ilustrando exatamente o comportamento detalhado anteriormente.


\chapter{Resultados}\label{ch:resultados}
Uma vez que todo o fluxo do programa foi exibido, seus resultados podem ser apresentados e compreendidos.
Diversos testes foram executados onde, a partir da função de tradução de tipos, puderam ter seus tipos verificados para validar a implementação.
Para tal, estão disponibilizados no repositório do projeto, arquivos de entrada com funções contendo diferentes características.
Afim de observar o comportamento do programa abrangendo uma maior gama de opções, algumas das funções testadas são extensas e contam com combinações de regras do cálculo-$\lambda$ simplesmente tipado.

\section{Combinador S}
Um desses é o combinador S ($\lambda x.\lambda y.\lambda z.\ x\ z\ (y\ z)$), que apesar de não ser um termo extenso, este utiliza de uma combinação dos três construtores (variáveis, abstrações e aplicações) para que o termo seja obtido.
O tipo deste, é representado por $(\alpha \to \beta \to \gamma) \to (\alpha \to \beta) \to \alpha \to \gamma$ e suas traduções, tanto em CBN quanto em CBV são grandes demais para serem aqui colocadas, mas estas podem ser encontradas no código fonte do repositório mencionado.
A tradução do seu tipo entretanto, para CBN, é apresentada a seguir.
\lstset{extendedchars=false, escapeinside=''}
\begin{lstlisting}[style=output,caption={Tradução em CBN do tipo do combinador S}]
  '$\forall\alpha,\beta,\gamma.\ \neg\neg(\neg\neg\neg(\neg\neg\alpha,\ \neg\neg(\neg\neg\beta,\ \neg\gamma)),\ \neg\neg(\neg\neg\neg(\neg\neg\alpha,\ \neg\beta),\ \neg\neg(\neg\neg\alpha,\ \neg\gamma)))$'
\end{lstlisting}
Enquanto que, a partir do tipo inferido a seguir para a mesma estratégia de avaliação, uma substituição $S$ tal que ao aplicá-la no tipo traduzido torne-se o inferido, é $S = \{\ \alpha \mapsto \neg\alpha,\ \beta \mapsto \neg\beta,\ \gamma \mapsto \neg\gamma\ \}$, validando assim a inferência para este termo.
\lstset{extendedchars=false, escapeinside=''}
\begin{lstlisting}[style=output,caption={Inferência do tipo do combinador S traduzido em CBN}]
  '$\forall\alpha,\beta,\gamma.\ \neg\neg(\neg\neg\neg(\neg\alpha,\ \neg\neg(\neg\beta,\ \gamma)),\ \neg\neg(\neg\neg\neg(\neg\alpha,\ \beta),\ \neg\neg(\neg\alpha,\ \gamma)))$'
\end{lstlisting}
Um comportamento semelhante pode ser percebido para a tradução por \textit{call-by-value}, onde respectivamente é apresentado a seguir a tradução e o resultado da inferência.
\lstset{extendedchars=false, escapeinside=''}
\begin{lstlisting}[style=output,caption={Tradução em CBV do tipo do combinador S}]
  '$\forall\alpha,\beta,\gamma.\ \neg\neg(\neg(\alpha,\ \neg\neg(\beta,\ \neg\gamma)),\ \neg\neg(\neg(\alpha,\ \neg\beta),\ \neg\neg(\alpha,\ \neg\gamma)))$'
\end{lstlisting}
Neste caso, a substituição $S$ que satisfaz a condição de subtipagem é tal que $S = \{\ \gamma \mapsto \neg\gamma\ \}$, tornando válida assim a inferência para este termo.
\lstset{extendedchars=false, escapeinside=''}
\begin{lstlisting}[style=output,caption={Inferência do tipo do combinador S traduzido em CBV}]
  '$\forall\alpha,\beta,\gamma.\ \neg\neg(\neg(\alpha,\ \neg\neg(\beta,\ \gamma)),\ \neg\neg(\neg(\alpha,\ \neg\beta),\ \neg\neg(\alpha,\ \gamma)))$'
\end{lstlisting}

\section{Soma}
O próximo exemplo apresentado é a função de soma de 2 e 3 feita com os numerais de Church ($(\lambda n.\ \lambda m.\ \lambda f.\ \lambda x.\ n\ f\ (m\ f\ x))\ (\lambda a.\ \lambda b.\ a\ (a\ b))\ (\lambda c.\ \lambda d.\ c\ (c\ (c\ d)))$).
Seu tipo, é representado por $((\alpha \to \alpha) \to \alpha \to \alpha)$.
A tradução do tipo para CBN é dado por:
\lstset{extendedchars=false, escapeinside=''}
\begin{lstlisting}[style=output,caption={Tradução em CBN do tipo da função de soma}]
  '$\forall\alpha.\ \neg\neg(\neg\neg\neg(\neg\neg\alpha,\ \neg\alpha),\ \neg\neg(\neg\neg\alpha,\ \neg\alpha))$'
\end{lstlisting}
E o inferido também para CBN, onde a substituição $S$ que satisfaz a subtipagem é tal que $S = \{\ \alpha \mapsto \neg\alpha\ \}$:
\lstset{extendedchars=false, escapeinside=''}
\begin{lstlisting}[style=output,caption={Inferência do tipo da função de soma traduzido em CBN}]
  '$\forall\alpha.\ \neg\neg(\neg\neg\neg(\neg\alpha,\ \alpha),\ \neg\neg(\neg\alpha,\ \alpha))$'
\end{lstlisting}
Enquanto que para CBV, o tipo traduzido é:
\lstset{extendedchars=false, escapeinside=''}
\begin{lstlisting}[style=output,caption={Tradução em CBV do tipo da função de soma}]
  '$\forall\alpha.\ \neg\neg(\neg(\alpha,\ \neg\alpha),\ \neg\neg(\alpha,\ \neg\alpha))$'
\end{lstlisting}
O inferido portanto, sendo que a substituição $S$ que satisfaz a subtipagem neste caso é a substituição trivial $S = \{\ \alpha \mapsto \alpha\ \}$:
\lstset{extendedchars=false, escapeinside=''}
\begin{lstlisting}[style=output,caption={Inferência do tipo da função de soma traduzido em CBV}]
  '$\forall\alpha.\ \neg\neg(\neg(\alpha,\ \neg\alpha),\ \neg\neg(\alpha,\ \neg\alpha))$'
\end{lstlisting}
Por fim, ao executar o código gerado, obtém-se o resultado $(5,\ 5)$, indicando uma correta tradução da expressão de entrada.

\section{Multiplicação}
Outro exemplo é o da multiplicação de 6 e 8, que o termo lambda por si é grande demais para ser exibido aqui.
A função lambda responsável pela multiplicação de dois argumentos, ($\lambda m.\ \lambda n.\ \lambda f.\ \lambda x.\ m\ (n\ f)\ x$) é extenso o bastante para seu sucesso dar uma noção boa de que o código está correto e que o algoritmo é capaz de inferir o tipo corretamente das mais diversas expressões.
Assim como o exemplo da função de soma, o tipo desta expressão é: $((\alpha \to \alpha) \to \alpha \to \alpha)$.
A tradução do tipo para CBN é dado por:
\lstset{extendedchars=false, escapeinside=''}
\begin{lstlisting}[style=output,caption={Tradução em CBN do tipo da função de soma}]
  '$\forall\alpha.\ \neg\neg(\neg\neg\neg(\neg\neg\alpha,\ \neg\alpha),\ \neg\neg(\neg\neg\alpha,\ \neg\alpha))$'
\end{lstlisting}
E o inferido também para CBN, onde a substituição $S$ que satisfaz a subtipagem é tal que $S = \{\ \alpha \mapsto \neg\alpha\ \}$:
\lstset{extendedchars=false, escapeinside=''}
\begin{lstlisting}[style=output,caption={Inferência do tipo da função de soma traduzido em CBN}]
  '$\forall\alpha.\ \neg\neg(\neg\neg\neg(\neg\alpha,\ \alpha),\ \neg\neg(\neg\alpha,\ \alpha))$'
\end{lstlisting}
Enquanto que para CBV, o tipo traduzido é:
\lstset{extendedchars=false, escapeinside=''}
\begin{lstlisting}[style=output,caption={Tradução em CBV do tipo da função de soma}]
  '$\forall\alpha.\ \neg\neg(\neg(\alpha,\ \neg\alpha),\ \neg\neg(\alpha,\ \neg\alpha))$'
\end{lstlisting}
O inferido portanto, sendo que a substituição $S$ que satisfaz a subtipagem neste caso é a substituição trivial $S = \{\ \alpha \mapsto \alpha\ \}$:
\lstset{extendedchars=false, escapeinside=''}
\begin{lstlisting}[style=output,caption={Inferência do tipo da função de soma traduzido em CBV}]
  '$\forall\alpha.\ \neg\neg(\neg(\alpha,\ \neg\alpha),\ \neg\neg(\alpha,\ \neg\alpha))$'
\end{lstlisting}
É possível perceber que, os resultados obtidos são os mesmos da função de soma.
Isso se deve ao fato de que, por mais que a expressão seja diferente, as duas possuem o mesmo tipo.
Então, o algoritmo infere que as duas possuem o mesmo tipo.

Por fim, ao executar o código gerado, obtém-se o resultado $(48,\ 48)$, indicando uma correta tradução da expressão de entrada.
\section{Identidade com `Let'}
Ainda que a tradução do `let' tenha sido apresentada e discutida, esta possui um problema relacionada à tipagem quanto ao \textit{call-by-value}.
Para melhor compreender esta questão, é necessário entender fundamentalmente a diferença entre as estratégias de avaliação por nome e por valor.
A por nome, também conhecida como avaliação preguiçosa, só irá avaliar a expressão no momento em que esta for necessária, desta forma, caso tenha alguma expressão que não seja utilizada, esta nem mesmo será computada.
Enquanto que a por valor não, ao invés disso, ela avalia toda expressão no início, sem se importar se será utilizada ou não, desta forma, ainda que uma expressão não seja utilizada, ela será calculada.

A função em questão é a identidade utilizando o `let', ($\mathtt{let\ id\ =\ \lambda x.\ x\ in\ id\ id}$).
Seu tipo é $\alpha \to \alpha$, enquanto que sua tradução em CBN é dada por:
\lstinputlisting[style=haskell, label=cps:let-id-cps-cbn, caption={Tradução em CBN da identidade com `let'}]{Code/Type-Inferer/CPS/let-id-cbn.cps}
A tradução do tipo para CBN é dada por:
\lstset{extendedchars=false, escapeinside=''}
\begin{lstlisting}[style=output,caption={Tradução em CBN da identidade com `let'}]
  '$\forall\alpha.\ \neg\neg(\neg\neg\alpha,\ \neg\alpha)$'
\end{lstlisting}
E o inferido também para CBN, onde a substituição $S$ que satisfaz a subtipagem é tal que $S = \{\ \alpha \mapsto \neg\alpha \ \}$:
\lstset{extendedchars=false, escapeinside=''}
\begin{lstlisting}[style=output,caption={Inferência da identidade com `let' traduzido em CBN}]
  '$\forall\alpha.\ \neg\neg(\neg\alpha,\ \alpha)$'
\end{lstlisting}
Já ao olhar para o resultado em CBV, temos um problema de tipagem.
A tradução da expressão, que num primeiro momento não há nenhum problema aparente, é dada por:
\lstinputlisting[style=haskell, label=cps:let-id-cps-cbv, caption={Tradução em CBV da identidade com `let'}]{Code/Type-Inferer/CPS/let-id-cbv.cps}
Ao investigar mais a fundo porém, é possível ser notado na linha 1 que, como o $\mathtt{id}$ é passado por parâmetro para a continuação $\mathtt{k}$, durante o processo de inferência, esta será inserida no contexto como sendo um tipo polimórfico.
Em momento posterior entretanto, na inferência da expressão $\mathtt{id\ id}$, a expressão $\mathtt{id}$ é assumida como sendo um tipo monomórfico.
É neste instante então que é feita a verificação do \textit{occurs check} para garantir que tipos cíclicos (ou seja, um tipo estar contido em outro, tornando assim impossível a unificação destes) não sejam permitidos, onde esta falha, retornando assim o erro.
Desta forma, o algoritmo falha, identificando um erro de \textit{OccursCheck}, que será apresentado a seguir:
\lstset{extendedchars=false, escapeinside=''}
\begin{lstlisting}[style=output,caption={Erro de Inferência da identidade com `let' traduzido em CBV}]
  '$OccursCheck{:}\ \beta\ in\ \neg(\beta,\ \alpha)$'
\end{lstlisting}
Este erro indica que, ao tentar encontrar uma unificação para as variáveis de tipo, foi encontrado que uma delas já estava presente na outra, neste caso, que o $\beta$ já ocorria em $\neg(\beta,\ \alpha)$.
Como isto torna impossível que uma substituição seja suficiente, de modo a gerar um loop infinito caso seja tentado, um erro é retornado.

\section{Autoaplicação Mal-Tipada}\label{sec:autoap-mal-tipada}
Este exemplo demonstra um caso clássico de termo não tipável no cálculo lambda simplesmente tipado: a autoaplicação ($\lambda x.\ x\ x$).
No sistema de tipos original, a expressão requer que o tipo de $x$ seja ao mesmo tempo uma função (para ser aplicada) e o argumento dessa função, levando a uma contradição.
O erro de tipo gerado é:

\lstset{extendedchars=false, escapeinside=''}
\begin{lstlisting}[style=output,caption={Erro de tipo no cálculo lambda original}]
  Cannot unify { '$\alpha$' } with { '$\alpha \to \beta$' }
\end{lstlisting}
A tradução para CPS em \textit{call-by-name} é dada por:
\lstinputlisting[style=haskell, caption={Tradução em CBN da autoaplicação}]{Code/Type-Inferer/CPS/ill-typed-cbn.cps}
Durante a inferência de tipos no sistema CPS, o algoritmo falha com:
\begin{lstlisting}[style=output,caption={Erro de inferência em CBN}]
  OccursCheck: '$\epsilon$' in '$\neg\neg(\neg\epsilon,\ \gamma)$'
\end{lstlisting}
Já na estratégia \textit{call-by-value}, a tradução é:
\lstinputlisting[style=haskell, caption={Tradução em CBV da autoaplicação}]{Code/Type-Inferer/CPS/ill-typed-cbv.cps}
E o erro de inferência correspondente:
\begin{lstlisting}[style=output,caption={Erro de inferência em CBV}]
  OccursCheck: '$\beta$' in '$\neg(\beta,\ \gamma)$'
\end{lstlisting}
Em ambos os casos, o sistema CPS apresenta erro na tipagem, refletindo a não tipagem do termo original.

\section{Combinador Y}\label{sec:y-combinator}
O combinador Y ($\lambda f.\ (\lambda x.\ f\ (x\ x))\ (\lambda x.\ f\ (x\ x))$) é outro termo não tipável no cálculo lambda simplesmente tipado, essencial para expressar recursão em cálculos não tipados.
O erro de unificação é semelhante ao caso anterior:

\lstset{extendedchars=false, escapeinside=''}
\begin{lstlisting}[style=output,caption={Erro de tipo no cálculo lambda original}]
  Cannot unify { '$\beta$' } with { '$\beta \to \gamma$' }
\end{lstlisting}
A tradução em \textit{call-by-name} é:
\lstinputlisting[style=haskell, caption={Tradução em CBN do combinador Y}]{Code/Type-Inferer/CPS/y-combinator-cbn.cps}
A inferência de tipos no sistema CPS para CBN gera:
\begin{lstlisting}[style=output,caption={Erro de inferência em CBN}]
  OccursCheck: '$\lambda$' in '$\neg\neg(\neg\lambda,\ \iota)$'
\end{lstlisting}
Para \textit{call-by-value}, a tradução é a seguinte:
\lstinputlisting[style=haskell, caption={Tradução em CBV do combinador Y}]{Code/Type-Inferer/CPS/y-combinator-cbv.cps}
Com o seguinte erro de inferência:
\begin{lstlisting}[style=output,caption={Erro de inferência em CBV}]
  OccursCheck: '$\zeta$' in '$\neg(\zeta,\ \neg\iota)$'
\end{lstlisting}
Estes resultados demonstram que o sistema de tipos proposto é consistente com o cálculo lambda tradicional, identificando corretamente termos não tipáveis através de erros de ocorrência durante a unificação.

\section{Conclusão}\label{sec:conclusao}
As IRs são muito importantes na compilação, elas permitem otimizar muitos processos para assim tornar mais eficiente o código.
Os sistemas de tipos permitem provas de propriedades de um sistema, garantindo que programas tenham comportamento esperado.
Por exemplo, uma função que opera sobre um conjunto numérico, utilizando um sistema de tipos, seria possível identificar um uso incorreto desta função ao passar um argumento de outro tipo, podendo assim impedir que a função compute para evitar um comportamento inesperado.
Sendo assim, pode-se dizer que o sistema de tipos adiciona uma camada de segurança ao programa.

Uma das maiores motivações de adicionar essa camada de segurança na representação intermediária, ou seja de desenvolver um sistema de tipos para o CPS, é extender a prova de programas para as etapas seguintes em que o sistema já não tem mais ciência dos tipos, ou seja, o ligador (do inglês \textit{linker}).
Considerando o cenário onde um programa possua milhares de arquivos fonte, com IRs não tipadas, as provas das propriedades de funções definidas em um arquivo somente são passadas para outra caso estes sejam compilados juntos -- o que se torna inviável em programas tão grandes.
Já com uma IR tipada, se uma função $f$ definida em um arquivo espera receber um inteiro for utilizada em outro módulo onde na verdade está sendo passado como argumento uma cadeia de caracteres, o erro seria facilmente identificado no momento da ligação (do inglês \textit{linking}), não havendo a necessidade de recompilar o arquivo que contém a função ou ainda todos os que o utilizam.

Mesmo que o sistema não funcione para o `let' no CBV, a tradução está correta, onde isso significa respeitar o teorema da simulação.
Isto é, no caso de uma implementação não tipada da tradução, não haveria problemas em simular a expressão lambda no cálculo de continuações.
Como a tradução para CBV não preserva tipos, um dos trabalhos futuros incluiria a correção deste.
Para tal, seria necessário propor um outro sistema de tipos polimórfico para o cálculo de continuações onde, sejam distinguidos tipos e cotipos dentre os argumentos para que o algoritmo de inferência saiba quando utilizar um ou outro.

Outro trabalho futuro são as provas de consistência e completude do sistema de tipos proposto em relação ao algoritmo de inferência.
Neste trabalho foram executados alguns casos de testes que demostraram empiricamente que o algoritmo está correto.
Em conjunto, um avanço seria a criação de um compilador que faça uso desta IR tipada, colocando em prática toda a teoria aqui apresentada.

