\chapter{Introdução}\label{sec:introducao}

A compilação de programas envolve diversas fases, cada uma com funções específicas, como análise léxica, análise sintática, análise semântica, otimizações, e, finalmente, a geração de código.
Uma etapa crítica nesse processo é a otimização, que frequentemente se baseia em representações intermediárias (IRs).
Essas representações atuam como ponte entre o código fonte e o código de máquina, permitindo que transformações e otimizações sejam aplicadas de maneira mais eficaz~\cite{plotkin1975call}.

As representações intermediárias variam conforme o paradigma da linguagem de programação.
Para linguagens imperativas, a Representação em Atribuição Única Estática (SSA) é amplamente adotada.
Já em linguagens funcionais, a Forma Normal A (ANF) e o Estilo de Passagem de Continuação (CPS) se destacam.
Este trabalho foca especificamente no CPS, uma IR que oferece vantagens particulares em termos de otimização e simplicidade na geração de código.

Essas características do CPS se tornam ainda mais evidentes quando é comparado como diferentes linguagens lidam com o fluxo de execução.
Em linguagens de alto nível, por exemplo, a pilha de chamadas atua como uma abstração fundamental para gerenciar o controle de retorno das funções.
No entanto, em linguagens de baixo nível, como \textit{assembly}, o controle de fluxo é mais explícito e depende do uso direto de instruções, que operam sobre a pilha de chamadas para empilhar e desempilhar os endereços de retorno.
Nesse contexto, o CPS se destaca ao tornar as continuações explicitamente representadas no código.
Em vez de confiar na pilha de chamadas para gerenciar retornos, o CPS introduz um parâmetro adicional em cada função, representando a continuação {---} isto é, o que deve ser feito com o resultado da função~\cite{kennedy2007continuations}.
Desta forma, em vez de simplesmente retornar um valor diretamente, a função invoca essa continuação, transferindo explicitamente o controle à próxima etapa da computação.
Isso elimina a dependência da pilha de chamadas, simplificando o modelo de execução e tornando-o mais alinhado com as necessidades de linguagens de baixo nível.

Além disso, a adoção do CPS como representação intermediária vai além da tradução de linguagens de alto nível para código de máquina.
O CPS facilita a aplicação de otimizações avançadas, como a eliminação de chamadas de cauda e a fusão de funções, além de permitir uma correspondência mais direta com o código gerado em linguagens de montagem~\cite{flanagan1993essence}.

Por outro lado, um ponto importante a ser considerado é que, apesar de existirem sistemas de tipos para CPS {---} como o sistema simplesmente tipado proposto por~\citeonline{thielecke1997categorical} {---} muitas implementações optam por representações não tipadas~\cite{morrisett1999systemF,torrens2024operational}.
Apesar da predominância de linguagens intermediárias não tipadas, pode ser citado como exceção o projeto LLVM {---} uma infraestrutura amplamente utilizada para construção de compiladores {---} que adota uma IR com suporte a um sistema de tipos próprio.
Esse suporte é ainda expandido com o uso do MLIR, uma extensão modular da LLVM voltada à construção de representações intermediárias com estruturas de tipos mais expressivas e flexíveis.
Ainda assim, na prática, é comum que as IRs não estendam completamente os tipos até essa etapa da compilação.

Embora essa abordagem simplifique a implementação inicial, ela pode comprometer a segurança e a correção do código.
Um sistema de tipos robusto pode não apenas garantir a correção de certas transformações e otimizações, mas também identificar uma classe inteira de erros antes da execução, proporcionando assim maior confiabilidade ao processo de compilação.

Diante dessas considerações, este trabalho propõe apresentar e desenvolver uma formalização de um sistema de tipos para CPS, bem como um algoritmo de inferência de tipos para o mesmo.
A escolha da linguagem de programação para a solução proposta será Haskell.
Por ser uma linguagem funcional pura fortemente tipada, possui características desejáveis como transparência referencial~\cite{sondergaard1990transparency} e um sistema de tipos robusto para explorar as vantagens do CPS e aplicar o sistema de tipos de maneira rigorosa.
Dessa forma, a escolha de Haskell não apenas facilita o desenvolvimento de uma implementação segura e eficiente do CPS, como também conta com garantias de seguranças que são fundamentais para o sucesso deste trabalho.

\section{Contribuições}\label{sec:contribuicoes}

Este trabalho teve como contribuições a formalização através de uma implementação em Haskell e testes unitários de uma extensão para o sistema de tipos para CPS proposto por~\citeonline{thielecke1997categorical}, adicionando polimorfismo com suporte a inferência para esta representação intermediária.

Ou então, especificamente:
\begin{itemize}
  \item Formalização de um sistema de tipos para CPS com suporte a polimorfismo, bem como um algoritmo de inferência para tal;\@
  \item Implementação em Haskell de um algoritmo de inferência de tipos para CPS;\@
  \item Validar a implementação do algoritmo por meio do teste de inferência para expressões, através de funções de tradução do cálculo-$\lambda$ para o de CPS.\@
  \item Conclusão de que o sistema não preserva tipos para CBV.
\end{itemize}

\section{Trabalhos Relacionados}
Após um estudo na literatura a respeito da tipagem de linguagens de mais baixo nível em relação ao código fonte, alguns trabalhos correlatos foram encontrados.

Em~\cite{necula1997pcc}, o autor introduz o conceito de um código que carrega provas (PCC do inglês \textit{Proof-Carrying Code}), um paradigma em que programas de código de máquina carregam junto a si provas formais de que obedecem a certas propriedades de segurança.
O sistema de verificação do consumidor do código pode então validar essas provas de forma automática, sem necessidade de confiar no produtor.
O modelo proposto utiliza uma linguagem de políticas de segurança, uma lógica para especificações e provas, e uma máquina abstrata como base formal.
O trabalho demonstra a viabilidade prática do modelo com a implementação de um verificador rápido para código de baixo nível, destacando a aplicabilidade do PCC em ambientes com execução de código não confiável, como acontece em sistemas distribuídos.

Em~\cite{morrisett1999systemF}, os autores propõem uma linguagem de montagem tipada (TAL do inglês \textit{Typed Assembly Language}) e um compilador que traduz programas do Sistema F (do inglês \textit{System F}) para TAL de forma preservadora de tipos.
A linguagem TAL é baseada em uma arquitetura reduzida (RISC do inglês \textit{Reduced Instruction Set Computer}) e possui um sistema de tipos capaz de representar abstrações de linguagens de alto nível, como fechamentos e tuplas.
A tradução é dividida em etapas como conversão para CPS e conversão de fechamentos, mantendo a tipagem correta em cada fase.
O trabalho também apresenta uma abordagem simplificada para conversão de fechamentos polimórficos e discute o uso de TAL para geração de código seguro em ambientes com código não confiável.

Em~\cite{shao1998implementing}, os autores descrevem a implementação da linguagem intermediária tipada FLINT no compilador SML/NJ, com foco na eficiência e escalabilidade de compiladores que preservam tipos.
O trabalho defende que um compilador que mantém informações de tipo não será escalável a menos que todas as suas fases também preservem a complexidade assintótica de tempo e espaço ao representar e manipular tipos.
Para isso, são combinadas técnicas como \textit{hash-consing}, memoização e codificação lambda com suspensões, garantindo representações compactas em forma de grafos acíclicos direcionados (DAGs do inglês \textit{Directed Acyclic Graphs}).
Os autores mostram que, com essas técnicas, é possível realizar operações como substituição, igualdade e redução de tipos de forma eficiente, o que viabiliza o uso prático de linguagens intermediárias fortemente tipadas.

Em~\cite{bowman2018cps}, os autores demonstram que a tradução CPS preservadora de tipos para linguagens dependentemente tipadas com tipos $\Sigma$ e $\Pi$ não é impossível, como sugerido por resultados anteriores.
Utilizando \textit{answer-type polymorphism}, eles propõem novas traduções CPS \textit{call-by-name} e \textit{call-by-value} a partir do Cálculo de Construções, provando preservação de tipos e correção da compilação.
Para isso, estendem a linguagem alvo com regras adicionais justificadas por um teorema livre, e provam a consistência desse sistema por meio de um modelo no Cálculo das Construções extensional, obtendo assim segurança em tempo de ligação (do inglês \textit{link-time safety}).

\section{Metodologia}\label{sec:metodologia}

A metodologia deste trabalho consistiu em duas principais etapas: pesquisa bibliográfica e implementação.
A primeira envolveu uma extensa revisão de literatura sobre continuações e seu cálculo, bem como um aprofundamento no estudo de sistemas de tipos, com o objetivo de proporcionar uma compreensão completa.
A segunda contemplou a formalização do sistema de tipos e do algoritmo de inferência para o cálculo de continuações, junto da implementação destes.

No escopo deste trabalho, a validação do algoritmo ocorreu por meio de testes de implementação, analisando os tipos inferidos das expressões.
Em etapa posterior, serão necessárias as provas de consistência e de completude do algoritmo em relação ao sistema de tipos proposto.

\section{Estrutura do Trabalho}\label{sec:estrutura-trabalho}

A primeira etapa do trabalho consistiu principalmente na fundamentação teórica e revisão bibliográfica no estudo de CPS e sistemas de tipos.
Em razão disto, o Capítulo~\ref{ch:fundamentacao-teorica} contém os conceitos e definições necessários para entendimento do tema.
Este é separado em seções, tal que a Seção~\ref{sec:IR} aborda representação intermediária, com um aprofundamento em CPS na Subseção~\ref{subsec:cps}.
Teoria de tipos é então apresentada na Seção~\ref{sec:type-theory}, detalhando o Cálculo Lambda Simplesmente Tipado na Seção~\ref{sec:simply-typed-lambda-calculus}
Um aprofundamento no sistema Damas-Hindley-Milner na Seção~\ref{sec:damas-milner}, discutindo de maneira mais específica o algoritmo W na Subseção~\ref{subsec:w-algo}.

Já a segunda etapa do trabalho, consistiu da parte prática do trabalho, envolvendo a formalização do sistema de tipos bem como seu desenvolvimento e por fim a conclusão do trabalho.
O Capítulo~\ref{ch:desenvolvimento} agrupa tanto a formalização do sistema proposto, presente na Seção~\ref{sec:formalizacao} quanto a implementação deste, apresentada em detalhes na Seção~\ref{sec:implementacao}.
Os resultados obtidos são então apresentados e discutidos no Capítulo~\ref{ch:resultados}, dividindo-o em Seções referentes aos testes realizados.
Por fim, o Capítulo~\ref{ch:conclusao} conta com considerações finais bem como as conclusões do autor.

É importante destacar que todas as formalizações do sistema de tipos apresentadas na Seção~\ref{sec:formalizacao} foram elaboradas pelos orientadores em conjunto com o autor, durante reuniões de desenvolvimento conceitual, explicando as motivações e os raciocínios para chegar no resultado.
Até a data de conclusão deste trabalho, tais formalizações ainda não foram publicadas, estando em preparação uma produção acadêmica que as reunirá para futura referência.
