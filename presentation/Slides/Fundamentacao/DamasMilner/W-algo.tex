\newcommand{\defas}{\ensuremath{\overset{def}{=}}}
\newcommand{\fv}{\ensuremath{\text{FV}}}
\newcommand{\fvc}{\ensuremath{\text{FVC}}}
\newcommand{\eeq}{\ensuremath{\overset{e}{=}}}
\newcommand{\Append}{\ensuremath{\texttt{++}}}
\newcommand{\If}{\ensuremath{\text{se}}}
\newcommand{\Let}{\ensuremath{\text{let}}}
\newcommand{\In}{\ensuremath{\text{in}}}
\newcommand{\Then}{\ensuremath{\text{então}}}
\newcommand{\Return}{\ensuremath{\text{retorna}}}
\newcommand{\Else}{\ensuremath{\text{senão}}}
\newcommand{\Elseif}{\ensuremath{\text{senão se}}}
\newcommand{\Fail}{\ensuremath{\text{falha}}}
\newcommand{\Unify}{\ensuremath{\textit{unify}}}
\newcommand{\Occurs}{\ensuremath{\textit{occurs}}}
\newcommand{\True}{\ensuremath{\texttt{Verdadeiro}}}
\newcommand{\False}{\ensuremath{\texttt{Falso}}}
\newcommand{\Whitespace}{\ensuremath{\texttt{ }}}
\newcommand{\TODO}[1]{\textcolor{red}{\textbf{TODO:} #1}}

\begin{frame}{Algoritmo W}
    Algoritmo de inferência introduzido por~\citeonline{DAMAS1982}:
    \begin{itemize}
        \item Para linguagens funcionais
        \item[]
        \item Algoritmo eficiente
              \begin{itemize}
                  \item[$\blacktriangleright$] Na maioria dos casos
              \end{itemize}
        \item[]
        \item Unificação
              \begin{itemize}
                  \item[$\blacktriangleright$] Solucionar equações de tipos
                  \item[$\blacktriangleright$] Dados dois tipos $\tau_1$ e $\tau_2$
                  \item[$\blacktriangleright$] A unificação procura uma substituição $S$ tal que $S\tau_1 = S\tau_2$
                  \item[$\blacktriangleright$] Atribui os tipos mais gerais possíveis
              \end{itemize}
    \end{itemize}
\end{frame}
