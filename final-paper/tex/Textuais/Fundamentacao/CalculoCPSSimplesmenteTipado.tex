\section{Cálculo de CPS Simplesmente Tipado}\label{sec:cps-calculus-thielecke}
O Cálculo de CPS se origina como a IR do compilador descrita no livro do~\citeonline{appel1992compiling}.
Sua versão simplesmente tipada foi estudada na tese do~\citeonline{thielecke1997categorical}, tendo sido descrita de maneira categórica.
Neste trabalho entretanto, é adotada uma versão baseada em Teoria de Tipos, como apresentada em~\citeonline{torrens2024operational}.

O cálculo de continuações (do inglês, \textit{CPS-calculus}), é um sistema formal que leva o CPS além de seu uso tradicional como uma técnica de transformação de código, tratando-o como um modelo computacional por si só.
Enquanto o CPS é utilizado como uma IR em compiladores, o cálculo de continuações oferece uma estrutura para raciocinar formalmente sobre computações onde o fluxo de controle é explicitamente representado.
A sintaxe deste cálculo é descrita a seguir:

\begin{tabular}{lccl}
     Comandos & b & $\Coloneqq$ & $x(\vv{x}) \mid \mathtt{let}\ x(\vv{x}) = b\ \mathtt{in}\ b$ \\
\end{tabular}\label{cps-calculus}

\phantom{Newline}

\noindent
Vale ressaltar que \citeonline{thielecke1997categorical} possui uma sintaxe diferente para o cálculo de continuações, onde os termos são respectivamente representados como $k\langle\vv{x}\rangle$ e $b\ \{\ x\langle \vv{x} \rangle = b\ \}$.
Em razão da natureza prática e do público alvo deste trabalho, a notação adotada para o \textit{CPS-Calculus} será a do Appel.
Aqui, $k(\vv{x})$ representa um salto (do inglês \textit{jump}), isto é, uma chamada direta para a continuação $k$ com os parâmetros $\vv{x}$, essencialmente passando o controle para essa $k$.
Enquanto que $\texttt{let }k(\vv{y}) = c \texttt{ in } b$ representa um vínculo (do inglês \textit{binding}), onde a continuação $k$ com os parâmetros $\vv{y}$ é ligada dentro do comando $b$ com o corpo $c$.
Isto modela uma construção intermediária em que se define uma nova continuação que poderá ser chamada dentro de $b$ para prosseguir com a computação.

Neste sistema, há apenas um construtor de tipos: a negação poliádica (do inglês \textit{polyadic negation type}), usada para representar continuações~\cite{torrens2024operational}.
Além disso, $X$ varia sobre os tipos base.

\phantom{Newline}

\begin{tabular}{lccl}
  Tipos & $\tau$ & $\Coloneqq$ & $\neg\vv{\tau} \enspace|\enspace X$ \\
  Contexto & $\Gamma$ & $\Coloneqq$ & $\cdot \enspace|\enspace \Gamma,\ x{:}\ \tau$ \\
\end{tabular}\label{cps-simple-type-system}

\phantom{Newline}

\noindent O sistema apresenta somente duas regras de tipagem principais, uma para o \textit{jump} e outra para o \textit{bind}.

\fbox{$\Gamma\vdash b$}

\begin{prooftree}
    \RightLabel{$\mathtt{[J]}$}
    \AxiomC{$\Gamma(k) = \neg\vv{\tau}$}
    \AxiomC{$\Gamma(\vv{x}) = \vv{\tau}$}
    \BinaryInfC{$\Gamma\vdash k(\vv{x})$}
\end{prooftree}
\begin{prooftree}
    \RightLabel{$\mathtt{[B]}$}
    \AxiomC{$\Gamma,\ k{:}\ \neg\vv{\tau} \vdash b$}
    \AxiomC{$\Gamma,\ \vv{x}{:}\ \vv{\tau} \vdash c$}
    \BinaryInfC{$\Gamma\vdash \mathtt{let}\ k(\vv{x}) = c\ \mathtt{in}\ b$}
\end{prooftree}
Os julgamentos de tipo nesse cálculo são unilaterais, no sentido de que tipos aparecem apenas em um dos lados do \textit{turnstile} ($\vdash$).
Diferentemente do cálculo de tipos tradicional como o de Damas-Milner, onde um julgamento do tipo $\Gamma \vdash e{:}\ \tau$ atribui um tipo $\tau$ à expressão $e$ sob o contexto $\Gamma$, no cálculo de CPS é escrito apenas $\Gamma \vdash b$.
Isso significa que, sob o contexto $\Gamma$, o termo $b$ é consistente, ou seja, não vai falhar durante a execução~\cite{thielecke1997categorical}.
Por exemplo, a derivação ${x{:}\ \tau,\ k{:}\ \neg\tau} \vdash k(x)$ indica que, assumindo $x$ com tipo $\tau$ e $k$ com tipo $\neg\tau$, passar $x$ como argumento para $k$ é seguro.

\citeonline{torrens2024operational} demonstram a segurança de tipos (do inglês \textit{type safety}) deste sistema, provando as propriedades de preservação e progresso.
Isso garante que expressões bem tipadas não entram em estados inválidos ao serem avaliadas.

Um contraste interessante pode ser observado entre o sistema Damas-Milner e o cálculo de CPS.
No primeiro, as expressões são testemunhas da veracidade de seus tipos.
Já no segundo, as expressões podem ser vistas como testemunhas de que o contexto de tipos leva a uma contradição {---} nas palavras de Thielecke, ``$b$ witnesses that $\Gamma$ entails a contradiction''.
Como no exemplo do ${x{:}\ \tau,\ k{:}\ \neg\tau} \vdash k(x)$, observa-se que $k(x)$ representa uma contradição, ao termos tanto $\tau$ (em $x$) como $\neg\tau$ (em $k$).

Enquanto o cálculo-$\lambda$ define sua semântica por meio das regras $\beta$ e $\eta$, o \textit{CPS-calculus} possui uma teoria equacional própria, formulada em \citeonline{thielecke1997categorical}.
Esta teoria é composta por quatro axiomas fundamentais que permitem raciocinar sobre a equivalência de programas escritos nessa representação.
Esses axiomas são:

\phantom{NewLine}

\begin{center}
     \begin{tabular}{llllr}
          (JMP) & $k\langle \vv{x} \rangle \, \{\ k\langle \vv{y} \rangle = c\ \}$ & = & $c[\vv{x} / \vv{y}]$ & $\mathtt{se}\ k \notin \vv{x}$ \\
     \end{tabular}
\end{center}
O \textit{Jump (JMP)} realiza uma chamada direta à continuação $k$, substituindo os parâmetros formais $\vec{y}$ pelos argumentos reais $\vec{x}$ no corpo $c$.
A necessidade da condição $k \notin \vec{x}$ é tal que esta evita capturas.

\begin{center}
     \begin{tabular}{llllr}
          (GC)    & $k\langle \vv{x} \rangle \, \{\ j\langle \vv{y} \rangle = c\ \}$ & = & $k\langle \vv{x} \rangle$ & $\mathtt{se}\ j \notin \mathtt{FV}(k\langle \vv{x}\rangle)$ \\
     \end{tabular}
\end{center}
O axioma \textit{Garbage Collection (GC)}, elimina os vínculos de continuações não referenciadas em $c$.

\begin{center}
     \begin{tabular}{llllr}
          (ETA)   & $b\ \{ k\langle \vv{x} \rangle = j\langle \vv{x} \rangle\ \}$ & = & $b[j / k]$ & $\mathtt{se}\ j \notin \vv{x}$ \\
     \end{tabular}
\end{center}
O axioma da extensionalidade (ETA) diz respeito as continuações redundantes, onde $k$ é definido apenas como uma chamada direta a $j$.

\begin{center}
     \begin{tabular}{llllr}
          (DISTR) & \multicolumn{4}{l}{$b\ \{\ k\langle \vv{x} \rangle = c\ \} \{\ j\langle \vv{y} \rangle = d\ \}$ = $b\ \{\ j\langle \vv{y} \rangle = d\ \} \{\ k\langle \vv{x} \rangle = c\ \{\ j\langle \vv{y} \rangle = d\ \} \}$} \\
                  & \multicolumn{4}{r}{$\mathtt{se}\ k \ne j,\ j \notin \vv{x},\ \{\ k,\ \vv{x} \ \} \notin \mathtt{FV}(d)$} \\
     \end{tabular}
\end{center}
Por fim, a distribuição, com o axioma (DISTR) reordena os vínculos para expor saltos.

\subsection{Tradução CPS}\label{subsec:cps-translation}
O processo de tradução, no contexto de linguagens é a etapa em que um código feito em uma linguagem é transcrito para outra, onde idealmente, semântica alguma deve ser perdida.
\citeonline{plotkin1975call} introduziu duas traduções do cálculo-$\lambda$, posteriormente adaptadas ao cálculo de CPS por \citeonline{thielecke1997categorical}, diferenciando-se conforme a estratégia de avaliação adotada.
A tradução por chamada por nome (CBN, do inglês \textit{call-by-name}) permite simular a execução no cálculo-${\lambda}$, representada por ${\llbracket - \rrbracket}_N$; de maneira análoga, a tradução por chamada por valor (CBV, do inglês \textit{call-by-value}) é representada por ${\llbracket - \rrbracket}_V$ no cálculo-${\lambda}_V$.
Embora essas traduções não sejam as mais eficientes, foram escolhidas por sua clareza e simplicidade.

\phantom{Newline}

\fbox{${\llbracket\ e\ \rrbracket}_N\ =\ b$} \textit{Plotkin's call-by-name translation}

\begin{tabular}{lcl}
  ${\llbracket\ x\ \rrbracket}_N$ & $=$ & $x(k)$ \\
  ${\llbracket\ \lambda x.\ e\ \rrbracket}_N$ & $=$ & $\mathtt{let}\ v(k,\ x) = {\llbracket e \rrbracket}_N\ \mathtt{in}\ k(v)$ \\
  ${\llbracket\ f\ e\ \rrbracket}_N$ & $=$ & $\mathtt{let}\ k(f) = \mathtt{let}\ v(k) = {\llbracket\ e\ \rrbracket}_N\ \mathtt{in}\ c\ \mathtt{in}\ {\llbracket\ f\ \rrbracket}_N$ \\
  ${\llbracket\ \mathtt{let}\ x = f\ \mathtt{in}\ e \rrbracket}_N$ & $=$ & $\mathtt{let}\ x(k) = {\llbracket\ f\ \rrbracket}_N\ \mathtt{in}\ {\llbracket\ e\ \rrbracket}_N$
\end{tabular}

\phantom{Newline}

\fbox{${\llbracket\ e\ \rrbracket}_V\ =\ b$} \textit{Plotkin's call-by-value translation}

\begin{tabular}{lcl}
  ${\llbracket\ x\ \rrbracket}_V$ & $=$ & $k(x)$ \\
  ${\llbracket\ \lambda x.\ e\ \rrbracket}_V$ & $=$ & $\mathtt{let}\ v(k,\ x) = {\llbracket e \rrbracket}_V\ \mathtt{in}\ k(v)$ \\
  ${\llbracket\ f\ e\ \rrbracket}_V$ & $=$ & $\mathtt{let}\ k(f) = \mathtt{let}\ v(k) = c\ \mathtt{in}\ {\llbracket\ e\ \rrbracket}_V\ \mathtt{in}\ {\llbracket\ f\ \rrbracket}_V$ \\
  ${\llbracket\ \mathtt{let}\ x = f\ \mathtt{in}\ e \rrbracket}_V$ & $=$ & $\mathtt{let}\ k(x) = {\llbracket\ e\ \rrbracket}_V\ \mathtt{in}\ {\llbracket\ f\ \rrbracket}_V$
\end{tabular}

\phantom{Newline}

\noindent As variáveis $f$, $v$ e $k$ que não aparecem no termo de origem são consideradas frescas, onde as duas primeiras são imediatamente ligadas, enquanto que $k$ é esperada que seja livre e por ser vinculada às traduções dos subtermos, um único $k$ é necessário~\cite{torrens2024operational}.

Ainda, a tradução do `let' não está presente no artigo.
Esta foi feita para que o trabalho em questão pudesse ser mais completo e mais próximo de algo utilizável em um caso real de implementação de compilador com uso desta IR.
A maneira de se obtê-la foi a partir de uma beta-redex, simplificando a tradução de $((\lambda x.\ b)\ a)$ por meio dos axiomas.

Além da tradução de expressões, uma vez que o ambiente sendo trabalhado é tipado, é necessário que a tradução preserve a tipagem dos programas,~\citeonline{torrens2024operational} apresentam a tradução tipada do CPS, onde, para o tipo funcional $\mathtt{A \Coloneqq A \to A\ |\ X}$, as seguintes traduções são definidas.
Note que aqui também as funções são definidas diferentemente de acordo com estratégia de avaliação adotada:

\phantom{Newline}

\begin{tabular}{lcl}
     ${\llbracket\ X\ \rrbracket}_N$ & $=$ & $\neg X$ \\ 
     ${\llbracket\ A \rightarrow B\ \rrbracket}_N$ & $=$ & $\neg\neg(\neg{\llbracket\ A\ \rrbracket}_N,\ {\llbracket\ B\ \rrbracket}_N)$ \\ 
     ${\llbracket\ \vv{x}{:}\ \vv{A} \vdash e{:}\ B\ \rrbracket}_N$ & $=$ & $\vv{x}{:}\ \vv{\neg{\llbracket\ A\ \rrbracket}_N},\ k{:}\ {\llbracket\ B\ \rrbracket}_N \vdash {\llbracket\ e\ \rrbracket}_N$ \\ 
\end{tabular}

\phantom{Newline}

\begin{tabular}{lcl}
     ${\llbracket\ X\ \rrbracket}_V$ & $=$ & $X$ \\ 
     ${\llbracket\ A \rightarrow B\ \rrbracket}_V$ & $=$ & $\neg({\llbracket\ A\ \rrbracket}_V,\ \neg{\llbracket\ B\ \rrbracket}_V)$ \\ 
     ${\llbracket\ \vv{x}{:}\ \vv{A} \vdash e{:}\ B\ \rrbracket}_V$ & $=$ & $\vv{x}{:}\ \vv{{\llbracket\ A\ \rrbracket}_V},\ k{:}\ \neg{\llbracket\ B\ \rrbracket}_V \vdash {\llbracket\ e\ \rrbracket}_V$ \\ 
\end{tabular}

\phantom{Newline}

\noindent Esta função de tradução tem como propósito, mostrar que se o termo lambda é tipado, então o termo traduzido para CPS também é tipado.
É importante notar que isto se aplica somente para os tipos simples, como definido no tipo funcional a partir do tipo primitivo $\mathtt{X}$ e do \textit{arrow type} $\mathtt{A \to A}$.
O sistema de tipos proposto neste trabalho, entretanto, é polimórfico, ou seja, aqui há a adição de variáveis de tipos quantificadas.
Sendo assim, esta função de tradução, se aplicando neste caso de uso, não necessariamente retornará o tipo mais geral de uma expressão, mas sempre um subtipo deste.
