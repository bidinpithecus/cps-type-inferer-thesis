\begin{frame}{Sistema Damas-Milner}
    Sistema de tipos robusto para linguagens funcionais~\cite{MILNER1978, DAMAS1982}
    \begin{itemize}
        \item Inferência automática de tipos polimórficos
        \item Cálculo-$\lambda$ com polomorfismo paramétrico introduzido via \texttt{let}
        \item[]
    \end{itemize}

    Sua sintaxe define as expressões e os tipos usados no processo de inferência:

    \begin{equation*}\label{eq:dm-syntax}
        \begin{array}{ll}
            \text{Variáveis}         & x                                                                                 \\
            \text{Expressões}        & e ::= x \mid e \ e' \mid \lambda x.e \mid \texttt{let} \ x = e \ \texttt{in} \ e' \\
            \\
            \text{Variáveis de tipo} & \alpha                                                                            \\
            \text{Tipos primitivos}  & \iota                                                                             \\
            \text{Tipos}             & \tau ::= \alpha \mid \iota \mid \tau \rightarrow \tau
            \alpha                                                                                                       \\
            \text{Schemes}           & \sigma ::= \forall \alpha. \sigma \mid \tau                                       \\
        \end{array}
    \end{equation*}
\end{frame}

\begin{frame}{Sistema Damas-Milner}
    \begin{itemize}
        \item Generalização
              \begin{itemize}
                  \item[$\blacktriangleright$] Tipo mais geral
              \end{itemize}
        \item Instanciação
              \begin{itemize}
                  \item[$\blacktriangleright$] Tipo concreto
              \end{itemize}
        \item $\texttt{let id} \ = \lambda x.x \ \texttt{in} \ (\texttt{id} \ \texttt{1}, \ \texttt{id} \ \texttt{`a'})$
              \begin{itemize}
                  \item[$\blacktriangleright$] $\forall \alpha. \alpha \rightarrow \alpha$
              \end{itemize}
        \item[]
        \item Substituição de tipos
              \begin{itemize}
                  \item[$\blacktriangleright$] Mapeamento finito de variáveis de tipos para tipos, denotado por $S$
                  \item[$\blacktriangleright$] $[ \alpha_1 \mapsto \tau_1, \alpha_2 \mapsto \tau_2, \ldots, \alpha_n \mapsto \tau_n ]$
                  \item[$\blacktriangleright$] $S$ associa cada variável de tipo $\alpha_i$ a um tipo $\tau_i$ específico
                  \item[$\blacktriangleright$] $S\tau$ substitui todas as ocorrências livres de $\alpha_i$ em $\tau$ por $\tau_i$
              \end{itemize}
    \end{itemize}
\end{frame}

\begin{frame}{Sistema Damas-Milner}
    \begin{figure}[ht!]
        \caption{Regras de Inferência do sistema Damas-Milner}
        \centering
        \[
            \begin{array}{c}
                \text{TAUT:} \quad \displaystyle\frac{x : \sigma \in \Gamma}{\Gamma \vdash x : \sigma}                                               \\[25pt]

                \text{ABS:} \quad \displaystyle\frac{\Gamma, x : \tau \vdash e : \tau'}{\Gamma \vdash (\lambda x. e) : \tau \to \tau'}               \\[25pt]

                \text{APP:} \quad \displaystyle\frac{\Gamma \vdash e : \tau' \to \tau \quad \Gamma \vdash e' : \tau'}{\Gamma \vdash (e \ e') : \tau} \\[25pt]
            \end{array}
        \]
        \small{Fonte: o autor. Adaptado de~\cite{DAMAS1982}}
    \end{figure}
\end{frame}

\begin{frame}{Sistema Damas-Milner}
    \begin{figure}[ht!]
        \caption{Regras de Inferência do sistema Damas-Milner (continuação)}
        \centering
        \[
            \begin{array}{c}
                \text{LET:} \quad \displaystyle\frac{\Gamma \vdash e : \sigma \quad \Gamma, x : \sigma \vdash e' : \tau}{\Gamma \vdash (\texttt{let } x = e \ \texttt{in} \ e') : \tau} \\[25pt]

                \text{INST:} \quad \displaystyle\frac{\Gamma \vdash e : \sigma}{\Gamma \vdash e : \sigma'} \quad (\sigma > \sigma')                                                     \\[25pt]

                \text{GEN:} \quad \displaystyle\frac{\Gamma \vdash e : \sigma}{\Gamma \vdash e : \forall \alpha \sigma} \quad (\alpha \text{ não livre em } \Gamma)
            \end{array}
        \]
        \small{Fonte: o autor. Adaptado de~\cite{DAMAS1982}}
    \end{figure}
\end{frame}
