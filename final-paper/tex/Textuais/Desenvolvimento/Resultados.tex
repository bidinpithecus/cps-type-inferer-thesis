\section{Resultados}\label{sec:resultados}
Uma vez que todo o fluxo do programa foi exibido, seus resultados podem ser apresentados e compreendidos.
Diversos testes foram executados onde, a partir da função de tradução de tipos, puderam ter seus tipos verificados para validar a implementação.
Para tal, estão disponibilizados no repositório do projeto, arquivos de entrada com funções contendo diferentes características.
Afim de observar o comportamento do programa abrangendo uma maior gama de opções, algumas das funções testadas são extensas e contam com combinações de regras do cálculo-$\lambda$ simplesmente tipado.

\subsection{Combinador S}
Um desses, é o combinador S ($\lambda x.\lambda y.\lambda z.\ x\ z\ (y\ z)$), que apesar de não ser um termo extenso, este utiliza de uma combinação dos três construtores (variáveis, abstrações e aplicações) para que o termo seja optido.
O tipo deste, é representado por $(\alpha \to \beta \to \gamma) \to (\alpha \to \beta) \to \alpha \to \gamma$ e suas traduções, tanto em CBN quanto em CBV são grandes demais para serem aqui colocadas, mas estas podem ser encontradas no código fonte do repositório mencionado.
A tradução do seu tipo entretanto, para CBN, é apresentada a seguir.
\lstset{extendedchars=false, escapeinside=''}
\begin{lstlisting}[style=output,caption={Tradução em CBN do tipo do combinador S}]
  '$\forall\alpha,\beta,\gamma.\ \neg\neg(\neg\neg\neg(\neg\neg\alpha,\ \neg\neg(\neg\neg\beta,\ \neg\gamma)),\ \neg\neg(\neg\neg\neg(\neg\neg\alpha,\ \neg\beta),\ \neg\neg(\neg\neg\alpha,\ \neg\gamma)))$'
\end{lstlisting}
Enquanto que, a partir do tipo inferido a seguir para a mesma estratégia de avaliação, uma substituição $S$ tal que ao aplicá-la no tipo traduzido torne-se o inferido, é $S = \{\ \alpha \mapsto \neg\alpha,\ \beta \mapsto \neg\beta,\ \gamma \mapsto \neg\gamma\ \}$, validando assim a inferência para este termo.
\lstset{extendedchars=false, escapeinside=''}
\begin{lstlisting}[style=output,caption={Inferência do tipo do combinador S traduzido em CBN}]
  '$\forall\alpha,\beta,\gamma.\ \neg\neg(\neg\neg\neg(\neg\alpha,\ \neg\neg(\neg\beta,\ \gamma)),\ \neg\neg(\neg\neg\neg(\neg\alpha,\ \beta),\ \neg\neg(\neg\alpha,\ \gamma)))$'
\end{lstlisting}
Um comportamento semelhante pode ser percebido para a tradução por \textit{call-by-value}, onde respectivamente é apresentado a seguir a tradução e o resultado da inferência.
\lstset{extendedchars=false, escapeinside=''}
\begin{lstlisting}[style=output,caption={Tradução em CBV do tipo do combinador S}]
  '$\forall\alpha,\beta,\gamma.\ \neg\neg(\neg(\alpha,\ \neg\neg(\beta,\ \neg\gamma)),\ \neg\neg(\neg(\alpha,\ \neg\beta),\ \neg\neg(\alpha,\ \neg\gamma)))$'
\end{lstlisting}
Neste caso, a substituição $S$ que satisfaz a condição de subtipagem é tal que $S = \{\ \gamma \mapsto \neg\gamma\ \}$, tornando válida assim a inferência para este termo.
\lstset{extendedchars=false, escapeinside=''}
\begin{lstlisting}[style=output,caption={Inferência do tipo do combinador S traduzido em CBV}]
  '$\forall\alpha,\beta,\gamma.\ \neg\neg(\neg(\alpha,\ \neg\neg(\beta,\ \gamma)),\ \neg\neg(\neg(\alpha,\ \neg\beta),\ \neg\neg(\alpha,\ \gamma)))$'
\end{lstlisting}

\subsection{Soma}
O próximo exemplo apresentado é a função de soma de 2 e 3 feita com os numerais de Church ($(\lambda n.\ \lambda m.\ \lambda f.\ \lambda x.\ n\ f\ (m\ f\ x))\ (\lambda a.\ \lambda b.\ a\ (a\ b))\ (\lambda c.\ \lambda d.\ c\ (c\ (c\ d)))$).
Seu tipo, é representado por $((\alpha \to \alpha) \to \alpha \to \alpha)$.
A tradução do tipo para CBN é dado por:
\lstset{extendedchars=false, escapeinside=''}
\begin{lstlisting}[style=output,caption={Tradução em CBN do tipo da função de soma}]
  '$\forall\alpha.\ \neg\neg(\neg\neg\neg(\neg\neg\alpha,\ \neg\alpha),\ \neg\neg(\neg\neg\alpha,\ \neg\alpha))$'
\end{lstlisting}
E o inferido também para CBN, onde a substituição $S$ que satisfaz a subtipagem é tal que $S = \{\ \alpha \mapsto \neg\alpha\ \}$:
\lstset{extendedchars=false, escapeinside=''}
\begin{lstlisting}[style=output,caption={Inferência do tipo da função de soma traduzido em CBN}]
  '$\forall\alpha.\ \neg\neg(\neg\neg\neg(\neg\alpha,\ \alpha),\ \neg\neg(\neg\alpha,\ \alpha))$'
\end{lstlisting}
Enquanto que para CBV, o tipo traduzido é:
\lstset{extendedchars=false, escapeinside=''}
\begin{lstlisting}[style=output,caption={Tradução em CBV do tipo da função de soma}]
  '$\forall\alpha.\ \neg\neg(\neg(\alpha,\ \neg\alpha),\ \neg\neg(\alpha,\ \neg\alpha))$'
\end{lstlisting}
O inferido portanto, sendo que a substituição $S$ que satisfaz a subtipagem neste caso é a substituição trivial $S = \{\ \alpha \mapsto \alpha\ \}$:
\lstset{extendedchars=false, escapeinside=''}
\begin{lstlisting}[style=output,caption={Inferência do tipo da função de soma traduzido em CBV}]
  '$\forall\alpha.\ \neg\neg(\neg(\alpha,\ \neg\alpha),\ \neg\neg(\alpha,\ \neg\alpha))$'
\end{lstlisting}
Por fim, ao executar o código gerado, obtém-se o resultado $(5,\ 5)$, indicando uma correta tradução da expressão de entrada.

\subsection{Multiplicação}
Outro exemplo é o da multiplicação de 6 e 8, que o termo lambda por si é grande demais para ser exibido aqui.
A função lambda responsável pela multiplicação de dois argumentos, ($\lambda m.\ \lambda n.\ \lambda f.\ \lambda x.\ m\ (n\ f)\ x$) é extenso o bastante para seu sucesso dar uma noção boa de que o código está correto e que o algoritmo é capaz de inferir o tipo corretamente das mais diversas expressões.
Assim como o exemplo da função de soma, o tipo desta expressão é: $((\alpha \to \alpha) \to \alpha \to \alpha)$.
A tradução do tipo para CBN é dado por:
\lstset{extendedchars=false, escapeinside=''}
\begin{lstlisting}[style=output,caption={Tradução em CBN do tipo da função de soma}]
  '$\forall\alpha.\ \neg\neg(\neg\neg\neg(\neg\neg\alpha,\ \neg\alpha),\ \neg\neg(\neg\neg\alpha,\ \neg\alpha))$'
\end{lstlisting}
E o inferido também para CBN, onde a substituição $S$ que satisfaz a subtipagem é tal que $S = \{\ \alpha \mapsto \neg\alpha\ \}$:
\lstset{extendedchars=false, escapeinside=''}
\begin{lstlisting}[style=output,caption={Inferência do tipo da função de soma traduzido em CBN}]
  '$\forall\alpha.\ \neg\neg(\neg\neg\neg(\neg\alpha,\ \alpha),\ \neg\neg(\neg\alpha,\ \alpha))$'
\end{lstlisting}
Enquanto que para CBV, o tipo traduzido é:
\lstset{extendedchars=false, escapeinside=''}
\begin{lstlisting}[style=output,caption={Tradução em CBV do tipo da função de soma}]
  '$\forall\alpha.\ \neg\neg(\neg(\alpha,\ \neg\alpha),\ \neg\neg(\alpha,\ \neg\alpha))$'
\end{lstlisting}
O inferido portanto, sendo que a substituição $S$ que satisfaz a subtipagem neste caso é a substituição trivial $S = \{\ \alpha \mapsto \alpha\ \}$:
\lstset{extendedchars=false, escapeinside=''}
\begin{lstlisting}[style=output,caption={Inferência do tipo da função de soma traduzido em CBV}]
  '$\forall\alpha.\ \neg\neg(\neg(\alpha,\ \neg\alpha),\ \neg\neg(\alpha,\ \neg\alpha))$'
\end{lstlisting}
É possível perceber que, os resultados obtidos são os mesmos da função de soma.
Isso se deve ao fato de que, por mais que a expressão seja diferente, as duas possuem o mesmo tipo.
Então, o algoritmo infere que as duas possuem o mesmo tipo.

Por fim, ao executar o código gerado, obtém-se o resultado $(48,\ 48)$, indicando uma correta tradução da expressão de entrada.
\subsection{Identidade com `Let'}
Ainda que a tradução do `let' tenha sido apresentada e discutida, esta possui um problema relacionada à tipagem quanto ao \textit{call-by-value}.
Para melhor compreender esta questão, é necessário entender fundamentalmente a diferença entre as estratégias de avaliação por nome e por valor.
A por nome, também conhecida como avaliação preguiçosa, só irá avaliar a expressão no momento em que esta for necessária, desta forma, caso tenha alguma expressão que não seja utilizada, esta nem mesmo será computada.
Enquanto que a por valor não, ao invés disso, ela avalia toda expressão no início, sem se importar se será utilizada ou não, desta forma, ainda que uma expressão não seja utilizada, ela será calculada.

A função em questão é a identidade utilizando o `let', ($\mathtt{let\ id\ =\ \lambda x.\ x\ in\ id\ id}$).
Seu tipo é $\alpha \to \alpha$, enquanto que sua tradução em CBN é dada por:
\lstinputlisting[style=haskell, label=cps:let-id-cps-cbn, caption={Tradução em CBN da identidade com `let'}]{Code/Type-Inferer/CPS/let-id-cbn.cps}
A tradução do tipo para CBN é dado por:
\lstset{extendedchars=false, escapeinside=''}
\begin{lstlisting}[style=output,caption={Tradução em CBN da identidade com `let'}]
  '$\forall\alpha.\ \neg\neg(\neg\neg\alpha,\ \neg\alpha)$'
\end{lstlisting}
E o inferido também para CBN, onde a substituição $S$ que satisfaz a subtipagem é tal que $S = \{\ \alpha \mapsto \neg\alpha \ \}$:
\lstset{extendedchars=false, escapeinside=''}
\begin{lstlisting}[style=output,caption={Inferência da identidade com `let' traduzido em CBN}]
  '$\forall\alpha.\ \neg\neg(\neg\alpha,\ \alpha)$'
\end{lstlisting}
Já ao olhar para o resultado em CBV, temos um problema de tipagem.
A tradução da expressão, é dada por, que num primeiro momento não há nenhum problema aparente:
\lstinputlisting[style=haskell, label=cps:let-id-cps-cbv, caption={Tradução em CBV da identidade com `let'}]{Code/Type-Inferer/CPS/let-id-cbv.cps}
Ao investigar mais a fundo porém, é possível ser notado na linha 1 que, como o $\mathtt{id}$ é passado por parâmetro para a continuação $\mathtt{k}$, durante o processo de inferência, esta será inserida no contexto como sendo um tipo polimórfico.
Em momento posterior entretanto, na inferência da expressão $\mathtt{id\ id}$, a expressão $\mathtt{id}$ é assumida como sendo um tipo monomórfico.
É neste instante então, que é feita a verificação do \textit{occurs check} para garantir que tipos cíclicos (ou seja, um tipo estar contido em outro, tornando assim impossível a unificação destes) não sejam permitidos, onde esta falha, retornando assim o erro.
Desta forma, o algoritmo falha, identificando um erro de \textit{OccursCheck}, que será apresentado a seguir:
\lstset{extendedchars=false, escapeinside=''}
\begin{lstlisting}[style=output,caption={Erro de Inferência da identidade com `let' traduzido em CBV}]
  '$OccursCheck{:}\ \beta\ in\ \neg(\beta,\ \alpha)$'
\end{lstlisting}
Este erro indica que, ao tentar encontrar uma unificação para as variáveis de tipo, foi encontrado que uma delas já estava presente na outra, neste caso, que o $\beta$ já ocorria em $\neg(\beta,\ \alpha)$.
Como isto torna impossível que uma substituição seja suficiente, de modo a gerar um loop infinito caso seja tentado, um erro é retornado.

\subsection{Considerações finais}
Uma vez que identificado ao menos um programa que não é capaz de inferir tipos para a tradução via `let' da expressão, o seguinte teorema pode ser definido.
\begin{teorema}[A tradução para CBV não preserva tipos]\label{teo:cbv-does-not-preserve-types}
  A tradução do cálculo-$\lambda$ para o cálculo de continuações em CBV não preserva tipos.
\end{teorema}

\begin{proof}[Prova do Teorema~\ref{teo:cbv-does-not-preserve-types}]
  Podemos observar pelo contra exemplo: $\mathtt{let\ id\ =\ \lambda x.\ x\ in\ id\ id}$.
  Este termo não é tipável no cálculo de CPS polimórfico proposto.
  \qedhere
\end{proof}
\noindent Ainda, durante o desenvolvimento, ao se deparar com o erro de tipagem do `let' no CBV, o pensamento inicial foi de que alguma confusão quanto a implementação da tradução havia sido cometida.
Somente depois de muita análise e raciocínio a respeito do ocorrido foi compreendido que não se tratava de um erro de implementação, e sim de uma prova de que a tradução para CBV não preserva a tipagem dos termos.

Todos os testes realizados para CBN funcionaram, assim, é um forte indício que a tradução para este preserva a tipagem das expressões lambda.
Desta forma, é definida a seguinte conjectura.
\begin{conjectura}[A tradução para CBN preserva tipos]\label{conj:cbn-preserve-types}
  A tradução do cálculo-$\lambda$ para o cálculo de continuações em CBN preserva tipos.
\end{conjectura}

Por mais que a implementação em si não tenha sido complicada de modo geral, algumas dificuldades mais teóricas se mantiveram latentes durante boa parte do desenvolvimento.
O mais incômodo deles, refere-se a falta de materiais de referência mais didáticos.
A maioria deles, se não todos, apresenta as continuações utilizando conceitos sem previamente contextualizá-los.
É esperado que o leitor tenha amplo conhecimento a respeito de Teoria de Tipos e Teoria das Categorias, o que dificulta o entendimento daqueles que não o tem.

Ao chegar no resultado de uma inferência, seja executando o algoritmo via código ou no papel, tem-se a prova do absurdo de uma continuação.
Um dos pontos mais iniciais de dúvidas era referente ao tipo desta continuação.
Seja a função identidade traduzida em CBV, ou seja, com tipo $\forall\alpha.\ \neg\neg(\neg\alpha,\ \alpha)$, a dificuldade que perdurava era como raciocinar sobre ele.
Mudar a maneira de pensar para compreender os resultados, que são duais dos tipos no cálculo-$\lambda$, ou seja, a conclusão deste tipo é que ele é falso e não verdadeiro, foi uma das maiores barreiras para entendimento.
