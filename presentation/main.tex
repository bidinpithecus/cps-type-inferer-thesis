\documentclass[xcolor=table]{beamer}
\usepackage[utf8]{inputenc}
\usepackage[T1]{fontenc}
\usepackage[alf]{abntex2cite}
\usepackage{udesc}
\usepackage{amsfonts,amsmath,amssymb,mathtools}
\usepackage{verbatim}
\usepackage{listings}
\usepackage[ddmmyyyy]{datetime}
\usepackage{hyperref, url}
\usepackage{graphicx}
\usepackage{bussproofs}
\usepackage{multirow}
\usepackage{changepage}
\usepackage{bussproofs}	

\usepackage{svg}
\setsvg{inkscapeexe=inkscape}
\setsvg{inkscapeopt=-z -D}

\newcommand{\Ltac}{$\mathcal{L}$\unskip~tac}

\graphicspath{{Figuras/}}
\setbeamertemplate{frametitle continuation}{}

% suprimindo warnings do hyperref
\pdfstringdefDisableCommands{%
  \def\\{}%
  \def\texttt#1{<#1>}%
  \def\smallskip{}%
  \def\medskip{}%
}

\lstdefinestyle{haskell}{
    language=Haskell,
    backgroundcolor=\color{white},   				  % background color
    basicstyle=\ttfamily\small,      					% font style
    keywordstyle=\color{blue},       				  % keyword color
    commentstyle=\color{green!50!black}, 			% comment color
    stringstyle=\color{red},          				% string color
    numberstyle=\tiny\color{gray},    				% line number style
    stepnumber=1,                     				% number every line
    numbers=left,                     				% line numbers on the left
    numbersep=5pt,                    				% distance of line numbers
    frame=single,                     				% adds a frame around the code
    tabsize=2,                        				% sets default tabsize
    captionpos=b,                     				% sets the caption position
}

\renewcommand{\figurename}{Figura}
\renewcommand{\tablename}{Tabela}
\sloppy
\title{Inferência de Tipos para CPS}

\author[Vinícios Bidin]{
    Vinícios Bidin Santos\\\smallskip
    {\scriptsize Universidade do Estado de Santa Catarina \\\smallskip
    \texttt{vinibidin@gmail.com}\\\medskip
    {Orientador: Dr.~Cristiano Damiani Vasconcellos}\\
    {Coorientador: Me. Paulo Henrique Torrens}\\
    }
}

\date{\today}

\begin{document}
\begin{frame}
    \titlepage%
\end{frame}

\begin{frame}[allowframebreaks]{Sumário}
    \tableofcontents
\end{frame}

\section[]{Introdução}
\begin{frame}{Introdução}
    \textbf{Compilação:}
    \begin{itemize}
        \item Tradução de código de uma linguagem para outra
              \begin{itemize}
                  \item[--] Geralmente do código-fonte para o de máquina
              \end{itemize}
        \item Composta por diferentes etapas como:
              \begin{itemize}
                  \item[--] Análise léxica
                  \item[--] Análise sintática
                  \item[--] Análise semântica
                  \item[--] Otimizações
                  \item[--] Geração de código
              \end{itemize}
        \item Ligadas por Representações Intermediárias
              \begin{itemize}
                  \item[--] Principalmente nas otimizações~\cite{PLOTKIN1975125}
              \end{itemize}
    \end{itemize}
\end{frame}

\begin{frame}{Introdução}
    \textbf{Representações Intermediárias:}
    \begin{itemize}
        \item Linguagens imperativas
              \begin{itemize}
                  \item[--] Atribuição Única Estática (SSA)
              \end{itemize}
        \item Linguagens funcionais
              \begin{itemize}
                  \item[--] Forma Normal Administrativa (ANF)
                  \item[--] Estilo de Passagem de Continuação (CPS)
              \end{itemize}
        \item CPS
              \begin{itemize}
                  \item[--] Continuações explícitas
                        \begin{itemize}
                            \item[--] Parâmetro extra na função
                            \item[--] Funções sem retorno
                        \end{itemize}
                  \item[--] Otimizações
                        \begin{itemize}
                            \item[--] Eliminação da pilha de chamadas
                            \item[--] Eliminação de chamadas de cauda
                        \end{itemize}
              \end{itemize}
    \end{itemize}
\end{frame}

\subsection[]{Objetivos}
\begin{frame}{Objetivo}
    \begin{itemize}
        \item Formalizar um sistema de tipos para CPS
        \item Propor e implementar em Haskell um algoritmo de inferência de tipos para CPS
        \item Validar a implementação do algoritmo por meio do teste de inferência para expressões
    \end{itemize}
\end{frame}

\section[]{Representação Intermediária de Código}
\begin{frame}{Representação Intermediária de Código}
    \begin{itemize}
        \item A estrutura de dados gerada pelo compilador, que representa o programa fonte, é chamada de Representação Intermediária (do inglês, \textit{intermediate representation}, ou IR) \cite{cooper2014}.

    \end{itemize}
\end{frame}

\begin{frame}{Representação Intermediária de Código}
    \begin{itemize}
        \item Um compilador pode utilizar uma ou mais IRs, podendo variar desde uma estrutura em árvore ou grafo até alguma outra linguagem de programação, como C e representações de código funcional \cite{aho2008compilers}.

        \item []

        \item[] \begin{figure}
            \centering
            \includegraphics[width=.56\textwidth]{Figuras/compilador.png}
            \caption{Tipos de Compilador \cite{cooper2014}}
            \label{fig:comp}
        \end{figure} 
    \end{itemize}
\end{frame}
\subsection{Estilo de Passagem de Continuação (CPS)}
\begin{frame}{Estilo de Passagem de Continuação (CPS)}
    \begin{itemize}
        \item Técnica de transformação de código que torna o fluxo de controle explícito
              \begin{itemize}
                  \item[--] Chamadas de função passam o controle para a próxima etapa explicitamente, conhecida como continuação~\cite{appel1992compiling}
                  \item[--] Ao invés das funções retornarem o resultado da computação, é invocado uma continuação, representando o próximo passo
              \end{itemize}
        \item Toda chamada de função passa então a ser uma chamada de cauda (do inglês \textit{tail-call})
    \end{itemize}
\end{frame}

\begin{frame}{Estilo de Passagem de Continuação (CPS)}
  \textbf{Chamada de cauda:}
  \begin{itemize}
    \item Última instrução executada em uma função é uma chamada a outra função, sem que restem computações adicionais a serem feitas após essa chamada~\cite{MUCHNICK1997}
          \begin{itemize}
            \item[--] Função atual pode liberar seu quadro de ativação
          \end{itemize}
  \end{itemize}
  \textbf{Chamada não de cauda:}
  \begin{itemize}
    \item Ainda restam operações, como somas ou multiplicações, após a chamada da função
          \begin{itemize}
            \item[--] Função atual precisa manter seu quadro de ativação até que as operações sejam concluídas
          \end{itemize}
  \end{itemize}
\end{frame}

\begin{frame}{Estilo de Passagem de Continuação (CPS)}
  \begin{itemize}
    \item[] \begin{figure}
            \caption{Função fatorial em Haskell com chamada não de cauda}
            \lstinputlisting[style=haskell, label=code:factorial_non_tail_call]{Code/factorial_non_tail_call.hs}
            \small{Fonte: o autor}
          \end{figure}

    \item[] \begin{figure}
            \caption{Função fatorial em Haskell com chamada de cauda}
            \lstinputlisting[style=haskell, label=code:factorial_tail_call]{Code/factorial_tail_call.hs}
            \small{Fonte: o autor}
          \end{figure}
  \end{itemize}
\end{frame}

\begin{frame}{Estilo de Passagem de Continuação (CPS)}
  \textbf{Cálculo Lambda:}
  \citeonline{church1932set} define o cálculo-$\lambda$, que é representado pela seguinte gramática:

  \begin{equation}
    e ::= x \mid \lambda x. e \mid e e\nonumber
  \end{equation}

  \begin{itemize}
    \item \textbf{Variável:} identificadores no sistema
    \item \textbf{Abstração:} função que associa um identificador $x$ a um termo $e$
    \item \textbf{Aplicação:} aplicação de um termo a outro
  \end{itemize}
\end{frame}

\begin{frame}{Estilo de Passagem de Continuação (CPS)}
  Variáveis no cálculo-$\lambda$ podem ser:
  \begin{itemize}
    \item \textbf{Livres:} quando não estão associadas a uma abstração de função
          \begin{itemize}
            \item[--] $\lambda x. y$
          \end{itemize}
    \item \textbf{Ligadas:} quando estão associadas a uma abstração de função
          \begin{itemize}
            \item[--] $(\lambda x. x) y$
          \end{itemize}
    \item[]
  \end{itemize}

  Para analisar expressões:
  \begin{itemize}
    \item \textbf{$\alpha$-redução:} Renomeação de variáveis ligadas.
          \begin{align}
            \lambda x . e[x] & \rightarrow \lambda y . e[y]\nonumber
          \end{align}

    \item \textbf{$\beta$-redução:} Aplicação de função.
          \begin{align}
            (\lambda x . e_1) e_2 & \rightarrow e_1 [e_2 / x]\nonumber
          \end{align}

    \item \textbf{$\eta$-redução:} Expansão de função.
          \begin{align}
            \lambda x . (e \, x) & \rightarrow e \quad \text{se } x \text{ não ocorre livre em } e\nonumber
          \end{align}
  \end{itemize}
\end{frame}


\section[]{Teoria de Tipos}
\begin{frame}{Teoria de Tipos}
    % \begin{itemize}
    %     \item \citeonline{appel1998ssa} demonstra a equivalência entre código funcional e uma representação na forma SSA

    %     \item []

    %     \item [] \begin{figure}
    %         \centering
    %         \includegraphics[width=.8\textwidth]{Figuras/ssa-funcional.png}
    %         \caption{Representações Equivalentes \cite{appel1998ssa}}
    %         \label{fig:conv1}
    %     \end{figure}
    % \end{itemize}
\end{frame}


\section[]{Sistema Damas-Milner}
% \chapter{Sistema Damas Milner}\label{ch:damas-milner}

% O sistema Damas-Milner é um sistema de tipos para o cálculo lamda, com polimorfismo paramétrico criado por Hindley, Damas e Milner para a linguagem de programação de propósito geral ML \cite{HINDLEY1969, MILNER1978, DAMAS1984}.
% Conforme destacado por \citeonline{DAMAS1982}, a linguagem ML possuia fortes aspectos como flexibilidade (através do polimorfismo), robustez (devido à \textit{soundness} semântica) e fortes mecanismos de detecção de erros em tempo de compilação.
% Este equilíbrio, permite que linguagens funcionais como Haskell e OCaml herdem suas características e usem este sistema como base em seus sistemas de tipos.

% \section{Algoritmo W}\label{sec:w-algo}

\chapter{Sistema Damas-Milner}\label{ch:damas-milner}

O sistema Damas-Milner, introduzido por Robin Milner e posteriormente formalizado em maior detalhe por Luis Damas \cite{MILNER1978, DAMAS1982}, é um dos sistemas de tipos mais influentes para linguagens funcionais.
Este sistema tem como principal característica a inferência automática de tipos polimórficos, sem a necessidade de anotações explícitas por parte do programador, como ocorre em linguagens como ML, Haskell e OCaml.
A sua base é o cálculo lambda com polimorfismo paramétrico, permitindo que funções possam operar sobre múltiplos tipos de maneira genérica.

A introdução do sistema Damas-Milner trouxe duas contribuições principais: a definição de um sistema de tipos que é robusto, garantindo propriedades como \textit{soundness} (consistência semântica), e a criação de um algoritmo, o Algoritmo W, capaz de inferir o tipo mais geral (também chamado de \textit{principal type-scheme}), conforme demonstrado em \citeonline{DAMAS1984}.
Como resultado, a linguagem ML e suas derivadas se tornaram notórias por fornecer ao programador a capacidade de escrever programas sem erros de tipo detectáveis durante a compilação, permitindo um desenvolvimento mais seguro e robusto \cite{MILNER1978, DAMAS1984}.

A sintaxe do sistema DM define as expressões e os tipos usados no processo de inferência.
Abaixo, segue a gramática das expressões e tipos:

\begin{equation}\label{eq:dm-syntax}
  \begin{array}{ll}
    \text{Variáveis}         & x                                                                                 \\
    \text{Expressões}        & e ::= x \mid e \ e' \mid \lambda x.e \mid \texttt{let} \ x = e \ \texttt{in} \ e' \\
    \\
    \text{Variáveis de tipo} & \alpha                                                                            \\
    \text{Tipos primitivos}  & \iota                                                                             \\
    \text{Tipos}             & \tau ::= \alpha \mid \iota \mid \tau \rightarrow \tau
    \alpha                                                                                                       \\
    \text{Schemes}           & \sigma ::= \forall \alpha. \sigma \mid \tau                                       \\
  \end{array}
\end{equation}

Na sintaxe, $x$ representa variáveis que podem ser nomes de qualquer identificador, e $e$ descreve expressões que podem ser variáveis, aplicações de função, funções anônimas ou declarações \texttt{let}, que introduzem polimorfismo através de generalização de tipos.
As variáveis de tipo $\alpha$ são usadas para representar tipos genéricos.
Os tipos primitivos $\iota$ são usados para representar tipos constantes.
Tipos $\tau$ podem ser tanto variáveis de tipo quanto funções entre tipos.
Por fim, $\sigma$ denota \textit{schemes}, ou tipos polimórficos, que podem quantificar variáveis de tipo, permitindo reutilização de tipos genéricos em diferentes contextos.

O polimorfismo no sistema Damas-Milner é introduzido pelas expressões \texttt{let}, que permitem a generalização de tipos. Ao declarar uma variável ou função usando \texttt{let}, o tipo inferido é generalizado para ser reutilizado em diferentes contextos. Isso significa que, ao declarar uma função como $\texttt{id} = \lambda x.x$, o sistema deduz o tipo mais geral $\forall \alpha. \alpha \rightarrow \alpha$, que pode ser usado com diferentes tipos conforme necessário.

A inferência de tipos envolve dois processos principais: generalização e instanciação.
A generalização ocorre quando o sistema identifica que uma expressão pode ser tipada com um tipo mais geral, permitindo que seja reutilizada de maneira polimórfica.
Já a instanciação ocorre quando um tipo polimórfico é aplicado a um tipo concreto, especializando-o para um uso específico.
Esse mecanismo garante a flexibilidade do sistema, ao mesmo tempo que mantém a segurança garantida pela inferência de tipos.

Por exemplo, considere a expressão $\texttt{let id} \ = \lambda x.x \ \texttt{in} \ (\texttt{id} \ \texttt{1}, \ \texttt{id} \ \texttt{`a'})$.
O sistema generaliza o tipo de $\texttt{id}$ para $\forall \alpha. \alpha \rightarrow \alpha$, e instancia este tipo tanto para inteiros quanto para caracteres nas duas aplicações subsequentes.

% Substituições

% Variáveis livres

% Regras de inferência

\section{Algoritmo W}\label{sec:w-algo}

% Algoritmo de unificação

% Algoritmo W

\subsection{Algoritmo W}
\newcommand{\defas}{\ensuremath{\overset{def}{=}}}
\newcommand{\fv}{\ensuremath{\text{FV}}}
\newcommand{\fvc}{\ensuremath{\text{FVC}}}
\newcommand{\eeq}{\ensuremath{\overset{e}{=}}}
\newcommand{\Append}{\ensuremath{\texttt{++}}}
\newcommand{\If}{\ensuremath{\text{se}}}
\newcommand{\Let}{\ensuremath{\text{let}}}
\newcommand{\In}{\ensuremath{\text{in}}}
\newcommand{\Then}{\ensuremath{\text{então}}}
\newcommand{\Return}{\ensuremath{\text{retorna}}}
\newcommand{\Else}{\ensuremath{\text{senão}}}
\newcommand{\Elseif}{\ensuremath{\text{senão se}}}
\newcommand{\Fail}{\ensuremath{\text{falha}}}
\newcommand{\Unify}{\ensuremath{\textit{unify}}}
\newcommand{\Occurs}{\ensuremath{\textit{occurs}}}
\newcommand{\True}{\ensuremath{\texttt{Verdadeiro}}}
\newcommand{\False}{\ensuremath{\texttt{Falso}}}
\newcommand{\Whitespace}{\ensuremath{\texttt{ }}}
\newcommand{\TODO}[1]{\textcolor{red}{\textbf{TODO:} #1}}

\begin{frame}{Algoritmo W}
    Algoritmo de inferência introduzido por~\citeonline{DAMAS1982}:
    \begin{itemize}
        \item Para linguagens funcionais
        \item[]
        \item Algoritmo eficiente
              \begin{itemize}
                  \item[$\blacktriangleright$] Na maioria dos casos
              \end{itemize}
        \item[]
        \item Unificação
              \begin{itemize}
                  \item[$\blacktriangleright$] Solucionar equações de tipos
                  \item[$\blacktriangleright$] Dados dois tipos $\tau_1$ e $\tau_2$
                  \item[$\blacktriangleright$] A unificação procura uma substituição $S$ tal que $S\tau_1 = S\tau_2$
                  \item[$\blacktriangleright$] Atribui os tipos mais gerais possíveis
              \end{itemize}
    \end{itemize}
\end{frame}


\section[]{Proposta}
\begin{frame}{Proposta}
    Sistema de tipos monomórfico para CPS com um único construtor (chamado de negação poliádica) para representar continuações \cite{TORRENS2024}:

    \[
        \textit{Types } \ \tau \ ::= \ \neg \vec{\tau} \ \mid \ X
    \]
    \[
        \textit{Environments } \ \Gamma ::= \cdot \mid \Gamma, x : \tau
    \]

    \[
        \frac{
            \Gamma(k) = \neg \vec{\tau}
            \quad \quad
            \Gamma(\vec{x}) = \vec{\tau}
        }{
            \Gamma \vdash k\langle \vec{x} \rangle
        } \quad (J)
    \]

    \[
        \frac{
            \Gamma, k : \neg \vec{\tau} \vdash b
            \quad \quad
            \Gamma, \vec{x} : \vec{\tau} \vdash c
        }{
            \Gamma \vdash b \{ k \langle \vec{x} \rangle = c \}
        } \quad (B)
    \]

    \begin{itemize}
        \item Propor uma extensão ao sistema de tipos
              \begin{itemize}
                  \item[$\blacktriangleright$] Suporte a tipos polimórficos
                  \item[$\blacktriangleright$] Algoritmo de inferência de tipos
              \end{itemize}
    \end{itemize}
\end{frame}

\begin{frame}{Cronograma}
    % Para o TCC2 as seguintes etapas ficam definidas:
    % \begin{enumerate}
    %     \item Formalização da tradução de código funcional com sistema de efeitos para mônadas
    %     \item Implementação da tradução formalizada na segunda etapa
    %     \item Análise dos resultados
    %     \item Escrita do texto
    % \end{enumerate}
    % \begin{table}[htp]
    %     \centering
    %     \noindent \begin{tabular}{|c|c|c|c|c|c|c|c|}
    %         \hline
    %         \multirow{2}{*}{\textbf{\small{Etapas}}} & \multicolumn{1}{|c|}{\textbf{\small{2023/1}}} &
    %         \multicolumn{5}{|c|}{\textbf{\small{2023/2}}} \\
    %         \cline{2-7}
    %         &\textbf{Dez} & \textbf{Fev} & \textbf{Mar} & \textbf{Abr} & \textbf{Mai} & \textbf{Jun} \\
    %         \hline
    %         \textbf{\small{1}}  & \cellcolor{gray} & \cellcolor{gray} & \cellcolor{gray} & & & \\
    %         \hline
    %         \textbf{\small{2}}  & & \cellcolor{gray} & \cellcolor{gray} & \cellcolor{gray} & & \\
    %         \hline
    %         \textbf{\small{3}}  & & & \cellcolor{gray} & \cellcolor{gray} & \cellcolor{gray} & \\
    %         \hline
    %         \textbf{\small{4}}  & & & \cellcolor{gray} & \cellcolor{gray} & \cellcolor{gray} & \cellcolor{gray}\\
    %         \hline
    %         \end{tabular}
    %     \caption[Cronograma Proposto para o TCC2]{Cronograma Proposto para o TCC2}
    %     \label{tab:cronograma}
    % \end{table}
\end{frame}


\section[]{Referências}
\begin{frame}[allowframebreaks]{Referências}
    \bibliography{referencias}
\end{frame}

\end{document}