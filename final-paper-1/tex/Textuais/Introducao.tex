\chapter{Introdução}\label{sec:introducao}

A compilação de programas envolve diversas fases, cada uma com funções específicas, como análise léxica, análise sintática, análise semântica, otimizações, e, finalmente, a geração de código.
Uma etapa crítica nesse processo é a otimização, que frequentemente se baseia em representações intermediárias (IRs).
Essas representações atuam como ponte entre o código fonte e o código de máquina, permitindo que transformações e otimizações sejam aplicadas de maneira mais eficaz~\cite{PLOTKIN1975125}.

As representações intermediárias variam conforme o paradigma da linguagem de programação.
Para linguagens imperativas, a Representação em Atribuição Única Estática (SSA) é amplamente adotada.
Já em linguagens funcionais, a Forma Normal Administrativa (ANF) e o Estilo de Passagem de Continuação (CPS) se destacam.
Este trabalho foca especificamente no CPS, uma IR que oferece vantagens particulares em termos de otimização e simplicidade na geração de código.

Essas características do CPS se tornam ainda mais evidentes quando comparamos como diferentes linguagens de lidam com o fluxo de execução. Em linguagens de alto nível, por exemplo, a pilha de chamadas atua como uma abstração fundamental para gerenciar o controle de retorno das funções. No entanto, em linguagens de baixo nível, como \textit{assembly}, não há tal abstração, exigindo que o controle do fluxo seja manualmente tratado por meio de endereços de retorno.

Nesse contexto, o CPS se destaca ao tornar as continuações explícitas no código.
Ao invés de confiar na pilha de chamadas para gerenciar retornos, o CPS introduz um parâmetro adicional em cada função, representando a continuação — isto é, o que deve ser feito com o resultado da função~\cite{KENNEDY2007}.
Desta forma, em vez de simplesmente retornar um valor diretamente, a função invoca essa continuação, transferindo explicitamente o controle à próxima etapa da computação.
Isso elimina a dependência da pilha de chamadas, simplificando o modelo de execução e tornando-o mais alinhado com as necessidades de linguagens de baixo nível.

Além disso, a adoção do CPS como representação intermediária vai além da tradução de linguagens de alto nível para código de máquina.
O CPS facilita a aplicação de otimizações avançadas, como a eliminação de chamadas de cauda e a fusão de funções, além de permitir uma correspondência mais direta com o código gerado em linguagens de montagem~\cite{FLANAGAN1993}.

Por outro lado, um ponto importante a ser considerado é que muitas implementações do CPS optam por uma representação não tipada~\cite{MORRISSET1999}. Embora essa abordagem simplifique a implementação inicial, ele pode comprometer a segurança e a correção do código.
Um sistema de tipos robusto pode não apenas garantir a correção de certas transformações e otimizações, mas também identificar uma classe inteira de erros antes da execução, proporcionando assim maior confiabilidade ao processo de compilação.

Diante dessas considerações, a linguagem de programação Haskell foi escolhida para a implementação deste trabalho.
Haskell, sendo uma linguagem funcional pura fortemente tipada, possui características desejáveis, como transparência referencial~\cite{SONDERGAARD1990} e um sistema de tipos robusto para explorar as vantagens do CPS e aplicar o sistema de tipos de maneira rigorosa.
Dessa forma, a escolha de Haskell não apenas facilita o desenvolvimento de uma implementação segura e eficiente do CPS, como também conta com garantias de seguranças que são fundamentais para o sucesso deste trabalho.

\section{Objetivo Geral}\label{sec:objetivo-geral}

Este trabalho visa explorar a inferência de tipos para CPS, propondo uma extensão do algoritmo W --- um algoritmo de inferência de tipos para o sistema Damas-Milner --- para CPS, definido em~\cite{TORRENS2024} na linguagem de programação Haskell.

\section{Objetivos específicos}\label{sec:objetivos-especificos}

\begin{itemize}
  \item Formular uma extensão do algoritmo W para CPS;
  \item Implementar essa extensão em Haskell;
  \item Validar a implementação do algoritmo por meio do teste de inferência para expressões. Se possível, realizar a geração de programas para verificação de que o algoritmo infere corretamente os tipos a eles.
\end{itemize}

\section{Metodologia}\label{sec:metodologia}

A metodologia deste trabalho consistirá em duas principais etapas: pesquisa bibliográfica e implementação.
A primeira etapa envolve uma extensa revisão de literatura sobre continuações e seu cálculo, bem como um aprofundamento do sistema Damas-Milner, com o objetivo de proporcionar uma compreensão completa ao autor.
A segunda etapa trata da formulação do algoritmo W estendido para o cálculo de continuações, junto com sua implementação.

Para validar a implementação, será testado a inferência do algoritmo com testes.
Caso seja possível, será feito ainda a geração de programas de teste, nos quais a inferência de tipos será verificada.
A validação consistirá em comparar os tipos inferidos pelo algoritmo com os tipos esperados para esses programas.

\section{Estrutura do Trabalho}\label{sec:estrutura-trabalho}

Este trabalho está estruturado de maneira que será fornecida, antes da formalização da extensão e implementação do algoritmo W, nos Capítulos~\ref{ch:formalizacao} e~\ref{ch:implementacao}, uma fundamentação teórica.
Inicialmente, será abordada a representação intermediária no Capítulo~\ref{ch:IR}, com um aprofundamento em CPS na Seção~\ref{sec:cps}.
Na sequência, o Capítulo~\ref{ch:type-theory} discutirá a teoria de tipos, apresentando os conceitos e definições necessários para a compreensão do tema.
O sistema Damas-Hindley-Milner será detalhado no Capítulo~\ref{ch:damas-milner}, incluindo o algoritmo W, discutido de forma mais específica na seção~\ref{sec:w-algo}.
Por fim, o Capítulo~\ref{ch:conclusao} trará as considerações finais, sintetizando os principais pontos e contribuições do trabalho.
